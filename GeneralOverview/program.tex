\documentclass[a4paper,12pt]{article}
\usepackage[unicode,verbose]{hyperref}
\usepackage{amsmath,amssymb,amsthm} \usepackage{pb-diagram}
\usepackage{ucs}
\usepackage[utf8x]{inputenc}
\usepackage[russian]{babel}
\usepackage{cmap}
\usepackage{graphicx}
\pagestyle{plain}
\theoremstyle{definition} \newtheorem{Def}{Definition}
\begin{document}
\section{Программа}
Программа кандидатского экзамена по теоретической физике (01.04.02), специальная
часть, по кафедре физики высоких энергий и элементарных частиц
аспиранта 2-го года обучения Назарова Антона Андреевича.

\subsection{Интегируемые системы}
\label{sec:-}

Введение в квантовые интегрируемые системы и метод квантовой обратной задачи рассеяния. 

Литература:
  \begin{itemize}
  \item Faddeev L.D. How Algebraic Bethe Ansatz works for integrable model. Arxiv preprint hep-th/9605187.
  \item Korepin, VE and Bogoliubov, NM and Izergin, AG. Quantum inverse scattering method and correlation functions. Cambridge Univ Pr 1997.
  \end{itemize}


1.Квантовые интегрируемые системы. Анзац Бете. Квантовый метод обратной задачи рассеяния.

2.Матрица монодромии. Интегралы движения.


\subsection{Алгебры Ли}
\label{sec:Lie}
Общие вещи про конечномерные и бесконечномерные алгебры Ли, книжек множество, все что надо есть в 
 \cite{kac1988modular}, \cite{difrancesco1997cft}, 

3.Алгебры Ли. Универсальная обертывающая алгебра. Теория представлений.

4.Алгебры Каца-Муди. Аффинные алгебры Ли. Алгебра Вирасоро.

\subsection{Квантовые группы}
\label{sec:Quantum-groups}

Общая теория по квантовым группам (без приложений).
  \begin{itemize}
  \item Etingof, P. and Schiffmann, O. Lectures on quantum groups. International press 1997
  \item Chari V., Pressley A. A guide to quantum groups. Cambridge Univ Press 1995.
  \item Majid S. Foundations of quantum group theory. Cambridge Univ Press 2000.
  \end{itemize}

5.Квантовые группы. Алгебры Хопфа. 

6.Квантование Дринфельда-Джимбо.

7.Уравнение Янга-Бакстера. R-матрица.

\subsection{Теория струн, конформная теория поля}
\label{sec:CFT}

Введение в теорию струн и CFT, приложение теории из раздела \ref{sec:Lie}.

Литература:
Zweibach B. A first course in string theory. Cambridge Univ Press 2004, \cite{difrancesco1997cft}, \cite{Walton:1999xc}, \cite{gaberdiel2000icf}

9.Теория струн. Действие струны Намбу, Полякова.

10.Критические струны и двумерная конформная теория поля.

11.Двумерная конформная теория поля. Конформная инвариантность. Уравнения Книжника-Замолодчикова.

12.Модели Весса-Зумино-Новикова-Виттена в двумерной конформной теории поля.

13.Алгебра токов в моделях Весса-Зумино-Новикова-Виттена. Конструкция Сугавары.


\subsection{AdS/CFT}
\label{sec:adscft}

Приложение всей предыдущей теории к актуальным современным проблемам :)

Литература:
\cite{Aharony:1999ti}, \cite{D'Hoker:2002aw}, Minahan, J.A. and Zarembo, K. The Bethe-ansatz for= 4 super Yang-Mills. //Journal of High Energy Physics, v 2003, 2003, \cite{Maldacena:2000hw,Maldacena:2000kv,Maldacena:2001km,Maldacena:2001ky}

14.Суперсимметричная теория Янга-Миллса. Модель N=4. Конформная инвариантность.

8.Интегрируемые системы в N=4 SYM.

15.AdS/CFT-соответствие.

16.Интегрируемость в суперсимметричной калибровочной теории поля.

17.Струны на пространстве анти-Де-Ситтера и модели Весса-Зумино-Виттена.

\bibliography{program}{}
\bibliographystyle{utphys}
\end{document}
