\documentclass[a4paper,12pt]{article}
\usepackage[unicode,verbose]{hyperref}
\usepackage{amsmath,amssymb,amsthm} \usepackage{pb-diagram}
\usepackage{ucs}
\usepackage[utf8x]{inputenc}
\usepackage[russian]{babel}
\usepackage{cmap}
\usepackage{graphicx}
\pagestyle{plain}
\theoremstyle{definition} \newtheorem{Def}{Definition}
\begin{document}
\section{Интегрируемые системы}
\label{sec:IntSys}

\subsection{Интегируемые системы}
\label{sec:-}

Введение в квантовые интегрируемые системы и метод квантовой обратной задачи рассеяния. 

Литература:
  \begin{itemize}
  \item Faddeev L.D. How Algebraic Bethe Ansatz works for integrable model. Arxiv preprint hep-th/9605187.
  \item Korepin, VE and Bogoliubov, NM and Izergin, AG. Quantum inverse scattering method and correlation functions. Cambridge Univ Pr 1997.
  \end{itemize}


1.Квантовые интегрируемые системы. Анзац Бете. Квантовый метод обратной задачи рассеяния.

2.Матрица монодромии. Интегралы движения.
\bibliography{program}{}
\bibliographystyle{utphys}
\end{document}
