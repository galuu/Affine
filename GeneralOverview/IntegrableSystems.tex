\documentclass[a4paper,12pt]{article}
\usepackage[unicode,verbose]{hyperref}
\usepackage{amsmath,amssymb,amsthm} \usepackage{pb-diagram}
\usepackage{ucs}
%\usepackage[utf8x]{inputenc}
%\usepackage[russian]{babel}
\usepackage{cmap}
\usepackage{graphicx}
\pagestyle{plain}
\theoremstyle{definition} \newtheorem{Def}{Definition}
\newenvironment{comment}
{\par\noindent{\bf Comment}\\}
{\\\hfill$\scriptstyle\blacksquare$\par}

\newtheorem{statement}{Statement}
\theoremstyle{definition} \newtheorem{Def}{Definition}
\newcommand{\go}{\overset{\circ }{\frak{g}}}
\newcommand{\ao}{\overset{\circ }{\frak{a}}}
\newcommand{\co}[1]{\overset{\circ }{#1}}

\begin{document}

\title{Introduction to quantum integrable systems and quantum groups}

\begin{abstract}
  We review quantum integrable systems and discuss the connection with the representation theory of finite-dimensional and affine Lie algebras. Then we consider the quantum groups which are constructed as the deformations of the universal enveloping algebras.
\end{abstract}

\section{Introduction. Historical overview.}
\label{sec:intr-hist-overv}


\bibliography{program}{}
\bibliographystyle{utphys}
\end{document}
