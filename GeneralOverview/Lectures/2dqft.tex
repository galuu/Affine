\documentclass[a4paper,12pt]{article}
\usepackage[utf8x]{inputenc}
\usepackage[russian]{babel}
\usepackage{ucs}
\usepackage[unicode,verbose]{hyperref}
\usepackage{amsmath,amssymb,amsthm}
\usepackage{pb-diagram}
\usepackage{multicol}
\usepackage{cmap}
\usepackage{color}
%\usepackage{graphicx}
%\usepackage{epstopdf}
\pagestyle{plain}

%\usepackage{verbatim} 
\newenvironment{comment}
{\par\noindent{\bf TODO}\\}
{\\\hfill$\scriptstyle\blacksquare$\par}

\newtheorem{statement}{Statement}
\newtheorem{lemma}{Lemma}
\newtheorem{theorem}{Theorem}
\theoremstyle{definition}
\newtheorem{corollary}{Corollary}[theorem]
\theoremstyle{definition}
\newtheorem{mynote}{Note}[section]
\theoremstyle{definition}
\newtheorem{definition}{Definition}
\newcommand{\go}{\stackrel{\circ }{\mathfrak{g}}}
\newcommand{\ao}{\stackrel{\circ }{\mathfrak{a}}}
\newcommand{\co}[1]{\stackrel{\circ }{#1}}
\newcommand{\pia}{\pi_{\mathfrak{a}}}
\newcommand{\piab}{\pi_{\mathfrak{a}_{\bot}}}
\newcommand{\af}{\mathfrak{a}}
\newcommand{\afb}{\mathfrak{a}_{\bot}}


\title{Лекции по теории поля в двух измерениях\\
\small{Академический университет, кафедра теоретической физики}
}
\author{Антон Назаров\\
  \small{СПбГУ, физический факультет}\\
  \small{кафедра физики высоких энергий и элементарных частиц}\\
  \texttt{antonnaz@gmail.com}
}
\date{Осенний семестр 2010 года}

\begin{document}
\maketitle
\thispagestyle{empty}
\begin{abstract}
  Текст представляет собой конспект лекций по квантовой теории поля в двух измерениях. Лекции читаются в осеннем семестре 2010 года в Академическом университете для магистрантов 6 курса, группа теории твердого тела. 
\end{abstract}
\tableofcontents
\section{Программа курса}
\label{sec:program}
\begin{enumerate}
\item Введение.
  \begin{itemize}
  \item Двумерные модели
  \item Фазовые переходы, критические индексы
  \item Универсальность
  \item Методы квантовой теории поля в статистической физике
  \end{itemize}
\item Перенормировки, ренормгруппа
\item Решение модели Изинга
  \begin{itemize}
  \item Интегрируемость
  \item Подход Каданова
  \end{itemize}
\item Конформная инвариантность
  \begin{itemize}
  \item Глобальная конформная инвариантность
  \item Ограничения на корреляционные функции
  \item Локальная конформная инвариантность в двух измерениях
  \item Алгебра Витта и алгебра Вирасоро.
  \end{itemize}
\item Конформная теория поля
  \begin{itemize}
  \item Примарные и вторичные поля
  \item Минимальные модели
  \end{itemize}
\end{enumerate}

\subsection{Литература}
\label{sec:literature}
В первых лекциях \ref{sec:lecture-1},\ref{sec:lecture-2} мы опираемся на вводные главы книги \cite{difrancesco1997cft}. Подробно решение  модели Изинга в двух измерениях рассмотрено в книге  \cite{belavin2001lect}. Последовательное изложение техники функционального интеграла в статистической физике можно найти в книге А.Н. Васильева \cite{vasiliev1976}.

В лекции 3 \ref{sec:lecture-3} мы обсуждаем применение ренормгруппы к описанию критического поведения, более подробно и достаточно педагогически этот вопрос рассматривается в книге \cite{ma1980} (\cite{ma2000modern}). Исчерпывающую информацию содержит моногорафия \cite{vasiliev1998} (английский вариант \cite{Vasilev:1027193}).

Изложение конформной теории поля в двух измерениях опирается на оригинальную статью \cite{belavin1984ics}, книгу \cite{difrancesco1997cft} и обзор \cite{zamolodchikov1989rus,zamolodchikov1989conformal}, недавно переизданный в виде книги.

\section{Лекция 1}
\label{sec:lecture-1}

\subsection{Введение}
\label{sec:intro}
Чем интересны двумерные системы? Во-первых, они существуют в природе. Например, критическое поведение на поверхности металлов напоминает поведение модели Изинга в двух измерениях \cite{campuzano1985110}. Во-вторых, в двух измерениях даже в простых моделях есть фазовые переходы. Это не так, например,  в одномерной модели Изинга. В простых двумерных моделях можно вычислить критические индексы и изучать поведение системы при фазовых переходах. Гипотеза универсальности утверждает, что все системы в критической точке разбиваются на небольшое число классов. Системы из одного класса ведут себя одинаково, имеют одинаковые критические индексы. Благодаря гипотезе универсальности существует лишь небольшое число типов критического поведения, поэтому значения индексов, вычисленные теоретически в простых моделях соответствуют гораздо более сложным реальным системам.
В критической точке наблюдается масштабная инвариантность, что ведет к конформной инвариантности \cite{Polyakov:1970xd}. В двух измерениях конформная инвариантность дает очень много сведений о системе, так как алгебра конформных преобразований бесконечномерна. В результате поведение системы может быть описано строго математически методами двумерной конформной теории поля. Двумерная конформная теория поля имеет и другое применение, не связанное с описанием фазовых переходов --- это теория струн, которая считается перспективным кандидатом на роль квантовой теории гравитации.

\subsection{Фазовые переходы. Основные понятия}
\label{sec:phase-transitions}

Напомним некоторые основные понятия теории фазовых переходов на примере двумерной модели Изинга.

Мы рассматриваем статистические системы. Все макроскопические характеристики таких систем вычисляются из статсуммы, которая связана с микроскопическим описанием системы. Для больцмановского распределения вероятность системы находиться в состоянии с номером $i$ с энергией $E_i$
\begin{equation}
  \label{eq:2}
  P_i=\frac{1}{Z}e^{-\beta E_i}, \quad \beta=\frac{1}{T}
\end{equation}
Мы используем систему единиц, в которой постоянная Больцмана равна единице $k_B=1$, введем $\beta=\frac{1}{T}$. Статсумма:
\begin{equation}
  \label{eq:3}
  Z=\sum_i e^{-\beta E_i}.
\end{equation}
Свободная энергия
\begin{equation}
  \label{eq:4}
  F=-T\ln Z
\end{equation}
Внутренняя энергия
\begin{equation}
  \label{eq:5}
  U=-\frac{1}{Z}\frac{\partial Z}{\partial \beta}=-T^2 \frac{\partial}{\partial T}\left(\frac{F}{T}\right)
\end{equation}
Телоемкость равна производной внутренней энергии по температуре при заданном объеме
\begin{equation}
  \label{eq:6}
  C=\left(\frac{\partial U}{\partial T}\right)_V=-T\frac{\partial^2 F}{\partial T^2}
\end{equation}
Таким образом статсумма является производящей функцией всех термодинамических величин. Вычисление статсуммы в реальных системах --- это сложная задача.
Макроскопическое описание имеет смысл только в термодинамическом пределе, когда число частиц стремится к бесконечности $N\to \infty$.

Двумерная модель Изинга --- это простейшая модель магнетика. Она формулируется на решетке, которую мы будем, для простоты, считать прямоугольной. Вершины решетки нумеруются латинскими индексами $i,j$. В вершинах решетки находятся частицы со спинами $\sigma_j$ равными $\pm 1$ (вверх или вниз). Взаимодействуют только ближайшие соседи, константу взаимодействия обозначим через $J$. Для энергии системы имеем
\begin{equation}
  \label{eq:1}
  E=J\sum_{\left<i,j\right>} \sigma_i\cdot \sigma_j-h\sum_i \sigma_i
\end{equation}
Суммирование в первом члене ведется только по парам ближайших соседей, второй член --- это взаимодействие со внешним полем $h$.
Пусть система состоит из $N$ вершин. Она имеет $2^N$ различных конфигураций. Случай $J>0$ соответствует ферромагнетику, $J<0$ --- антиферромагнетику. 
При $h=0$ низшее энергетическое состояние двукратно вырождено --- это состояние, в котором все спины направлены вверх или вниз. 

Существует несколько точных решений двумерной модели Изинга в отсутствие внешнего поля. Решений со внешним полем и в большем числе измерений пока не известно. Недавнее достижение Станислава Смирнова, за которое он получил в этом году премию Филдса, непосредственно связано с темой наших лекций. В своих работах Станислав Смирнов впервые строго доказал наличие конформного предела в двумерной модели Изинга \cite{smirnov2007conformal,smirnov2006towards,smirnov2001critical}.
Фазовые переходы в модели Изинга наблюдаются только в этом пределе, при конечных размерах системы никаких переходов нет.

При исследовании магнетиков нас интересуют следующие термодинамически величины.
Намагниченность, которая равна среднему значению спина по всем конфигурациям $[\sigma]$:
\begin{equation}
  \label{eq:7}
  M=\left<\sigma_j\right>=\frac{1}{NZ}\sum_{[\sigma]}\left(\sum_i\sigma_i\right)e^{-\beta E[\sigma]}=-\frac{1}{N}\frac{\partial F}{\partial h}.
\end{equation}
Магнитная восприимчивость
\begin{equation}
  \label{eq:8}
  \chi=\left.\frac{\partial M}{\partial h}\right|_{h=0}=\frac{1}{N}\frac{\partial}{\partial h}\left(\frac{1}{Z}\sum_{[\sigma]}\left(\sum_i \sigma_i\right)e^{-\beta E[\sigma]}\right)=\frac{1}{NT}\left(\left<\sigma_{\mathrm{tot}}^2\right>-\left<\sigma_{\mathrm{tot}}\right>^2\right)
\end{equation}
Здесь мы ввели обозначение $\sigma_{\mathrm{tot}}=\sum_i \sigma_i$. Видно, что магнитная восприимчивость пропорциональна дисперсии полного спина. Введем парную корреляционную функцию
\begin{equation}
  \label{eq:9}
  \Gamma(i-j)=\left<\sigma_i-\sigma_j\right>.
\end{equation}
Из-за вращательной и трансляционной инвариантности $\Gamma$ зависит только от расстояния между вершинами $\left|i-j\right|$. Связная корреляционная функция
\begin{equation}
  \label{eq:10}
  \Gamma_c(i-j)=\left<\sigma_i-\sigma_j\right>_c=\left<\sigma_i\sigma_j\right>-\left<\sigma_i\right>\left<\sigma_j\right>.
\end{equation}
Можно показать, что
\begin{equation}
  \label{eq:11}
  \chi=\beta\sum_i\Gamma(i)_c.
\end{equation}
То есть магнитная восприимчивость --- это мера статистической когерентности системы, она растет с ростом зависимости спинов между собой.

\subsection{Классические статистические модели. Связь с теорией поля.}
\label{sec:statistical-models-qft}

Распределение Больцмана инвариантно относительно сдвига энергии на константу, поэтому можно переписать гамильтониан в более удобном для обобщения виде:
\begin{equation}
  \label{eq:12}
  \begin{array}{l}
    \sigma_i\sigma_j=2\delta_{\sigma_i\sigma_j}-1\\
    E[\sigma]=-2J\sum_{\left<ij\right>}\delta_{\sigma_i\sigma_j}-h\sum_i \sigma_i
  \end{array}
\end{equation}
Обсудим некоторые обобщения модели Изинга и проиллюстрируем сходство статистической механики с теорией поля.
Если $\sigma_i$ может принимать значения не $-1,1$, а $1,2\dots,q$, то мы получаем модель Поттса с $q$ состояниями.
Подобным образом получаются и модели Ашкина-Теллера.
Другой класс дискретных моделей отличается более существенно. Если в модели Изинга переменные (спины) живут в вершинах, а энергия --- на ребрах, то в вершинных моделях все наоборот. В качестве переменных берутся направленные ребра (стрелки), а энергия в вершине зависит от числа входящих и выходящих из нее стрелок. Пусть ребра, входящие в вершину имеют значения $0,1$ в зависимости от направления стрелки. Тогда энергия вершины обозначается через $R_{\alpha\mu}^{\beta\nu},\; \alpha,\beta,\mu,\nu=0,1$. Бывает 16 таких членов. В зависимости от того, сколько из них не равны нулю выделяют 6-вершинную и 8-вершинную модели. 8-вершинная модель --- ненулевой вклад только когда входит четное число стрелок. 

Другой класс статистических моделей --- это модели с непрерывными степенями свободы. Например, заменим спины $\sigma_i$ в \eqref{eq:1} на единичные вектора в $m$-мерном пространстве $\vec n_i$.
\begin{equation}
  \label{eq:14}
  \left|\vec n\right|^2=1
\end{equation}
В результате получим гамильтониан
\begin{equation}
  \label{eq:13}
  E[\vec n]=J\sum_{\left<ij\right>}\vec n_i\cdot \vec n_j-\sum_i \vec h\cdot \vec n_i
\end{equation}
Это гамильтониан классической $O(m)$-модели Гейзенберга. 

В критической точке параметр порядка системы стремится к бесконечности, кроме того, фазовый переход в дискретных моделях наблюдается только в термодинамическом пределе числа вершин стремящегося к бесконечности ($N\to \infty$). Поэтому при описании фазового перехода имеет смысл перейти модели с дискретным пространством  к непрерывному пространству $d$-мерному пространству. В дальнейшем $d$ будет равно двум, но сейчас мы пишем в самом общем виде, чтобы указать аналогию с теорией поля.
При переходе к непрерывному пространству гамильтониан \eqref{eq:13} принимает вид
\begin{equation}
  \label{eq:15}
  E[\vec n]=\int d^d x \left(J\partial_kn_i\partial_k  n_i-h_i\cdot n_i\right)
\end{equation}
Здесь подразумевается суммирование по повторяющимся индексам. (Легко понять обратный переход, если заменить производные разностями по ближайшим соседям, а потом отбросить в энергии постоянный член $J\sum_i \left|\vec n_i\right|^2=JN$).
Условие $\vec n^2(x)=1$ сложно использовать, поэтому мы заменим его на другое аналогичное. Первый вариант такой замены --- это $\frac{1}{V}\int d^dx\; \vec n^2=1$. В результате получается так называемая сферическая модель. Другая возможность --- добавить в гамильтониан \eqref{eq:15} потенциал $V(\left|\vec n\right|)$ с минимумом в $\left|\vec n\right|=1$ и устремить константу связи к бесконечности. Простейший вид такого потенциала --- $V(\left|\vec n\right|)=a\vec n^2+b\left(\vec n^2\right)^2$. Добавим его в гамильтониан и изменим нормировку $\vec n$ так, чтобы константа $J$ ушла. Получим
\begin{equation}
  \label{eq:16}
  E[\vec n]=\int d^d x \left(\frac{1}{2}\partial_k n_i\cdot \partial_k n_i-\frac{1}{2}\mu^2 \vec n^2+\frac{1}{4}u\left(\vec n^2\right)^2\right)
\end{equation}
Если у $\vec n$ всего одна компонента $\varphi$, то мы получаем скалярную модели $\varphi^4$. Если $u=0$, то такая модель называется Гауссовой и допускает точное решение.
\begin{equation}
  \label{eq:17}
  E[\varphi]=\int d^d x (\frac{1}{2}(\nabla \varphi)^2+\frac{1}{2}\mu^2 \varphi^2)
\end{equation}
Статсумма --- это сумма $e^{-\beta E[\varphi]}$ по всем конфигурациям поля $\varphi$, то есть она записывается при помощи функционального интеграла:
\begin{equation}
  \label{eq:18}
  Z=\int \mathcal{D}\varphi e^{-\beta E[\varphi]}
\end{equation}
В гауссовой модели функциональный интеграл имеет гауссовский вид, поэтому он хорошо определен и легко вычисляется. Значит можно вычислить статсумму и все корреляционные функции. Величины в модели $\varphi^4$ можно получать по теории возмущений вокруг решения гауссовой модели. Это и есть вычисление функционального интеграла как суммы по фейнмановским диаграммам. 

\section{Лекция 2. }
\label{sec:lecture-2}

\subsection{Квантовые статистические модели}
\label{sec:quantum-statistical-models}

Микросостояния квантовых статистических систем описываются при помощи матрицы плотности
\begin{equation}
  \label{eq:19}
  \rho=e^{-\beta H}
\end{equation}
Статсумма дается суммой по собственным состояниям гамильтониана или следом матрицы плотности:
\begin{equation}
  \label{eq:20}
  Z=\sum_n e^{-\beta E_n}=\mathrm{Tr} \rho
\end{equation}
Значение физической величины определим как среднее по состояниям:
\begin{equation}
  \label{eq:21}
  \left< A\right>=\sum_n \left< n\right| e^{-\beta H}A\left| n \right>=\mathrm{Tr}(\rho A)
\end{equation}
$e^{-\beta H}$ похоже на оператор эволюции $e^{-i H t}$, то есть его тоже можно записать через функциональный интеграл. Покажем, как это делается на примере системы с одной степенью свободы. В этом случае ядро оператора плотности имеет вид
\begin{equation}
  \label{eq:22}
  \rho(x_f,x_i)=\left<x_f\right|e^{-\beta H} \left|x_i\right>
\end{equation}
Вспомним, что в квантовой механике оператор эволюции имеет вид $U(t)=e^{-i H t}$, а его ядро --- амплитуда перехода из состояния $\left|x_i\right>$ в $\left<x_f\right|$ дается интегралом по траекториям
\begin{equation}
  \label{eq:23}
  \left<x_f\right| U(t)\left| x_i\right>=\int_{(x_i,0)}^{(x_f,t)} [dx] e^{i S[x]}
\end{equation}
Заменой $t\to i\tau, \; \tau\in [0,\beta]$ (виковским поворотом) переходим к евклидову действию $S[x(t)]\to iS_E[x(\tau)]$. Для ядра оператора плотности получаем
\begin{equation}
  \label{eq:24}
  \rho(x_f,x_i)=\int_{(x_i,0)}^{(x_f,\beta)} [dx] e^{-S_E[x]}
\end{equation}
Статсумма
\begin{equation}
  \label{eq:25}
  Z=\int dx \rho(x,x)=\int [dx] e^{-S_E[x]}
\end{equation}
Среднее значение $A$
\begin{equation}
  \label{eq:26}
  \begin{array}{l}
    \left<A\right>=\frac{1}{Z}\int dx \left<x\right|\rho A\left|x\right>=\frac{1}{Z}\int dx\; dy \left<x\right|\rho\left|y\right>\left<y\right|A\left|x\right>=    \\
    =\frac{1}{Z}\int dx\; dy \int_{(x,0)}^{(y,\beta)} [dx] \left<y\right| A\left| x\right> e^{-S_E[x]}=\frac{1}{Z}\int dx\; dy\int_{(x,0)}^{(y,\beta)} [dx] A(x) \delta(x-y) e^{-S_E[x]}=\\
    =\frac{1}{Z}\int [dx] A(x(0))e^{-S_E[x]}
  \end{array}
\end{equation}
Здесь мы предположили, что $A$ зависит только от $x$, то есть $\left<y\right| A\left|x\right>=A(x)\delta(x-y)$. Мы видим, что значение $A$ дается функциональным интегралом, но $A$ вычисляется в точке $\tau=0$. 

Обобщение на систему с континуумом степеней свободы и многочастичные функции естественно. Статсумма квантовой системы получается из обычного интеграла по траекториям путем викова поворота и ограничения евклидова времени на конечный промежуток $[0,\beta]$. При нулевой температуре ($\beta\to \infty$) мы получаем обычный производящий функционал в евклидовом времени, то есть теорию поля. При конечных температурах статсумма квантовой $d$-мерной системы напоминает статсумму $d+1$-мерной классической системы на полосе шириной $\beta$.

\subsection{Критические явления}
\label{sec:critical-phenomena}

Фазовые переходы характеризуются скачком в макроскопических характеристиках системы. При переходах первого рода скачком меняется внутренняя энергия (например, при переходе жидкость-газ). При переходах второго рода наблюдается скачок производных макроскопических термодинамических величин (теплоемкость, магнитная восприимчивость). 

Строго говоря, фазовые переходы бывают только в термодинамическом пределе. Это легко понять для систем типа модели Изинга в нулевом поле, где энергия любой конфигурации кратна энергетическому масштабу $\epsilon=-J$, а статсумма представляет собой полином от $z=e^{-\beta\epsilon}$. В модели Изинга наибольшая энергия $E=2N\epsilon$, $Z$- полином степени $2N$ от $z$ с единичными коэффициентами. Корни этого полинома лежат вне положительной вещественной оси и комплексно сопряжены. Сингулярности свободной энергии или ее производных могут быть только в корнях, которые находятся вне физической области при $N<\infty$. При $N\to \infty$ число корней становится бесконечным, они лежат на дугах, которые могут касаться положительной вещественной оси. В этих точках поведение термодинамических величин и становится сингулярным.

Нас будут интересовать переходы второго рода. Именно в них имеет место конформная инвариантность. 
Перечислим основные результаты для модели Изинга в двух измерениях.

Здесь есть один фазовый переход при конечной температуре. 
Критическая температура
\begin{equation}
  \label{eq:27}
  T_c:\quad \sh \frac{2J}{T_c}=1
\end{equation}
При $T>T_c$ спонтанная намагниченность исчезает. При $T<T_c$ она не равна нулю и стремится к $1$ при $T\to 0$ и к $0$ при $T\to T_c$. Это ферромагнитная фаза. Около критической точки спонтанная намагниченность ведет себя как
\begin{equation}
  \label{eq:28}
  M\sim \left|T_c-T\right|^{\frac{1}{8}}
\end{equation}
Направление намагниченности определяется тем, куда было направлено внешнее поле до того, как его устремили к $0$. 

При подходе к критической точке магнитная восприимчивость расходится
\begin{equation}
  \label{eq:29}
  \chi=\frac{\partial M}{\partial h}\sim \left|T-T_c \right|^{-\frac{7}{4}}.
\end{equation}
Вдали от критической точки корреляторы $\Gamma_c(i)$ убывают экспоненциально с расстоянием. Характерная длина убывания $\xi$ называется корреляционной длиной
\begin{equation}
  \label{eq:30}
  \left<\sigma_i\sigma_j\right>_c\sim e^{-\frac{\left|i-j\right|}{\xi(T)}}\quad \left|i-j\right|>>1.
\end{equation}
При $T\to T_c$ корреляционная длина расходится
\begin{equation}
  \label{eq:31}
  \xi(T)\sim \frac{1}{\left|T-T_c\right|}.
\end{equation}
Такое поведение характеризует переходы второго рода и наблюдается в критических точках фазовой диаграммы.

Около критической точки система состоит из доменов разной намагниченности. Существуют домены всевозможных размеров от шага решетки до корреляционной длины $\xi$. Свободная энергия $F$ получает вклады от доменных стенок. Сингулярное поведение возникает из-за вкладов доменов большого размера, оно определяется флуктуациями и не может быть правильно учтено в теории среднего поля. Вблизи критической температуры $\xi$ больше физических размеров системы. Свободная энергия перестает зависеть от $\xi$. 

Парная корреляционная функция ведет себя как
\begin{equation}
  \label{eq:32}
  \Gamma(\vec n)\sim \frac{1}{\left|\vec n\right|^{d-2+\eta}}
\end{equation}
Поведение термодинамических величин определяется критическими индексами. Перечислим их вместе со значениями в двумерной модели Изинга. Эти значения получены из точных решений модели.
\begin{table}[h!tb]
\label{tab:diagrams}
\noindent  \centering{
\begin{tabular}{|l|l|l|}
  \hline
  Критический индекс & Термодинамическая величина & Значение индекса в \\
  &  & двумерной модели Изинга\\
  \hline
  $\alpha$ & $C\sim \frac{1}{\left|T-T_c\right|^{\alpha}}$ & $0$\\
  $\beta$ & $M\sim \left|T-T_c\right|^{\beta}$ & $\frac{1}{8}$\\
  $\gamma$ & $\chi\sim \frac{1}{\left|T-T_c\right|^{\gamma}}$ & $\frac{7}{4}$\\
  $\delta$ & $M\sim h^{\frac{1}{\delta}}$ & $15$ \\
  $\nu$ & $\xi\sim \frac{1}{\left|T-T_c\right|^{\nu}}$ & $1$\\
  $\eta$ & $\Gamma(\vec n)\sim \left|\vec n\right|^{2-d-\eta}$ & $\frac{1}{4}$\\
  \hline
\end{tabular}
\caption{Критические индексы и их значения в модели Изинга в 2 измерениях}
}
\end{table}

Заметим, что систему можно описывать классической статистической механикой (обычной квантовой теорией поля с неограниченным временем) когда корреляционная длина больше характерной длины волны де Бройля. Если характерная скорость $v$ (скорость света, скорость Ферми или скорость какого-либо возбуждения), то $\lambda_T=\frac{v\hbar}{k_B T}\sim \beta$. То есть классическая статистика применима при больших температурах или вблизи критической точки, если $T_c\neq 0$.

\subsection{Скейлинг или масштабная инвариантность}
\label{sec:scaling}

Гипотеза подобия связывает между собой критические индексы. Ее можно сформулировать так: плотность свободной энергии $f(t,h)=\frac{F}{N}$ около критической точки является однородной функцией $h$ и безразмерной температуры $t=\frac{T}{T_c}-1$. То есть существуют $a,b$:
\begin{equation}
  \label{eq:33}
  f(\lambda^a t,\lambda^b h)=\lambda f(t,h)
\end{equation}
В следующих лекциях мы обоснуем эту гипотезу при помощи ренормгруппы.

Покажем теперь, какие условия накладывает гипотеза подобия на критические индексы. Заметим, что $t^{-\frac{1}{a}}f$ инвариантна при преобразовании $t\to \lambda^a t, h\to \lambda^b h$. Введем
\begin{equation}
  \label{eq:34}
  y=\frac{h}{t^{\frac{b}{a}}}
\end{equation}
Тогда
\begin{equation}
  \label{eq:35}
  f(t,h)=t^{\frac{1}{a}}g(y).
\end{equation}
Для спонтанной намагниченности имеем
\begin{equation}
  \label{eq:36}
  M=-\left.\frac{\partial f}{\partial h}\right|_{h=0}=t^{\frac{1-b}{a}}g'(0),
\end{equation}
а для  магнитной восприимчивости
\begin{equation}
  \label{eq:37}
  \chi=\left.\frac{\partial^2 f}{\partial h^2}\right|_{h=0}=t^{\frac{1-2b}{a}}g''(0).
\end{equation}
Аналогично для удельной теплоемкости получаем
\begin{equation}
  \label{eq:38}
  c=-T\left.\frac{\partial^2 f}{\partial T^2}\right|_{h=0}=-\frac{1}{T_c}t^{\frac{1}{a-2}}g''(0)
\end{equation}
В пределе $t\to 0$ $M\sim h^{\frac{1}{\delta}}$, поэтому $g'(y)\sim y^{\frac{1}{\delta}}$ при $y\to \infty$. А значит, если в критической точке намагниченность $M$ конечна и не равна нулю, должно выполняться равенство $1-b-\frac{b}{\delta}=0$. Подставляя выражение для $\delta$ через $b$ в (\ref{eq:36},\ref{eq:37},\ref{eq:38}), мы получаем связь критических индексов
  \begin{eqnarray}
    \label{eq:39}
    \alpha=2-\frac{1}{a}\\
    \beta=\frac{1-b}{a}\\
    \gamma=-\frac{1-2b}{a}\\
    \delta=\frac{b}{1-b}
  \end{eqnarray}
\subsection{Процедура Каданова}
\label{sec:kadanoff-procedure}
Теперь обоснуем гипотезу подобия в модели Изинга при помощи процедуры Каданова и свяжем индексы $a,b$ с $\nu,\eta$.
Рассмотрим кубическую $d$-мерную модель Изинга.
\begin{equation}
  \label{eq:40}
  H=-J\sum_{\left<ij\right>}\sigma_i \sigma_j-h\sum_i \sigma_i
\end{equation}
Разбиваем решетку на блоки со стороной в $r$ ячеек. Будем нумеровать блоки большими латинскими буквами $I,J$ и понимать под записью $i\in I$ перечисление вершин в $I$-том блоке.
Сумма спинов в блоке принимает значения от $r^{-d}$ до $r^d$. Введем блочный спин
\begin{equation}
  \label{eq:41}
  \Sigma_I=\frac{1}{R}\sum_{i\in I}\sigma_i
\end{equation}
$R$ - нормировочный множитель, который выбирается так, чтобы блочный спин мог принимать значения $\pm 1$. Если бы спины были всегда в одном направлении, то $R$ имел бы значение $r^d$. 

Мы предполагаем, что поведение вблизи критической точки описывается в терминах блочных спинов, так как корреляционная длина $\xi>>r$. Гамильтониан для блочных спинов имеет вид
\begin{equation}
  \label{eq:42}
  H'=-J' \sum_{\left<IJ\right>} \Sigma_I \Sigma_J-h' \sum_I \Sigma_I
\end{equation}
Так как корреляционная длина для блоков
\begin{equation}
  \label{eq:43}
  \xi'=\frac{\xi}{r},
\end{equation}
то из (\ref{eq:31}) для температуры имеем
\begin{equation}
  \label{eq:44}
  t'=r^{\frac{1}{\nu}} t.
\end{equation}
Суммарная энергия взаимодействия со внешним полем должна оставаться неизменной, то есть
\begin{equation}
  \label{eq:45}
  h\sum_i \sigma_i=h' \sum_I \Sigma_I=\frac{h'}{R}\sum_i \sigma_i\quad\Longrightarrow\quad h'= Rh
\end{equation}
Так как общая свободная энергия тоже не должна меняться, то свободная энергия на блок $\sim r^d f$, то есть
\begin{equation}
  \label{eq:46}
  f(t',h')=r^d f(t,h)
\end{equation}
или
\begin{equation}
  \label{eq:47}
  f(t,h)=r^{-d} f(r^{\frac{1}{\nu}} t, Rh)
\end{equation}
Осталось вычислить $R$ как функцию от $r$, чтобы оправдать гипотезу подобия. Смотрим на парный коррелятор в критической точке
\begin{multline}
  \label{eq:48}
  \Gamma'(\vec n)=\left< \Sigma_I\Sigma_J\right>-\left< \Sigma_I\right>\left<\Sigma_J\right>=\frac{1}{R^2}\sum_{i\in I}\sum_{j\in J}\left(\left<\sigma_i \sigma_j\right>-\left<\sigma_i\right>\left<\sigma_j\right>\right) =\\
  = \frac{r^{2d}}{R^2}\Gamma(r\vec n)=\frac{R^{-2}r^{2d}}{\left|r\vec n\right|^{d-2+\eta}}=\frac{R^{-2}r^{d+2-\eta}}{\left|\vec n\right|^{d-2+\eta}}
\end{multline}
Отсюда
\begin{equation}
  \label{eq:49}
  R=r^{\frac{d+2-\eta}{2}}, \quad h'=r^{\frac{d+2-\eta}{2}}h
\end{equation}
Подставляя в (\ref{eq:33}) $r=\lambda^{\frac{1}{d}}$ получаем выражение для скейлинговых параметров $a,b$:
\begin{eqnarray}
  \label{eq:50}
  a=\frac{1}{\nu d}\\
  b=\frac{d+2-\eta}{2d}
\end{eqnarray}
Для критических индексов имеем
\begin{eqnarray}
  \label{eq:51}
  \alpha=2-\nu d\\
  \beta=\frac{1}{2}\nu (d-2+\eta)\\
  \gamma=\nu (2-\eta)\\
  \delta=\frac{d+2-\eta}{d-2+\eta}
\end{eqnarray}

\section{Лекция 3}
\label{sec:lecture-3}
В прошлой лекции мы обосновали гипотезу подобия при помощи рассмотрения блочных спинов и введения эффективного гамильтониана, имеющего ту же форму, но другие константы связи. Эта процедура является перенормировкой в координатном пространстве, она определяет отображение гамильтониана $H\to H'$. Последовательное применение такой процедуры порождает так называемую ренормгруппу. Строго говоря, ренормгруппа является не группой, полугруппой, так как у преобразования нет обратного, ведь при переходе к блокам большего размера мы теряем информацию. 
Опишем ренормгруппу для решеточных моделей.

\subsection{Ренормгруппа в решеточных моделях}
\label{sec:renormgroup-lattice-models}
Рассматриваем решеточную модель, состоящую из $N$ спинов $\sigma_i$ на $d$-мерной решетке. Общий вид гамильтониана таких моделей
\begin{equation}
  \label{eq:55}
  H(\vec J,[\sigma],N)=J_0 + J_1\sum_i \sigma_i +J_2\overset{(1)}{\sum_{\left<ij\right>}}\sigma_i \sigma_j+J_3\overset{(2)}{\sum_{\left<ij\right>}}\sigma_i \sigma_j + \dots
\end{equation}
Здесь мы явно указали на зависимость гамильтониана от констант связи, которые собраны в вектор $\vec J=(J_0,J_1,J_2,J_3,\dots)$. Сумма $\overset{(1)}{\sum_{\left<ij\right>}}$ - это сумма по ближайшим соседям, $\overset{(2)}{\sum_{\left<ij\right>}}$ - сумма по вершинам, разделенным двумя ребрами (через одного) и так далее. 
Статсумма
\begin{equation}
  \label{eq:56}
  Z(\vec J,N)=\sum_{[\sigma]}e^{-H(\vec J,[\sigma],N)}
\end{equation}
Мы так переопределили константы связи, чтобы избавиться от множителя $\beta$.

Теперь мы вводим блочные переменные с блоком со стороной в $r$ ячеек решетки, содержащим $r^d$ вершин. $\Sigma_I$ - блочные спины, $\xi_i$ - переменные внутри блока. В таких переменных статсумма переписывается так:
\begin{equation}
  \label{eq:57}
  Z(\vec J,N)=\sum_{[\Sigma][\sigma]}e^{-H(\vec J,[\Sigma],[\sigma],N)}
\end{equation}
Если мы просуммируем по внутриблочным степеням свободы $\xi$, то получится перенормированный гамильтониан $H'(\vec J,[\Sigma],Nr^{-d})$.
\begin{equation}
  \label{eq:58}
  e^{-H'(\vec J',[\Sigma],Nr^{-d})}=\sum_{[\xi]}e^{-H(\vec J,[\Sigma],[\xi],N)}
\end{equation}
Вблизи критической точки можно предполагать что у $H'$ такой же вид, как и у $H$, так как параметр порядка расходится и поведение определяется дальними корреляциями. Описания поведения системы в терминах $H'$ и $H$ эквивалентны, поэтому соответствующие выражения для статсуммы должны совпадать (с точностью до умножения на константу):
\begin{equation}
\label{eq:52}
  Z(\vec J,N)=\sum_{[\Sigma]}e^{-H'(\vec J',[\Sigma],Nr^{-d})}=Z(\vec J',Nr^{-d})
\end{equation}
Удельная свободная энергия отображается как $f(\vec J)=r^{-d} f(\vec J')$.

Отображение констант связи $\vec J\to\vec J'$ и порождает ренормгруппу. Будем его записывать как
\begin{equation}
  \label{eq:53}
  \vec J'=T(\vec J)
\end{equation}
Последовательное применение таких преобразований порождает в пространстве констант связи некоторый набор точек, который называется траекторией ренормгруппы.

На каждом шаге корреляционная длина уменьшается в $r$ раз, поэтому траектория уводит систему из окрестности критической точки. Но в самой критической точке корреляционная длина бесконечна, поэтому эта точка неподвижна относительно действия ренормгруппы. 

В общем случае критическое поведение наблюдается на некоторой критической поверхности в пространстве $\vec J$. Под действием ренормгруппы точки двигаются по ней. Стационарная точка
\begin{equation}
  \label{eq:54}
  \vec J_c=T(\vec J_c)
\end{equation}
называется фиксированной точной ренормгруппы.

Вообще говоря, преобразование $T$ нелинейно, однако в окрестности фиксированной точки его можно линеаризовать. Вводим переменную $\delta\vec J=\vec J-\vec J_c$ и раскладываем $T$ в ряд Тейлора по $\delta\vec J$. Тогда
\begin{equation}
  \label{eq:59}
  \delta\vec J'=A\delta\vec J,
\end{equation}
где $A$ - матрица, $A_{ij}=\frac{\partial T_i}{\partial J_j}$. Мы можем диагонализовать матрицу $A$, ее собственные значения мы обозначим через $\lambda_i$, собственные векторы через $\vec u_i$. Тогда для точки в пространстве констант связи мы можем написать (в окрестности  критической точки):
\begin{equation}
  \label{eq:60}
  \vec J=\vec J_c+\sum_i t_i \vec u_i
\end{equation}
Переменные $t_i$ под действием ренормгруппы преобразуются так:
\begin{equation}
  \label{eq:61}
  t'_i=\lambda_i t_i=r^{y_i} t_i
\end{equation}
Здесь мы ввели обозначение $y_i$ для критических индексов. В примере из раздела \ref{sec:kadanoff-procedure} такими индексами были $a$ и $b$. 
Для свободной энергии имеем
\begin{equation}
  \label{eq:62}
  f(t_1,t_2,\dots)=r^{-d} f(r^{y_1}t_1,r^{y_2}t_2,\dots)
\end{equation}
Таким образом, критические индексы получаются из собственных значений линеаризованной ренормгруппы в фиксированной точки.

Если часть собственных значений $\lambda_i$ положительна, а часть - отрицательная, то фиксированная точка называется гиперболической. 
Критической поверхностью называются такие точки $\vec J$, что
\begin{equation}
  \label{eq:63}
  \lim_{n\to \infty}T^n (\vec J)=\vec J_c
\end{equation}
Около точки $\vec J_c$ она представляет собой векторное пространство, натянутое на те векторы $\vec u_i$, для которых соответствующие собственные значения $\lambda_i<1$. 

Параметры $t_i$, которые соответствуют $\lambda_i>1$ называются релевантными, так как они растут под действием ренормгруппы. Параметры с $y_i<0 \; (\lambda_i<1)$ --- иррелевантные, с $y_i=0$ --- маргинальные.

Существование критической поверхности объясняет универсальность критических индексов в разных моделях. Статистические системы разбиваются на классы универсальности, члены которых имеют одинаковое критическое поведение. Это так, если разные системы живут на подмногообразиях в пространстве констант связи, которые пересекают одну и ту же критическую поверхность. Тогда критическая точка у них общая.

\subsection{Ренормгруппа в непрерывных моделях}
\label{sec:renormgroup-general}
В непрерывных моделях перенормировка осуществляется в импульсном пространстве. Рассмотрим скалярное поле $\varphi(\vec x)$, где $\vec x\in \mathbb{R}^d$. Проведем преобразование Фурье
\begin{equation}
  \label{eq:64}
  \varphi(\vec x)=\int \frac{d^d \vec k}{(2\pi)^d}\tilde \varphi(\vec k) e^{i\vec k\cdot \vec x}
\end{equation}
Функционал действия может быть переписан в виде, зависящем только от $\tilde\varphi$: $S[\varphi]\to S[\tilde\varphi]$. Например, для модели $\varphi^4$ \eqref{eq:16} получается
\begin{multline}
  \label{eq:65}
  S[\varphi,r,u]=\int  \frac{d^d \vec k}{(2\pi)^d} \frac{1}{2} \tilde\varphi(-\vec k)\tilde\varphi(\vec k)(\vec k^2+r)+\\
  \frac{1}{4}u\int  \frac{d^d \vec k_1}{(2\pi)^d} \frac{d^d \vec k_2}{(2\pi)^d} \frac{d^d \vec k_3}{(2\pi)^d} \tilde\varphi(-\vec k_1-\vec k_2 -\vec k_3)\tilde \varphi(\vec k_1)\tilde\varphi (\vec k_2)\tilde\varphi(\vec k_3).
\end{multline}
В общем случае функционал действия зависит от поля $\varphi$ и параметров $u_i$ в плотности Лагранжиана (констант связи).

Квантовая теория поля всегда содержит расходящиеся интегралы, поэтому нужна регуляризация. Будем считать, что в качестве такой регуляризации используется обрезание по импульсам $\left|\vec k\right|<\Lambda$. После преобразования Фурье мера в функциональном интеграле переписывается как
\begin{equation}
  \label{eq:66}
  [d\varphi]_{\Lambda}=\prod_x d\varphi(x)=\prod_{\left|\vec k\right|<\Lambda}d\tilde\varphi(\vec k)
\end{equation}
Теперь опишем перенормировку Вильсона-Каданова. Она проводится в два этапа. Сначала мы интегрируем по $\tilde\varphi(\vec k): \Lambda/s<\left|k\right|<\Lambda$. То есть мы исключаем быстрые моды, так как критическое поведение описывается медленными модами.  $s$-масштабный фактор. В результате мы получаем новое обрезание $\Lambda/s$ и действие $S'[\varphi,\{u_i\}]$.
\begin{equation}
  \label{eq:67}
  e^{-S'[\varphi,\{u_i\}]}=\int \prod_{\Lambda/s<\left|\vec k\right|<\Lambda}d\varphi(\vec k)e^{-S[\varphi,\{u_i\}]}
\end{equation}
Пока нас интересую только медленные моды, действие $S'$ эквивалентно $S$. Вторым шагом осуществим масштабное преобразование
\begin{eqnarray}
  \label{eq:68}
  \vec k \to \vec k'=s \vec k\\
  \vec x\to \vec x'=\vec x/s
\end{eqnarray}
При этом поле преобразуется так:
\begin{eqnarray}
  \label{eq:69}
  \varphi(\vec x)\to \varphi'(\vec x/s)=s^{\Delta}\varphi (\vec x)\\
  \tilde\varphi'(s\vec k)=s^{\Delta-d}\tilde\varphi (\vec k)
\end{eqnarray}
Здесь $\Delta$- скейлинговая размерность поля $\varphi$, она связана с индексом $\eta$: $\Delta=\eta/2$. Мера в функциональном интеграле при масштабном преобразовании умножится на константу. Теперь можно сравнить $S'$ и $S$, так как масштаб обрезания и там и там равен $\Lambda$ и те же степени свободы. Эти два действия описывают одинаковое поведение на больших расстояниях. Поэтому вид действия $S'$ должен быть тот же, что и у $S$, но с другими константами связи $u_i'$.
\begin{equation}
  \label{eq:70}
  S'[\varphi,\{u_i\}]=S[\varphi,\{u_i'\}]
\end{equation}
Последовательным действием таких преборазований мы получаем кривую $u_i(s)$ в пространстве параметров лагранжиана.  Мы можем записать уравнение этой кривой как
\begin{equation}
  \label{eq:71}
  \frac{du_i}{d(\ln s)}=\beta_i(\{u_i\})
\end{equation}
$\beta_i$ называется $\beta$-функцией параметра $u_i$. 
Фиксированная точка $u_j^{*}$:
\begin{equation}
  \label{eq:72}
  \beta_i(\{u^{*}_j\})=0
\end{equation}
Эта точка неподвижна относительно преобразований ренормгруппы. В фиксированной точке теория масштабно-инвариантна на квантовом уровне.


\bibliography{2dqft}{}
\bibliographystyle{utphys}

\end{document}
