\documentclass[a4paper,12pt]{article}
\usepackage[unicode,verbose]{hyperref}
\usepackage{amsmath,amssymb,amsthm} \usepackage{pb-diagram}
\usepackage{ucs}
\usepackage[utf8x]{inputenc}
\usepackage[russian]{babel}
\usepackage{cmap}
\usepackage{graphicx}
\pagestyle{plain}
\theoremstyle{definition} \newtheorem{Def}{Definition}
\begin{document}
\tableofcontents
\section{String theory}
\label{sec:string-theory}

Free relativistic particle action is equal to length of trajectory. 
\begin{equation}
  \label{eq:1}
  \begin{split}
  x^{\mu}(0)=x^{\mu}_0,\; x^{\mu}(1)=x^{\mu}_1\\
  S=\int_{0}^{1} \sqrt{\left(\frac{dx^{\mu}}{d\tau}^2\right)}d\tau
\end{split}
\end{equation}
There is Poincare invariance $x^{\mu}\to A^{\mu}_{\nu}x^{\nu}+ B^{\mu}$ and reparametrisation invariance $\tau\to f(\tau),\; x^{\mu}(\tau)\to x^{\mu}(f(\tau))$.

String action is the area of world-sheet:
\begin{equation}
  \label{eq:2}
  \begin{split}
    X^{\mu}(x^1,x^2),\; \mu=1\dots d\\
    S_N [X^{\mu}(x^1,x^2)]=\int \sqrt{det \frac{\partial X^{\mu}}{\partial x^a}\frac{\partial X^{\mu}}{\partial x^b}} dx^1 dx^2
  \end{split}
\end{equation}
This action is called Nambu action, and it is invariant under Lorenz transformations and reparametrisations. 

Transition amplitude for string is given by following formula:
\begin{equation}
  \label{eq:3}
  A=\int {\cal D}X^{\mu}(x^1,x^2)e^{-S_N[X^{\mu}(x^1,x^2)]}
\end{equation}
The problem here is the definition of measure of integration. 
It is not strictly defined in most cases, so there is nothing unusual here.

We want to change ``integration variable'' in order to fix the arbitrariness connected with reparametrisations of world-sheet.

Obviously our theory should be invariant under reparametrisation:
\begin{equation}
  \label{eq:4}
  \begin{split}
    x^a \to f^a (x^1,x^2)\\
    X^{\mu}(x^1,x^2)\to \tilde{X}^{\mu}(x^1,x^2)=X^{\mu}(f^1(x^1,x^2),f^2(x^1,x^2))
  \end{split}
\end{equation}

But Nambu action is not invariant. So we will use equivalent Polyakov action, which has following form:
\begin{equation}
  \label{eq:5}
  \begin{split}
    S_P [g^{ab}(x^1,x^2), X^{\mu}(x^1,x^2)]=\int g^{ab}(x)\partial_a X^{\mu} \partial_b X^{\mu}\sqrt{g} \;d^2 x\\
    g=\det g_{ab}
  \end{split}
\end{equation}

Polyakov and Nambu actions are equivalent in the sense that equations of motion are the same after the exclusion of auxiliary field $g_{ab}$.
\begin{equation}
  \label{eq:6}
  \delta_{X}S_N=0 \Longleftrightarrow \left\{
      \begin{aligned}
        \delta_X S_P=0\\
        \delta_g S_P=0
      \end{aligned}
      \right.
\end{equation}
Polyakov action is invariant under Lorenz transformations and reparametrisations:
\begin{equation}
  \label{eq:7}
  \begin{split}
    x^a \to f^a (x^1,x^2)\\
    X^{\mu}(x^1,x^2)\to \tilde{X}^{\mu}(x^1,x^2)=X^{\mu}(f^1(x^1,x^2),f^2(x^1,x^2))\\
    g_{ab}(x)\to \tilde{g}_{ab}(x)=\frac{\partial f^{a_1}}{\partial x^{a}}\frac{\partial f^{b_1}}{\partial x^b} g_{a_1 b_1}(x)
  \end{split}  
\end{equation}

There is also one additional invariance, which is called conformal or Weyl invariance.
\begin{equation}
  \label{eq:8}
  \begin{split}
    g_{ab}(x)\to e^{\sigma(x)}g_{ab}(x)\\
    g\to e^{\sigma} g\\
    g^{ab}\to e^{-\sigma} g^{ab}\\
    S_P[e^{\sigma(x)} g_{ab}(x),X^{\mu}] = S_P[g_{ab}(x),X^{\mu}] 
  \end{split}
\end{equation}

So we have rather big freedom. For example, any metric on sphere can be obtained from one (constant) metric with reparametrisation and Weyl transformation.
\begin{equation}
  \label{eq:9}
  g_{ab}(x)=[e^{\sigma(x)}\hat{g}_{ab}]^{x^a\to f^a (x)}
\end{equation}
For surfaces of higher genus there are discrete families of equivalent metrics.

We could not preserve all the symmetries during quantization. So we choose to save Lorenz invariance and reparametrisation invariance and break conformal invariance. It is so called conformal anomaly.

\begin{equation}
  \label{eq:10}
  A=\int {\cal D}X^{\mu}(x^1,x^2)e^{-S_P[g_{ab},X^{\mu}(x^1,x^2)]}
\end{equation}

It is easy to see that norm is invariant under reparametrisations:
\begin{equation}
  \label{eq:11}
  \begin{split}
    \|\delta X^{\mu}\|^2 = \int \left(\delta X^{\mu}(x)\right)^2 \sqrt{g} d^2x\\
    \|\delta g_{ab}\|^2 = \int \delta g_{a_1 b_1}\delta g_{a_2 b_2} g^{a_1 a_2} g^{b_1 b_2} \sqrt{g}\; d^2x
  \end{split}
\end{equation}

To investigate what happens top Weyl invariance after quantization let's introduce so called effective action. We integrate out fields $X^{\mu}$ and get following result.
\begin{equation}
  \label{eq:12}
  e^{-S_X^{eff}[g_{ab}]} \overset{def}{=} \int {\cal D}X^{\mu} e^{-S_P [g_{ab},X]}
\end{equation}
\begin{equation}
  \label{eq:13}
  \begin{split}
    S^{eff}_X [e^{\sigma(x)} g_{ab}(x) ] = S_X^{eff}[g_{ab}]+\frac{d}{48\pi}W[g,\sigma]\\
    W[g,\sigma]=\int \left[ \frac{1}{2} g^{ab}\partial_a\sigma \partial_b\sigma+R[g]\sigma+\mu e^{\sigma}\right] \sqrt{g}\; d^2x
  \end{split}
\end{equation}
Here $R$ is scalar curvature of $g_{ab}$, and $\mu$ is a parameter depending on regularisation, for example for proper time regularisation $\mu=\frac{1}{4\pi\epsilon}$. The term $W[g,\sigma]$ is called Liouville's action.

{\bf Notice} $\int R[x] \sqrt{g}d^2x = 4\pi \chi_E$, $\chi_E=2-2h$, $\chi_E$ - Euler characteristic of the surface, $h$ - genus of the surface.

$W[g,\sigma]$  is called conformal anomaly.

Variance of action
\begin{equation}
  \label{eq:14}
  \begin{split}
    \delta S[g_{ab},X] = -\frac{1}{4\pi} \int  \delta g^{ab} T_{ab} \sqrt{g} \; d^2x\\
    g_{ab}\to g_{ab}+\delta g_{ab}
  \end{split}
\end{equation}
\begin{equation}
  \label{eq:15}
  \begin{split}
    g_{ab} \to e^{\sigma} g_{ab} \xrightarrow[\sigma \ll 1]{} (1+\delta\sigma) g_{ab}\\
    \delta S [g_{ab},X] = -\frac{1}{4\pi} \int \delta\sigma g^{ab} T_{ab} \sqrt{g}\, d^2 x
  \end{split}
\end{equation}
\begin{equation}
  \label{eq:16}
    \delta S =0 \Rightarrow g^{ab}T_{ab}=0
\end{equation}
We see that Weyl invariance is equivalent to tracelessness of energy-momentum tensor.

After the quantization we have
\begin{equation}
  \label{eq:17}
  g^{ab}\left< T_{ab} \right> = -\frac{d}{12} R[g]
\end{equation}
\begin{Def}
  {\bf Two dimensional conformal field theory} - is the theory that transforms as in (\ref{eq:13}).

  The factor before $W[g,\sigma]$ is called {\it central charge}.
\end{Def}

!!!!!!!!!

Why it is so? How this definition is connected with another definitions and properties of conformal field theory?

!!!!!!!!!


At this moment we have too much freedom in our Feynman integral (\ref{eq:10}). We have diffeomorphisms (\ref{eq:4}) and Weyl transformations (\ref{eq:8}). We will fix the gauge so that the metric is given by the formula 
\begin{equation}
  \label{eq:20}
  g_{ab}=[e^{\phi} \hat{g}_{ab}]^{f^a}
\end{equation}

Faddeev-Popov ghosts appear as usual after this procedure.

\begin{equation}
  \label{eq:19}
  \begin{split}
    A=\int {\cal D} X^{\mu}(x^1,x^2)e^{-S_P[g_{ab},X^{\mu}(x^1,x^2)]} = \\
    = \int {\cal D}\phi\; {\cal D}X\; \mathcal{D} B_{ab}\; \mathcal{D} C^a e^{-S_P [e^{\phi}\hat{g}_{ab},X]-
      S_{gh}[e^{\phi}\hat{g}_{ab},B,C]}
  \end{split}
\end{equation}
Here $B,C$ are ghost fields, $B_{ab}(X)$ is symmetric and traceless tensor. All the functional integrals use metric $e^{\phi}\hat{g}$. 
\begin{equation}
  \label{eq:21}
  S_{gh} = \int g^{ac} B_{ab} \nabla_c C^b \sqrt{g} \, d^2x
\end{equation}
We can fix Weyl invariance so that measures of integration over matter fields and ghosts fields in (\ref{eq:19}) don't depend on $\phi$. Then we get conformal anomaly in the exponent:
\begin{equation}
  \label{eq:22}
  A = \int \mathcal{D}_{e^{\phi} \hat{g}} \mathcal{D}_{\hat{g}} X \mathcal{D}_{\hat{g}} B_{ab} \mathcal{D}_{\hat{g}} C^a e^{-S_P [e^{\phi}\hat{g}_{ab},X]-S_{gh}[e^{\phi}\hat{g}_{ab},B,C]+\frac{d-26}{48\pi} W[\hat{g}_{ab},\phi]}
\end{equation}
$\frac{d}{48\pi}$ comes from matter fields, and $\frac{-26}{48\pi}$ from ghost fields, since both actions are actions of two-dimensional conformal field theory.

Ordinary bosonic string lives in 26-dimensional space and doesn't interact with Liouville's gravity. Such a string is called ``critical string'' since it can exist only in critical dimension. Other string models are called non-critical strings.

First integration in (\ref{eq:22}) is very inconvenient, since we integrate over $\phi$ with measure depending on $\phi$. This measure $\mathcal{D}_{e^{\phi}\hat{g}}\phi$ corresponds to interval
\begin{equation}
  \label{eq:23}
  \|\delta\phi\|^2_P = \int e^{\phi} (\delta\phi)^2 \sqrt{\hat{g}}\, d^2 x
\end{equation}
It is non-linear and is not invariant under the translations $\phi(x)\to \phi(x)+\eta(x)$.
So for some time this was a problem since nobody knew how to work with such objects.

But David-Distler-Kawai proposed to change action in order to take into account change of measure of integration to normal measure.

They conjectured that the amplitude can be written in the form
\begin{equation}
  \label{eq:24}
  A=\int \mathcal{D}_{\hat{g}}\phi\, \mathcal{D}_{\hat{g}} X\, \mathcal{D}_{\hat{g}} (B,C)\, e^{-S_P[\hat{g},X]-S_{gh}[\hat{g},B,C]-S_L[\hat{g},\phi]}
\end{equation}
where the measure $\mathcal{D}_{\hat{g}} \phi$ corresponds to the interval $\|\delta\phi\|^2_{DDK} = \int (\delta\phi)^2 \sqrt{\hat{g}}\, d^2x$ and it is invariant under the translations $\phi\to \phi+\eta$ ($\delta\phi\to \delta\phi$). 
$S_L$ here is Liouville's action with two undefined for now parameters $Q$ and $b$:
\begin{equation}
  \label{eq:25}
  S_L[\hat{g},\phi] = \frac{1}{8\pi} \int \left[ \frac{1}{2} \hat{g}^{ab} \partial_a \phi \partial_b \phi + QR\phi +\mu e^{2b\phi}\right] \sqrt{\hat{g}} \, d^2x
\end{equation}
We choose following normalization:
\begin{equation}
  \label{eq:26}
  \langle \phi(x) \phi(0) \rangle = \log |x|^2
\end{equation}
Liouville's term should be invariant under dilation $\hat{g}_{ab}\to e^{\sigma}\hat{g}_{ab}$.

After we put
\begin{equation}
  \label{eq:27}
  Q=\frac{1}{b}+b
\end{equation}
Liouville's action becomes conformal, so that
\begin{equation}
  \label{eq:28}
  \begin{split}
    \int \mathcal{D}_{\hat{g}} \phi e^{-S_L[\hat g,\phi]} = e^{-S_L^{eff}[\hat g]}\\
    S_L^{eff} [e^{\sigma}\hat{g}]=S_L^{eff}[\hat g] +\frac{C_L}{12} W[\hat g,\sigma]
  \end{split}
\end{equation}
Where
\begin{equation}
  \label{eq:29}
  C_L=1+6 Q^2
\end{equation}

Then we require total central charge to vanish:
\begin{equation}
  \label{eq:30}
  C_X+C_{gh}+C_L=C_{tot}=0
\end{equation}

Now the amplitude doesn't depend on $\hat{g}$ and parameters $Q$,$b$ are fixed.
\begin{equation}
  \label{eq:18}
  d+1+6\left(b+\frac{1}{b}\right)^2=26
\end{equation}
In some finite area we can assume
\begin{equation}
  \label{eq:31}
  \hat{g}_{ab}=\delta_{ab}
\end{equation}
There also could be additional term connected with the topology.

Let's change variables
\begin{equation}
  \label{eq:32}
  \begin{split}
    x^1+i\, x^2=z\\
    x^1-i\, x^2=\bar{z}\\
    (dx^1)^2+(dx^2)^2=dz\,d\bar{z}
  \end{split}
\end{equation}
For fields and terms of the action we have
\begin{equation}
  \label{eq:33}
  \begin{split}
    C^z=C,\; C^{\bar{z}}=\bar{C}\\
    B_{zz}=B, \; B_{\bar{z}\bar{z}}=\bar{B}\\
    S_{gh}=\int d^2z [B\bar{\partial}C+\bar{B}\partial \bar{C}]\\
    S_P[X]=\int d^2z \bar{\partial}X \partial X\\
    S_L[\phi] \propto \int d^2z \left[\bar{\partial} \phi \partial \phi+\mu e^{2 b \phi}\right] + \mbox{(topological terms)}
  \end{split}
\end{equation}
\begin{equation}
  \label{eq:34}
  e^{-S^{eff}[g]}=\int \mathcal{D} X e^{-\int g^{ab}\partial_a X \partial_b X \sqrt{g}\, d^2x} = \int \mathcal{D}_g X e^{-\int X \Delta X\sqrt{g}\, d^2x}
\end{equation}
Laplace operator depends on metric
\begin{equation}
  \label{eq:35}
  \Delta = \frac{1}{\sqrt{g}}\partial_a \sqrt{g} g^{ab}\partial_b
\end{equation}

It is self-conjugate operator.

We can represent field $X$ as combination of eigenfunctions of Laplace operator $X(x)=\sum c_n X_n,\; \Delta X_n = \lambda_n X_n,\; \lambda_n=\lambda_n[g]$. Then we will integrate over all eigenfunctions:
\begin{equation}
  \label{eq:36}
  e^{-S^{eff}[g]}=\left(\prod_{n} \lambda_n\right)^{-\frac{1}{2}}=(\det \Delta)^{-\frac{1}{2}}
\end{equation}
Obviously we could not calculate $\lambda_n$, but we can get the determinant.
\begin{equation}
  \label{eq:37}
  S^{eff}[g]=\log \det \Delta=tr \log \Delta
\end{equation}
\begin{equation}
  \label{eq:38}
  \log\frac{\lambda}{\lambda_0}=\int_0^{\infty}\frac{e^{-\lambda t}-e^{\lambda_0 t}}{t}dt
\end{equation}
\begin{equation}
  \label{eq:39}
  \delta S^{eff}[g]=tr (\log\Delta -\log\Delta_{g_0})=\delta tr \int_0^{\infty}\frac{e^{-\Delta t}}{t} dt
\end{equation}
The last integral diverges, so we need to regularise it. We will use heat kernel regularisation.
\begin{equation}
  \label{eq:40}
  \delta S^{eff}[g]=\delta tr \int_{\epsilon}^{\infty}\frac{e^{-\Delta t}}{t} dt  
\end{equation}
As we already mentioned, $\hat{g}_{ab}$ could be put equal to $\rho dz d\bar{z}$ in finite area. Then
\begin{equation}
  \label{eq:41}
  \Delta=\frac{1}{\rho}\bar{\partial}\partial
\end{equation}
For variation of effective action we have
  \begin{multline}
  \label{eq:42}
    \delta_{\rho}S^{eff}[g]=\delta_{\rho} tr \int_{\epsilon}^{\infty} e^{-\frac{1}{\rho}\bar{\partial}\partial t}\frac{dt}{t}=\\
    - tr \int_{\epsilon}^{\infty}\delta_{\Delta}e^{-\Delta t} dt=
    -tr \int_{\epsilon}^{\infty}\frac{\delta\rho}{\rho}\Delta e^{-\Delta t} dt=\\
    -tr \int_{\epsilon}^{\infty}\frac{\delta\rho}{\rho}\frac{d}{dt} e^{-\Delta t} dt=-tr \frac{\delta\rho}{\rho}e^{-\Delta \epsilon}
  \end{multline}

Let's introduce Green function for heat equation
\begin{equation}
  \label{eq:43}
  G(t,x,y)=\sum_n e^{-\lambda_n t}X_n(x) X_n(y)
\end{equation}
\begin{equation}
  \label{eq:44}
  \begin{split}
    \frac{\partial G}{\partial t}=\Delta_x G(t,x,y)\\
    G(o,x,y)=\delta^2 (x-y)\\
    G(t,x,x)=\frac{1}{4\pi t}+aR+O(t)
  \end{split}
\end{equation}
Then for variation of action we have
\begin{equation}
  \label{eq:45}
  \delta S^{eff}[g]=\int d^2x \frac{\delta\rho}{\rho}\left(\frac{1}{4\pi\epsilon}+aR\right)
\end{equation}

!!!! !!!! !!!! !!!! 
Be more elaborate here.
!!!!!!!

...
\section{Conformal Field Theory}
\label{sec:cft}

\subsection{General properties}
\label{sec:general-properties}


In Conformal Field Theory 
\begin{equation}
  \label{eq:46}
  g^{ab}\left<T_{ab}\right>=\frac{C}{12}R[g]
\end{equation}
$C$ is a central charge.

For flat metric $g_{ab}=\delta_{ab}$ we have
\begin{equation}
  \label{eq:47}
  g^{ab}\langle T_{ab} \rangle=0
\end{equation}
Then we put $T_{zz}=T,\quad T_{\bar{z}\bar{z}}=\bar{T},\quad T_{z\bar{z}}=T_{\bar{z}z}$. In flat space from continuity equation $\bar{\partial}T_{zz}+\partial T_{\bar{z}\bar{z}}=0$ we have
\begin{equation}
  \label{eq:49}
  \begin{split}
    T_{z\bar{z}}=0\\
    \bar{\partial}T_{zz}=\bar{\partial}T=0,\quad T \mbox{- holomorphic}\\
    \partial T_{\bar{z}\bar{z}}=\partial \bar{T}=0,\quad \bar{T} \mbox{- antiholomorphic}\\
  \end{split}
\end{equation}
In curved space continuity equation reads $\nabla_a T^{ab}=0$ and we can choose coordinate system where $g_{ab}dx^a dx^b = \rho dz\, d\bar{z}=e^{\sigma}dz\,d\bar{z}$. Then we introduce pseudo energy $T$:
\begin{equation}
  \label{eq:50}
  T_{zz}+\frac{C}{12}t_{zz}=T
\end{equation}
where $t_{zz}=(\partial \sigma)^2-2\partial^2 \sigma$. It is easy to check that $\bar{\partial}T=0$ in this case.

For transformations $z\to w(z)$
\begin{equation}
  \label{eq:51}
  \begin{split}
    T_{zz}\to \left(\frac{dw}{dz}\right)^2 T_{ww}(w(z))\\
    t\to \left(\frac{dw}{dz}\right)t(w)+2\{w,z\}\\
    \mbox{where}\; \{w,z\}=\left(\frac{\partial_z^3 w}{\partial_z w}-\frac{3}{2}\left(\frac{\partial_z^2 w}{\partial_z}\right)^2\right)\\
    T\to \left(\frac{dw}{dz}\right)^2 T(w)+\frac{C}{12}\{w,z\}\\
  \end{split}
\end{equation}
Energy-momentum tensor is the generator of coordinate transformations $x^a\to x^a+\epsilon^a(x)$:
\begin{equation}
  \label{eq:52}
  \delta_{\epsilon}\langle A_1(x_1)\dots A_N(x_N)\rangle = \int d^2x\, \partial_a \epsilon_b(x)\langle T_{ab}(x) A_1(x_1)\dots A_N(x_N)\rangle
\end{equation}

\begin{equation}
  \label{eq:53}
  \begin{split}
    T_{aa}=0\\
    \partial_a \epsilon_b+\partial_b \epsilon_a\sim \delta_{ab}\\
    \epsilon^a=\epsilon x^a \;\mbox{or}\; x^a\to \frac{x^a}{(\vec x)^2}
  \end{split}
\end{equation}
Correlation functions don't change under conformal transformations.
\begin{equation}
  \label{eq:54}
  \delta_{\epsilon}A(x)=\int_{\partial D} dy^a\, \epsilon^b \tilde{T}_{ab}(y)A(x)-\int_{D}d^2y\, \partial^a \epsilon^b(y)T_{ab}(y)A(x)
\end{equation}
Where $\tilde{T}_{ab}=\epsilon_{ac}T_{cb}$.

Let
\begin{equation}
  \label{eq:55}
  \begin{split}
    \partial_a \epsilon_b+\partial_b \epsilon_a\sim \delta_{ab}\\
    \bar{\partial}\epsilon=0\\
    z\to z+\epsilon(z,\bar{z})\\
    \bar{z}\to\bar{z}+\bar{\epsilon}(z,\bar{z})
  \end{split}
\end{equation}
\begin{equation}
  \label{eq:56}
  \delta_{\epsilon}A(z,\bar{z})=\oint_z du\, \epsilon(u) T(u) A(z,\bar{z})
\end{equation}
Substituting $T(z)$ in place of arbitrary operator $A$ we get 
\begin{equation}
  \label{eq:57}
  \delta_{\epsilon}T(z)=\oint_z du\, \epsilon(u)T(u)T(z)
\end{equation}

If $w(z)=z+\epsilon(z)$ where $\epsilon$ is small, $T$ transforms as following:
\begin{equation}
  \label{eq:58}
  T(z)\to \epsilon T'(z) + 2\epsilon' T(z)+\frac{C}{12}\epsilon'''
\end{equation}
\begin{equation}
  \label{eq:59}
  \delta_{\epsilon}T(z)=\oint du\, \epsilon(u) T(u) T(z) = \epsilon T'(z) + 2\epsilon' T(z)+\frac{C}{12}\epsilon'''
\end{equation}
We suppose that for full set of operators $\{A_j(x)\}$ there exists operator product expansion:
!!! How can it be proved ????
\begin{equation}
  \label{eq:60}
  A_i(x)A_j(0)=\sum C^k_{ij}(x)A_k(0)
\end{equation}
Then for product $T(u)T(z)$ we have
\begin{equation}
  \label{eq:61}
  T(u)T(z)=\sum C_k(u-z) A_k(z)
\end{equation}
We can recover the poles from formula (\ref{eq:59}):
\begin{equation}
  \label{eq:62}
  T(u)T(z)=\frac{C}{2(u-z)^4}+\frac{2T(z)}{(u-z)^2}+\frac{T'(z)}{u-z}+\mbox{non-singular terms}
\end{equation}
Now introduce the operators $L_n$:
\begin{equation}
  \label{eq:63}
  L_n A(z)\equiv \frac{}{2\pi i}\oint_{C}du\, (u-z)^{n+1}T(u)A(z)
\end{equation}
It is possible to show that for $n,m\in \mathbb{Z}$
\begin{equation}
  \label{eq:64}
  [L_n,L_m]=(n-m) L_{n+m}+\frac{C}{12}(n^3-n)\delta_{n,-m}
\end{equation}
and $\{L_n\}$ constitute Virasoro algebra. 

All the fields in the theory are sums of multiplets of Virasoro algebra:
\begin{equation}
  \label{eq:65}
  \{A_j\}=\sum [\Phi_{\Delta,A}] %%% It is incorrect, fix this formula
\end{equation}

There is a primary field in every multiplet:
\begin{equation}
  \label{eq:66}
  \begin{split}
    \Phi_{\Delta}(z)\underset{z\to w(z)}{\longrightarrow} \left(\frac{dw}{dz}\right)^{\Delta}\Phi_{\Delta}(w(z))\\
    L_n \Phi=0,\quad n>0\\
    L_0 \Phi=\Delta \Phi\\
  \end{split}
\end{equation}
All other fields are built of primary fields and are called secondary:
\begin{equation}
  \label{eq:67}
  L_{-n_1}L_{-n_2}\dots \Phi_{\Delta}
\end{equation}

Correlation functions are expressed as combinations of correlation functions of primary fields.

The only thing that is yet undefined is the set of scaling dimensions of primary fields. So to fix the theory we should define $\Phi_a,\; \Delta_a,\; C^c_{ab}$. Then our theory is solved. If the number of primary fields is finite, such a model is called {\it minimal}.

For example, following fields are Weyl-invariant:
\begin{equation}
  \label{eq:68}
  O_a=\int \Phi_{\Delta}e^{2a\phi}\,d^2x,\quad \mbox{where}\;\Delta+a(Q-a)=1
\end{equation}
$O_a$ - is the main observable of string theory with gravitational dimension $\delta_a=-\frac{a}{b}$.
\begin{equation}
  \label{eq:69}
  \int\langle O_{a_1}\dots O_{a_n}\rangle_{\mu} e^{-\mu A}d\mu=\langle O_{a_1}\dots O_{a_n}\rangle_{A}=A^{\sum \delta_k},\quad \delta_k=\frac{a_k}{b}
\end{equation}
\subsection{Connection with string theory}

We can start from arbitrary conformal field theory, find $b$ from equation $C+6\left(b+\frac{1}{b}\right)^2=25$ and obtain some non-critical string theory. But for $d>1$ $b$ becomes complex and this fact leads to unsolved yet problems.


\subsection{Wess-Zumino-Novikov-Witten models}
\label{sec:WZNW}

Let's start with non-linear $\sigma$-model.
\begin{equation}
  \label{eq:48}
  S_0=\frac{1}{4a^2}\int d^2x\; Tr' (\partial^{\mu}g^{-1}\partial_{\mu}g)
\end{equation}
Here $a^2>0$ is a positive parameter, $g(x)\in G$ - field on Lie group $G$, which we will assume to be semi-simple and by $Tr'$ we have denoted representation independent trace of Lie algebra $\mathfrak{g} $
\begin{equation}
  \label{eq:70}
  Tr'=\frac{1}{x_{rep}}Tr
\end{equation}
$x_{rep}$ is the Dynkin index of the representation. 

In this model the conformal invariance is lost at the quantum level, as it can be seen along the lines of previous sections.
!!! Show it by example !!!
    
Also we can see that holomorphic and antiholomorphic currents aren't conserved individually. The equations of motion are
\begin{equation}
  \label{eq:71}
  \partial^{\mu}(g^{-1}\partial_{\mu}g)=0
\end{equation}
The currents are
\begin{equation}
  \label{eq:currents}
  J_{\mu}=g^{-1}\partial_{\mu}g
\end{equation}
or in the complex coordinates
\begin{equation}
  \label{eq:74}
  \begin{matrix}
    & \tilde{J}_z=g^{-1}\partial_z g, & \tilde{J}_{\bar{z}}=g^{-1}\partial_{\bar{z}}g\\
    & \partial_z \tilde{J}_{\bar{z}}+\partial_{\bar{z}}\tilde{J}_z=0 & \\
  \end{matrix}
\end{equation}
The terms of the equation of motion can not be null separately because $\partial_{\mu}(\epsilon^{\mu\nu}J_{\nu})\neq 0$.

So we add WZ-term to the action and redefine the currents in more appropriate way
\begin{equation}
  \label{eq:72}
  J_z=\partial_z g\;g^{-1}, \qquad J_{\bar{z}}=g^{-1}\partial{\bar z}g
\end{equation}

Wess-Zumino term has the following form
\begin{equation}
  \label{eq:73}
\Gamma=  - \frac{i }{24\pi} \int_{B}\epsilon_{ijk} Tr'\left(
    \tilde g^{-1}\frac{\partial \tilde g}{\partial y^i}
      \tilde g^{-1}\frac{\partial \tilde g}{\partial y^j}
      \tilde g^{-1}\frac{\partial \tilde g}{\partial y^k}\right) d^3y
\end{equation}

This term is defined on the three-dimensional manifold $B$ whose boundary is original two-dimensional space. By $\tilde{g}$ we have denoted the extension of the field $g$ to $B$. This extension isn't unique. In a compactified three-dimensional space a compact two-dimensional manifold delimits two three-dimensional manifolds. The difference between this two choices $\Delta\Gamma$ is given by the right hand side of (\ref{eq:73}) with the integration range extended over the whole compact three-dimensional space. Since the latter is topologically equivalent to the three-sphere we have
\begin{equation}
  \label{eq:75}
\Delta\Gamma=  - \frac{i }{24\pi} \int_{S^3}\epsilon_{ijk} Tr'\left(
    \tilde g^{-1}\frac{\partial \tilde g}{\partial y^i}
      \tilde g^{-1}\frac{\partial \tilde g}{\partial y^j}
      \tilde g^{-1}\frac{\partial \tilde g}{\partial y^k}\right) d^3y
\end{equation}
$\Delta\Gamma$ is defined modulo $2\pi i$, so the Euclidean functional integral with weight $exp(-\Gamma)$ is well-defined. Clearly, any coupling constant multiplying this term must be an integer. 

We then consider the action
\begin{equation}
  \label{eq:76}
  S=S_0+k\Gamma
\end{equation}
where $k$ is an integer. 
The equation of motion for the full action (\ref{eq:76}) is
\begin{equation}
  \label{eq:77}
  \partial^{\mu}(g^{-1}\partial_{\mu}g)+\frac{a^2 ik}{4\pi}\epsilon_{\mu\nu}\partial^{\mu}(g^{-1}\partial^{\nu}g)=0
\end{equation}
It can be rewritten using complex coordinates as
\begin{equation}
  \label{eq:78}
  (1+\frac{a^2 k}{4\pi})\partial_z(g^{-1}\partial_{\bar z}g)+(1-\frac{a^2 k}{4\pi})\partial_{\bar z}(g^{-1}\partial_z g)=0
\end{equation}
We see that when $a^2=\frac{4\pi}{k}$ we have the desired conservation law
\begin{equation}
  \label{eq:79}
  \partial_z(g^{-1}\partial{\bar z}g)=0
\end{equation}
For currents
\begin{equation}
  \label{eq:4}
  \partial_{\bar z}J=0,\quad \partial_z \bar J=0
\end{equation}

The solution of classical equation is
\begin{equation}
  \label{eq:80}
  g(z,\bar z)=f(z)\bar f(\bar z)
\end{equation}
for arbitrary functions $f(z)$ and $\bar f (\bar z)$.

The separate conservation of the currents $J_z,\; J_{\bar z}$ implies the invariance of the action under the transformations
\begin{equation}
  \label{eq:81}
   g(z,\bar z)\to \Omega(z)g(z,\bar z)\bar \Omega^{-1}(\bar z)
\end{equation}
where $\Omega,\;\bar \Omega \in G$. So we have local $G(z)\times G(\bar z)$ invariance. 

In order to move analysis to quantum level we rescale the currents
\begin{equation}
  \label{eq:82}
  J(z)\equiv -k \partial_zg g^{-1}\quad \bar J(\bar z)=k g^{-1}\partial_{\bar z}g
\end{equation}
So the variation of the action under infinitesimal transformation $\Omega=1+\omega,\; \bar \Omega =1+\bar \omega$ is
\begin{equation}
  \label{eq:83}
  \delta_{\omega,\bar\omega}S=\frac{i}{4\pi}\oint dz Tr' (\omega(z)J(z))-\frac{i}{4\pi}\oint d\bar z Tr'(\bar\omega(\bar z)\bar J(\bar z))
\end{equation}
Expanding the currents as
\begin{equation}
  \label{eq:85}
  \begin{aligned}
    J=\sum J^a t^a,\bar J=\sum \bar J^a t^a \\
    \omega=\sum \omega^a t^a\\
  \end{aligned}
\end{equation}
we get
\begin{equation}
  \label{eq:86}
  \delta_{\omega,\bar \omega}S=-\frac{1}{2\pi i}\oint dz \sum\omega^a J^a+\frac{1}{2\pi i} \oint d\bar z \sum \bar \omega^a \bar J^a
\end{equation}
We can also obtain Ward identities $\delta\left< X\right>=\left<(\delta S)X\right>$
\begin{equation}
  \label{eq:87}
  \delta_{\omega,\bar \omega}\left< X \right>=-\frac{1}{2\pi i}\oint dz \sum\omega^a \left< J^a X\right>+
  \frac{1}{2\pi i} \oint d\bar z \sum \bar \omega^a \left< \bar J^a X\right>
\end{equation}
Then for the currents we have
\begin{equation}
  \label{eq:88}
  \delta_{\omega}J=[\omega,J]-k\partial_z\omega,\quad \delta_{\omega}J^a=\sum i f_{abc}\omega^b J^c-k\partial_z\omega^a
\end{equation}
The operator product expansion for currents has the form
\begin{equation}
  \label{eq:89}
  J^a(z) J^b(w) \sim \frac{k\delta_{ab}}{(z-w)^2}+\sum i f_{abc}\frac{J^c(w)}{(z-w)}
\end{equation}
Expanding currents into series we get
\begin{equation}
  \label{eq:90}
  \begin{aligned}
    J^a(z)=\sum_{n\in \mathbb Z}z^{n-1}J^a_n\\    
    \left[J^a_n,J^b_m\right]=\sum_c i f^{abc}J^c_{n+m}+kn\delta^{ab}\delta_{n+m,0}  
  \end{aligned}
\end{equation}
We see that components of currents realize affine Lie algebra $\hat g$.  


Energy-momentum tensor is introduced with Sugawara construction as the sum of normal-ordered products of current components
\begin{equation}
  \label{eq:6}
  T(z)=\frac{1}{2(k+h^v)}\sum_a N(J^a J^a)(z)
\end{equation}
Here $h^v$ is the dual Coxeter number.

Energy-momentum tensor can be expanded into modes $L_n$
\begin{equation}
  \label{eq:91}
  L_n=\frac{1}{2(k+h^v)}\sum_a\sum_m:J^a_m J^a_{n-m}:
\end{equation}
The commutation relations for the modes $L_n$ are
\begin{equation}
  \label{eq:92}
  \begin{aligned}
    \left[L_n,L_m\right]=(n-m)L_{n+m}+\frac{c}{12}(n^3-n)\delta_{n+m,0}\\
    \left[L_n,J^a_m\right]=-mJ^a_{n+m}
  \end{aligned}
\end{equation}

So the Sugawara construction is the way to embed Virasoro algebra into the universal enveloping of affine Lie algebra. 

Complete chiral algebra of WZW-model is equal to semidirect product $Vir\ltimes \hat g$

\section{Affine Lie algebras representations}
\label{sec:affine-lie-algebras}

\subsection{Verma modules}
\label{sec:verma-modules}

Verma module is a simple thing. It can be easily understood in the framework of induced representations. 

Verma module with highest weight $\lambda$ of algebra $g$ is the space $U(g)\underset{U(b_{+})}{\otimes} D_{\lambda}(b_{+})$, where $\underset{U(b_{+})}{\otimes}$ means that the action of elements of $U(b_{+})$ ``falls through'' left part of tensor product onto the right part.

Here $b_{+}$ is Borel sub-algebra ($x_{+},h$ for $sl(2)$), $D_{\lambda}(b_{+})$ is special representation of $b_{+}$, such that $D(x_{+})=0,\; D(h)=\lambda(h)$.
$g$ acts on the representation space from the left, it means that we should expand element of $g$ in some base, and then commute all the elements of $b_{+}$ to the right, so that they can act on space $D_{\lambda}(b_{+})$.



The infinite set of fields (\ref{eq:65}) corresponds to Verma module of algebra, which is Virasoro algebra in the simplest case and semidirect product of Virasoro and Kac-Moody algebra for WZW model. Also it is shown that for Wess-Zumino model the representations of the algebra are in fact the representations of current algebra only, which is Kac-Moody algebra.

Note also, that Viraso algebra belongs to universal enveloping of Kac-Moody algebra.

\subsection{Irreducible representations}
\label{sec:irred-repr}


\bibliography{article}{}
\bibliographystyle{utphys}

\end{document}
