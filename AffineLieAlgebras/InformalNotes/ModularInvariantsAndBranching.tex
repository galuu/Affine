\documentclass[a4paper,12pt]{article}
\usepackage[unicode,verbose]{hyperref}
\usepackage{amsmath,amssymb,amsthm} \usepackage{pb-diagram}
\usepackage{ucs}
\usepackage[utf8x]{inputenc}
\usepackage[russian]{babel}
\usepackage{cmap}
\usepackage{graphicx}
\pagestyle{plain}
\theoremstyle{definition} \newtheorem{Def}{Definition}
\begin{document}
\tableofcontents

\section{Модулярная инвариантность}
\label{sec:modular-invariance}

Если мы изучаем конформную теорию поля на плоскости или на сфере, то мы можем рассматривать голоморфный и антиголоморфный сектора независимо. Если говорить о WZW-моделях, то примарные поля принадлежат тензорному произведению неприводимых представлений аффинной алгебры.

Если мы говорим о применении конформной теории для описания поведения струн, то теория должна быть определена на римановых поверхностях большего рода ($h>0$), чтобы можно было описывать взаимодействия струн. Считается, что для этого необходимо (и, возможно, достаточно \cite{gaberdiel2000icf}) чтобы теория была определена на торе. 

В теории критического поведения конформная инвариантность имеет место только в критической точке, где голоморфный и антиголоморфный сектора расцеплены. Но вблизи критической точки эти сектора должны быть связаны, и так как мы предполагаем плавный переход к критической точке в пространстве параметров, то эта связь должна сохраняться и в критической точке. Физический спектр теории должен плавно меняться, когда мы покидаем критическую точку, и связь голоморфного и антиголоморфного сектора вдали от критической точки должна приводить к ограничениям на набор состояний в критической точке. Этого можно достичь через геометрию, то есть накладывая граничные условия на состояния \cite{difrancesco1997cft}. Здесь естественно рассматривать периодические граничные условия, которые эквивалентны рассмотрению теории на торе.

Если мы наложили периодические граничные условия с периодами $\omega_1, \omega_2,\; \tau=\omega_2/\omega_1$, то статсумма записывается в виде
\begin{equation}
  \label{eq:1}
  Z(\tau)=Tr \exp 2\pi i (\tau (L_0-c/24)-\bar{\tau} (\bar{L}_0-c/24))
\end{equation}
Или, если ввести $q=\exp 2\pi i \tau$
\begin{equation}
  \label{eq:2}
  Z(\tau)=Tr \left (q^{L_0-c/24}\bar{q}^{\bar{L}_0-c/24}\right)
\end{equation}
Двумерный тор представляет собой фактор пространство $\mathbb{R}^2\approx \mathbb{C}$ по отношениям эквивалентности $z\sim z+w_1$ and $z\sim z+w_2$, где $w_1$ и $w_2$ не параллельны. 

Разные параметризации тора связаны модулярными преобразованиями, таким образом возникает требование модулярной инвариантности статсуммы.

Комплексная структура такого тора конформно эквивалентна тору, для которого соотношения эквивалентности записываются в виде $z\sim z+1$ и $z\sim z+\tau$, где $\tau$ в верхней полуплоскости $\mathbb{C}$.

Легко видеть, что $\tau$, $T(\tau)=\tau+1$ и $S(\tau)=-\frac{1}{\tau}$ описывают конформно-эквивалентные торы. Отображения $T$ и $S$ порождают группу  $SL(2,\mathbb{Z})/\mathbb{Z}_2$, состоящую из матриц вида
\begin{equation}
  \label{eq:99}
  A=
  \begin{pmatrix}
    a & b\\
    c & d 
  \end{pmatrix}
  \quad\mbox{где}\; a,b,c,d\in\mathbb{Z},\quad ad-bc=1,
\end{equation}
и матрицы $A$ и $-A$ действуют одинаково на $\tau$
\begin{equation}
  \label{eq:100}
  \tau\to A\tau=\frac{a\tau+b}{c\tau+d}
\end{equation}
 $\tau$ называется модулярным параметром, а группа $SL(2,\mathbb{Z})/\mathbb{Z}_2$ --- модулярной группой.

Конформная теория поля задаётся примарными полями $\Phi_a$ с конформными размерностями $\Delta_a$:
\begin{equation}
  \label{eq:3}
  \begin{split}
    \Phi_{a}(z)\underset{z\to w(z)}{\longrightarrow} \left(\frac{dw}{dz}\right)^{\Delta_a}\Phi_{a}(w(z))\\
    L_n \Phi_a=0,\quad n>0\\
    L_0 \Phi_a=\Delta_a \Phi_a\\
  \end{split}
\end{equation}

Примарные поля живут в пространствах $\mathcal{H}_{(i,j)}$, которые представляют собой тензорные произведения неприводимого представления  $\mathcal{H}_j$ киральной алгебры и неприводимого представления $\bar{\mathcal{H}}_{\bar{j}}$ антикиральной алгебры. Тогда статсуммы на торе (\ref{eq:2}) может быть переписана в виде  
\begin{equation}
  \label{eq:4}
  \sum_{(j,\bar j)}\chi_j(q)\bar \chi_{\bar j}(\bar q)
\end{equation}
где $\chi_j$ --- характер пердставления $\mathcal{H}_j$,
\begin{equation}
  \label{eq:5}
  \chi_j(\tau)=Tr_{\mathcal{H}_j}(q^{L_0-\frac{c}{24}})\quad \mbox{где}\; q=e^{2\pi i \tau}
\end{equation}
Характеры переходят друг в друга при модулярных преобразованиях:
\begin{equation}
  \label{eq:107}
  \chi_j\left(-\frac{1}{\tau}\right)=\sum_k S_{jk}\chi_k(\tau)\quad \mbox{и}\quad \chi_j(\tau+1)=\sum_kT_{jk}\chi_k(\tau),
\end{equation}
где $S$ и $T$ --- постоянные матрицы. Это верно для большого класса конформных теорий поля \cite{gaberdiel2000icf}. 

Для WZW-моделей представления определяются старшими весами $\hat \lambda, \hat \xi$. Тогда
\begin{equation}
  \label{eq:6}
  \mathcal{H}=\bigoplus_{\hat \lambda,\hat \xi\in P^{(k)}_{+}}M_{\hat \lambda,\hat \xi} L_{\hat \lambda}\otimes L_{\hat \xi}
\end{equation}
Статсумма даётся выражением
\begin{equation}
  \label{eq:7}
  Z(\tau)=\sum_{\hat \lambda,\hat \xi\in P^{(k)}_{+}} \chi_{\hat \lambda}(\tau)M_{\hat \lambda\hat\xi}\bar \chi_{\hat \xi}(\bar \tau)
\end{equation}
Элементы так называемой матрицы масс $M_{\hat \lambda\hat\xi}$ можно рассматривать как кратности примарных полей с весами $\hat\lambda,\hat \xi$. У них есть следующие свойства: $M_{\hat \lambda\hat\xi}\in \mathbb{Z}_+$, модулярная инвариантность
\begin{equation}
  \label{eq:8}
  \begin{aligned}
    T^{\dagger}MT=S^{\dagger}MS=M,\\
    [M,S]=[M,T]=0,
  \end{aligned}
\end{equation}
и $M_{00}=1$ для единственности вакуума.

Простейший случай диагональной матрицы $M$ соответствует равным голоморфным и антиголоморфным конформным размерностям. Есть несколько способов построения недиагональных модулярных инвариантов из диагональных \cite{difrancesco1997cft}:
\begin{itemize}
\item Метод внешних автоморфизмов
\item Конформное вложение в большую теорию
\item Перестановки Галуа
\end{itemize}
Мы будем обсуждать только конформные вложения.

\section{Конформные вложения}
\label{sec:conformal-embeddings}

Состояния в теории соответствующей алгебре $g$ (с киральной алгеброй $Vir\ltimes \hat g$) имеют вид
\begin{equation}
  \label{eq:9}
  J^{a_1}_{-n_1}J^{a_2}_{-n_2}\dots\left|\lambda\right>\quad n_1\geq n_2\geq\dots>0
\end{equation}
А для подалгебры $\mathfrak{p}\subset\mathfrak{g}$
\begin{equation}
  \label{eq:10}
  \tilde{J}^{a'_1}_{-n_1}\tilde{J}^{a'_2}_{-n_2}\dots\left|\mathcal{P}\lambda\right>
\end{equation}
Здесь $\tilde{J}^{a'_j}_{-n_j}$ --- генераторы $\mathfrak{p}$, а $\mathcal{P}$ --- проекция $\mathfrak{g}$ на $\mathfrak{p}$. Очевидно, что $\mathfrak{g}$-инвариантность вакуума ведёт к его $\mathfrak{p}$-инвариантности, но нет оснований считать, что проекция сохраняет конформную инвариантность. Это легко увидеть из рассмотрения тензора энергии - импульса. Действительно, в тензоре энергии - импульса в виде Сугавары можно выделить часть, составленную из тех комбинаций генераторов $g$, которые входят в $p$. Но есть ещё и остаток. Из-за него действие генераторов Вирасоро на состояния (\ref{eq:10}) будет выводить из этого набора.
Таким образом в ситуации общего положения конформная инвариантность нарушается.

Однако есть исключения. Так для представлений уровня 1 simply-laced алгебр тензор энергии - импульса в форме Сугавары состоит только из генераторов Картана. В этом случае он может быть пере-выражен через генераторы подалгебры и конформная инвариантность сохраняется. Тогда $T_{\hat{\mathfrak{g}}_k}=T_{\hat{\mathfrak{p}}_{\tilde k}}\Rightarrow c(\hat{\mathfrak{g}}_k)=c(\hat{\mathfrak{p}}_{\tilde k})$. Это можно переписать в виде равенства
\begin{equation}
  \label{eq:11}
  \frac{k\;dim\mathfrak{g}}{k+g}=\frac{x_e k\; dim\mathfrak{p}}{x_ek+p}
\end{equation}
Где $x_e$ - индекс вложения, а $g$, $p$ - дуальные числа Кокстера соответствующих алгебр и было использовано равенство $\tilde k=x_e k$.
\begin{equation}
  \label{eq:12}
  x_e=\frac{\left|\mathcal{P}\Theta_g\right|^2}{\left| \Theta_p\right|^2},\quad
  x_e=\sum_{\mu\in P_{+}}b_{\lambda\mu}\frac{x_{\mu}}{x_{\lambda}}
\end{equation}
(Здесь $x_{\lambda}$ - индекс представления со старшим весом $\lambda$:
\begin{equation}
  \label{eq:13}
  x_{\lambda}=\frac{dim \left|\lambda\right|(\lambda,\lambda+\rho)}{2 dim g}
\end{equation})

Вложения, которые удовлетворяют условию (\ref{eq:11}), называются конформными.

Нетрудно показать, что решения уравнения (\ref{eq:11}) существуют только для уровня $k=1$.

(Существует конечное число конформных вложений, они классифицированы).

{\bf Примеры}
\begin{itemize}
\item $su(2)\subset su(3),\; x_e=4$
\item $\hat{su}(2)_{10}\subset\hat{sp}(4)_1$
\item $\hat{su}(2)_{28}\subset(\hat{G_2})_1$
\item $\hat{su}(2)_{16}\oplus\hat{su}(3)\subset (\hat{E_8})_1$
\end{itemize}

\section{Конформные коэффициенты ветвления}
\label{sec:conformal-branching-rules}

Для ветвления аффинных алгебр $\hat{\lambda}\to \bigoplus_{\hat{\mu}}b_{\hat{\lambda}\hat{\mu}}\hat{\mu}$ существуют разные алгоритмы, однако мы можем существенно сократить работу при рассмотрении конформных вложений.

Во-первых, заметим, что если коэффициент $b_{\hat\lambda\hat\mu}$ отличен от нуля, то конечная часть старшего веса $\mu$ модуля $L_{\hat\mu}$ находится в некотором грейде $n$  бесконечномерного модуля $L_{\hat\lambda}$ уровня 1. Сохранение конформной инвариантности ведет к соотношению для конформных размерностей соответствующих полей
\begin{equation}
  \label{eq:14}
  \Delta_{\hat\lambda}+n=\Delta_{\hat\mu},
\end{equation}
которое переписывается в виде

\begin{equation}
  \label{eq:15}
  \frac{(\lambda,\lambda+2\rho)}{2(1+g)}+n=\frac{(\mu,\mu+2\rho)}{2(x_e+p)}
\end{equation}

Используя этот факт можно  легко вычислять правила ветвления. Для этого надо вычислить размерности интегрируемых представлений обеих алгебр  $\mathfrak{g},\mathfrak{p}$ и найти все тройки  $(\lambda,\mu,n)$, удовлетворяющие соотношению (\ref{eq:15}). Затем рассматриваем разложения представления $L_{\hat\lambda}$ в грейде $n$ в сумму неприводимых представлений конечномерной алгебры $\mathfrak{g}$ и выписываем правила ветвления этих представлений на представления подалгебры $\mathfrak{p}$. Коэффициент ветвления $b_{\hat{\lambda},\hat{\mu}}$ --- это сколько раз представление со старшим весом $\mu$ появляется в этом списке.

{\bf Пример}

$\hat{su}(2)_4\subset \hat{su}(3)_1$.
Список конформных размерностей:
\begin{equation}
  \label{eq:16}
  \begin{aligned}
    \hat{su}(2)_4&: h_{[4,0]}=0, h_{[3,1]}=\frac{1}{8}, h_{[2,2]}=\frac{1}{3}, h_{[1,3]}=\frac{5}{8}, h_{[0,4]}=1\\
    \hat{su}(3)_1&: h_{[1,0,0]}=0, h_{[0,1,0]}=h_{[0,0,1]}=\frac{1}{3}    
  \end{aligned}
\end{equation}
Отсюда видно, что
\begin{equation}
  \label{eq:17}
  \begin{aligned}
    \left[1,0,0\right] &\to c_1 [4,0]_0\oplus c_2 [0,4]_1\\
    [0,1,0] &\to c_3 [2,2]_0\\
    [0,0,1] &\to c_4 [2,2]_0
  \end{aligned}
\end{equation}
Где $c_1, c_2, c_3, c_4$ - коэффициенты, которые надо найти, а нижний индекс указывает грейд $n$.

Коэффициенты $c_1, c_3, c_4$ вычисляются из правил ветвления в нулевом грейде. В этом грейде $L_{\hat\lambda}$ содержит только $L_{\lambda}$. Из правил ветвления
\begin{equation}
  \label{eq:18}
  (0,0)\to (0),\quad (1,0)\to (2),\quad (0,1)\to (2)
\end{equation}
мы получаем, что $c_1=c_3=c_4=1$. $c_2$ вычисляется из представления грейда 1, содержащего конечномерное представление $(1,1)$ с правилом ветвления
\begin{equation}
  \label{eq:19}
  (1,1)\to (4)\oplus (2)
\end{equation}
То есть $c_2$ тоже равен 1.

После нахождения коэффициентов ветвления, недиагональные модулярные инварианты строятся путем подстановки соотношений для характеров. Например, зная коэффициенты ветвления для вложения $\hat{su}(2)_{28}\subset (\hat{G_2})_1$
  \begin{equation}
    \label{eq:20}
    \begin{aligned}
      & [1,0,0]\to [28,0]\oplus [18,10]\oplus [10,18]\oplus [0,28]\\
      & [0,0,1]\to [22,6]\oplus [16,12]\oplus [12,16]\oplus [6,22]\\
    \end{aligned}
  \end{equation}
мы получаем следующую модулярно-инвариантную статсумму:
\begin{equation}
  \label{eq:21}
  Z=\left|\chi_{[28,0]}+\chi_{[18,10]}+\chi_{[10,18]}+\chi_{[0,28]}\right|^2+\left|\chi_{[22,6]}+\chi_{[16,12]}+\chi_{[12,16]}+\chi_{[6,22]}\right|^2
\end{equation}

\bibliography{CFTNotes}{}
\bibliographystyle{utphys}

\end{document}
