\documentclass[a4paper,12pt]{article}
\usepackage[unicode,verbose]{hyperref}
\usepackage{amsmath,amssymb,amsthm} \usepackage{pb-diagram}
\usepackage{ucs}
%\usepackage[utf8x]{inputenc}
%\usepackage[russian]{babel}
\usepackage{cmap}
\usepackage[pdftex]{graphicx}
\pagestyle{plain}
\theoremstyle{definition} \newtheorem{Def}{Definition}
%\usepackage{verbatim} 
\newenvironment{comment}
{\par\noindent{\bf TODO}\\}
{\\\hfill$\scriptstyle\blacksquare$\par}

\begin{document}
\title{\textbf{{\Large {Recurrent relation for branching coefficients of affine Lie algebras}}}}
\author{Vladimir Lyakhovsky\\
Theoretical Department, SPb State University,\\
198904, Sankt-Petersburg, Russia \\
e-mail:lyakh1507@nm.ru \\
[5mm] Anton Nazarov \\
%EndAName
Theoretical Department, SPb State University,\\
198904, Sankt-Petersburg, Russia \\
e-mail:antonnaz@gmail.com\\
}
\maketitle

\begin{abstract}

\end{abstract}


\section{Introduction}
\label{sec:introduction}

\subsection{Notation}
\label{sec:notation}

Consider the affine Lie algebras $\frak{g}$ and $\frak{a}$ with the
underlying finite-dimensional subalgebras $\overset{\circ }{\frak{g}}$ and $%
\overset{\circ }{\frak{a}}$ and an injection $\frak{a}\longrightarrow \frak{g%
}$ such that $\frak{a}$ is a reductive subalgebra $\frak{a\subset g}$ with
correlated root spaces: $\frak{h}_{\frak{a}}^{\ast }\subset \frak{h}_{\frak{g%
}}^{\ast }$ and $\frak{h}_{\overset{\circ }{\frak{a}}}^{\ast }\subset \frak{h%
}_{\overset{\circ }{\frak{g}}}^{\ast }$\ .

We use the following notations adopted from the paper \cite{ilyin812pbc}.

$L^{\mu }$\ $\left( L_{\frak{a}}^{\nu }\right) $\ -- the integrable module
of $\frak{g}$ with the highest weight $\mu $\ ; (resp. integrable $\frak{a}$
-module with the highest weight $\nu $ );

$r$ , $\left( r_{\frak{a}}\right) $ -- the rank of the algebra $\frak{g}$ $%
\left( \text{resp. }\frak{a}\right) $ ;

$\Delta $ $\left( \Delta _{\frak{a}}\right) $-- the root system; $\Delta
^{+} $ $\left( \text{resp. }\Delta _{\frak{a}}^{+}\right) $-- the positive
root system (of $\frak{g}$ and $\frak{a}$ respectively);

$\mathrm{mult}\left( \alpha \right) $ $\left( \mathrm{mult}_{\frak{a}}\left(
\alpha \right) \right) $ -- the multiplicity of the root $\alpha$ in $\Delta 
$ (resp. in $\left( \Delta _{\frak{a}}\right) $);

$\overset{\circ }{\Delta }$ , $\left( \overset{\circ }{\Delta _{\frak{a}}}%
\right) $ -- the finite root system of the subalgebra $\overset{\circ }{%
\frak{g}}$ (resp. $\overset{\circ }{\frak{a}}$);

$\mathcal{N}^{\mu }$ , $\left( \mathcal{N}_{\frak{a}}^{\nu }\right) $ -- the
weight diagram of $L^{\mu }$ $\left( \text{resp. }L_{\frak{a}}^{\nu }\right) 
$ ;

$W$ , $\left( W_{\frak{a}}\right) $-- the corresponding Weyl group;

$C$ , $\left( C_{\frak{a}}\right) $-- the fundamental Weyl chamber;

$\rho $\ , $\left( \rho _{\frak{a}}\right) $\ -- the Weyl vector;

$\epsilon \left( w\right) :=\det \left( w\right) $ ;

$\alpha _{i}$ , $\left( \alpha _{\left( \frak{a}\right) j}\right) $ -- the $i
$-th (resp. $j$-th) basic root for $\frak{g}$ $\left( \text{resp. }\frak{a}%
\right) $; $i=0,\ldots ,r$ ,\ \ $\left( j=0,\ldots ,r_{\frak{a}}\right) $;

$\delta $ -- the imaginary root of $\frak{g}$ (and of $\frak{a}$ if any);

$\alpha _{i}^{\vee }$ , $\left( \alpha _{\left( \frak{a}\right) j}^{\vee
}\right) $-- the basic coroot for $\frak{g}$ $\left( \text{resp. }\frak{a}%
\right) $ , $i=0,\ldots ,r$ ;\ \ $\left( j=0,\ldots ,r_{\frak{a}}\right) $;

$\overset{\circ }{\xi }$ , $\overset{\circ }{\xi _{\left( \frak{a}\right) }}$
-- the finite (classical) part of the weight $\xi \in P$ , $\left( \text{%
resp. }\xi _{\left( \frak{a}\right) }\in P_{\frak{a}}\right) $\ ;

$\lambda =\left( \overset{\circ }{\lambda };k;n\right) $ -- the
decomposition of an affine weight indicating the finite part $\overset{\circ 
}{\lambda }$, level $k$ and grade $n$\ .

$P$ $\left( \text{resp. } P_{\frak{a}}\right) $ \ -- the weight lattice;

$M \left( \text{resp. }M_{\frak{a}}\right) :=$

\noindent $=\left\{ 
\begin{array}{c}
\sum_{i=1}^{r}\mathbf{Z}\alpha _{i}^{\vee }\text{ }\left( \text{resp. }%
\sum_{i=1}^{r}\mathbf{Z}\alpha _{\left( \frak{a}\right) i}^{\vee }\right) 
\text{for untwisted algebras or }A_{2r}^{\left( 2\right) }, \\ 
\sum_{i=1}^{r}\mathbf{Z}\alpha _{i}\text{ }\left( \text{resp. }\sum_{i=1}^{r}%
\mathbf{Z}\alpha _{\left( \frak{a}\right) i}\right) \text{for }A_{r}^{\left(
u\geq 2\right) }\text{ and }A\neq A_{2r}^{\left( 2\right) },
\end{array}
\right\} ;$

$\mathcal{E}$\ , $\left( \mathcal{E}_{\frak{a}}\right) $-- the group algebra
of the group $P$ (resp. $P_{\frak{a}} $);

$\Theta _{\lambda }:=e^{-\frac{\left| \lambda \right| ^{2}}{2k}\delta
}\sum\limits_{\alpha \in M}e^{t_{\alpha }\circ \lambda }$ -- the classical
theta-function;

$\Theta _{\left( \frak{a}\right) \nu }:=e^{-\frac{\left| \nu \right| ^{2}}{%
2k_{\frak{a}}}\delta }\sum\limits_{\beta \in M_{\frak{a}}}e^{t_{\beta }\circ
\nu }$;\newline
notice that when the injection is considered the level $k_{\frak{a}}$ must
be correlated with the corresponding rescaling of roots;

$A_{\lambda }:=\sum\limits_{s\in \overset{\circ }{W}}\epsilon (s)\Theta
_{s\circ \lambda }$ $\left( \text{resp. }A_{\left( \frak{a}\right) \nu
}:=\sum\limits_{s\in \overset{\circ }{W_{\frak{a}}}}\epsilon (s)\Theta
_{\left( \frak{a}\right) s\circ \nu }\right) $;

$\Psi ^{\left( \mu \right) }:=e^{\frac{\left| \mu +\rho \right| ^{2}}{2k}%
\delta \ -\ \rho }A_{\mu +\rho }=e^{\frac{\left| \mu +\rho \right| ^{2}}{2k}%
\delta \ -\ \rho }\sum\limits_{s\in \overset{\circ }{W}}\epsilon (s)\Theta
_{s\circ \left( \mu +\rho \right) }=$

\noindent $=\sum\limits_{w\in W}\epsilon (w)e^{w\circ (\mu +\rho )-\rho }$
-- the singular weight element for the $\frak{g}$-module $L^{\mu }$;

$\Psi _{\left( \frak{a}\right) }^{\left( \nu \right) }:=e^{\frac{\left| \nu
+\rho _{_{\frak{a}}}\right| ^{2}}{2k_{\frak{a}}}\delta \ -\ \rho _{_{\frak{a}%
}}}A_{\left( \frak{a}\right) \nu +\rho _{_{\frak{a}}}}=e^{\frac{\left| \nu
+\rho _{_{\frak{a}}}\right| ^{2}}{2k_{\frak{a}}}\delta \ -\ \rho _{_{\frak{a}%
}}}\sum\limits_{s\in \overset{\circ }{W_{\frak{a}}}}\epsilon (s)\Theta
_{\left( \frak{a}\right) s\circ \left( \nu +\rho _{_{\frak{a}}}\right) }=$

\noindent $=\sum\limits_{w\in W_{\frak{a}}}\epsilon (w)e^{w\circ (\nu +\rho
_{_{\frak{a}}})-\rho _{_{\frak{a}}}}$ -- the corresponding singular weight
element for the $\frak{a}$-module $L_{\frak{a}}^{\nu }$;

$\widehat{\Psi ^{\left( \mu \right) }}$ $\left( \widehat{\Psi _{\left( \frak{%
a}\right) }^{\left( \nu \right) }}\right) $ -- the set of singular weights $%
\xi \in P$ $\left( \text{resp. }\in P_{\frak{a}}\right) $ for the module $%
L^{\mu }$ $\left( \text{resp. }L_{\frak{a}}^{\nu }\right) $ with the
coordinates $\left( \overset{\circ }{\xi },k,n,\epsilon \left( w\left( \xi
\right) \right) \right) \mid _{\xi =w\left( \xi \right) \circ (\mu +\rho
)-\rho },$ (resp. $\left( \overset{\circ }{\xi },k,n,\epsilon \left(
w_{a}\left( \xi \right) \right) \right) \mid _{\xi =w_{a}\left( \xi \right)
\circ (\nu +\rho _{a})-\rho _{a}}$ ), (this set is similar to $P_{\mathrm{%
nice}}^{\prime }\left( \mu \right) $ in \cite{wakimoto2001idl})

$m_{\xi }^{\left( \mu \right) }$ , $\left( m_{\xi }^{\left( \nu \right)
}\right) $ -- the multiplicity of the weight $\xi \in P$ \ $\left( \text{%
resp. }\in P_{\frak{a}}\right) $ in the module $L^{\mu }$ , (resp. $\xi \in
L_{\frak{a}}^{\nu } $);

$ch\left( L^{\mu }\right) $ $\left( \text{resp. }ch\left( L_{\frak{a}}^{\nu
}\right) \right) $-- the formal character of $L^{\mu }$ $\left( \text{resp. }%
L_{\frak{a}}^{\nu }\right) $;

$ch\left( L^{\mu }\right) =\frac{\sum_{w\in W}\epsilon (w)e^{w\circ (\mu
+\rho )-\rho }}{\prod_{\alpha \in \Delta ^{+}}\left( 1-e^{-\alpha }\right) ^{%
\mathrm{{mult}\left( \alpha \right) }}}=\frac{\Psi ^{\left( \mu \right) }}{%
\Psi ^{\left( 0\right) }}$ -- the Weyl-Kac formula.

$R:=\prod_{\alpha \in \Delta ^{+}}\left( 1-e^{-\alpha }\right) ^{\mathrm{{%
mult}\left( \alpha \right) }}=\Psi ^{\left( 0\right) }\quad $

\noindent $\left( \text{resp. }R_{\frak{a}}:=\prod_{\alpha \in \Delta _{%
\frak{a}}^{+}}\left( 1-e^{-\alpha }\right) ^{\mathrm{mult}_{\frak{a}}\mathrm{%
\left( \alpha \right) }}=\Psi _{\frak{a}}^{\left( 0\right) }\right) $-- the
denominator.


\section{Recurrent formula for branching coefficients}
\label{sec:recurr-form-branch}

\begin{equation}
  \label{eq:1}
  k_{\nu}=\sum_{w\in W_a} k_{\rho_a-w(\nu+\rho_a)} + \sum_{w\in W}dim_{w(\mu+\rho)-\rho} \delta_{\nu,w(\mu+\rho)-\rho}
\end{equation}

\subsection{Proof}
\label{sec:proof}
Proof of the formula.

The decomposition of the representation of the algebra to the representations of the subalgebra can be symbolically written using formal characters and projection operator:
\begin{equation}
  \label{eq:3}
  L_{\frak{g}\downarrow \frak{a}}^{\mu }=\bigoplus\limits_{\nu \in P_{\frak{a}%
    }^{+}}b_{\nu }^{\left( \mu \right) }L_{\frak{a}}^{\nu }\quad
  \Longrightarrow\quad
  \pi_{\mathfrak{a}}(ch L^{\mu}_{\mathfrak{g}})=\sum_{\nu\in P^{+}_{\mathfrak{a}}}b^{(\mu)}_{\nu} ch L^{\nu}_{\mathfrak{a}}
\end{equation}
Now using the Weyl-Kac formula for the character of the module
\begin{equation}
  \label{eq:2}
  ch L^{\mu}=\frac{\sum_{\omega\in W} \epsilon(\omega) e^{\omega(\mu+\rho)-\rho}}{\prod_{\alpha\in\Delta^{+}}(1-e^{-\alpha})^{mult(\alpha)}}
\end{equation}
we obtain the equality
\begin{equation}
  \label{eq:4}
  \pi_{\mathfrak{a}}\left(\frac{\sum_{\omega\in W} \epsilon(\omega) e^{\omega(\mu+\rho)-\rho}}{\prod_{\alpha\in\Delta^{+}}(1-e^{-\alpha})^{mult(\alpha)}}\right) = 
  \sum_{\nu\in P^{+}_{\mathfrak{a}}}b^{(\mu)}_{\nu}
  \frac{\sum_{\omega\in W_{\mathfrak{a}}}\epsilon(\omega)e^{\omega(\nu+\rho_{\mathfrak{a}})-\rho_{\mathfrak{a}}}}{\prod_{\beta\in \Delta_{\mathfrak{a}}^{+}}(1-e^{-\beta})^{mult_{\mathfrak{a}}(\beta)}}
\end{equation}

It is important to mention that the projection of some of the positive roots of the algebra $\mathfrak{g}$ can be equal to zero. These roots are orthogonal to the root space of the subalgebra $\mathfrak{a}$ embedded into the root space of the algebra $\mathfrak{g}$. Let's denote the subset of these roots by $\Delta_{\bot}=\left\{\alpha\in\Delta_{\mathfrak{g}}^{+}:\forall \beta\in \Delta_{\mathfrak{a}}^{+},\; \alpha\bot\beta \right\}$.

Now we should notice that if the set $\Delta_{\bot}$ is non-empty than Weyl reflections which correspond to the positive roots of $\Delta_{\bot}$ generate a subgroup $W_{\bot}$ of Weyl group $W$. 

Let's denote the subalgebra with the root space spanned over the set $\Delta_{\bot}$ by $\mathfrak{a}_{\bot}$.

Now we should discuss when the subset $\Delta_{\bot}$ is non-empty and the subgroup $W_{\bot}$ and subalgebra $\mathfrak{a}_{\bot}$ are non-trivial.

If $\mathfrak{a}$ is a maximal regular subalgebra of $\mathfrak{g}$ then rank of $\mathfrak{a}$ is equal to the rank of $\mathfrak{g}$ and it is clear that $\Delta_{\bot}$ is empty.
Then the modules $L_{\mathfrak{a}_{\bot}}$ are trivial, the dimensions are equal to 1 and we get the formula (11) from the paper \cite{ilyin812pbc}.

Non-maximal regular embedding of $\mathfrak{a}$ into $\mathfrak{g}$ can be obtained through the chain of maximal embeddings $\mathfrak{a}\subset \mathfrak{p}_1\subset \mathfrak{p}_2\subset\dots \subset \mathfrak{g}$. Also the maximal regular embeddings are constructed by the exclusion of one or two roots from the extended Dynkin diagram of the algebra. Since this process can give us non-connected Dynkin diagrams we can see which roots are orthogonal to the root space of non-maximal regular subalgebra $\mathfrak{a}$. 

Consider for example regular embedding of $A_1\subset B_2$ ($su(2)\subset so(5)$). 

The extended Dynkin diagram of $B_2$ 
\begin{equation}
  \label{eq:8}
  \begin{diagram}
    \node{\circ}\arrow{e}{}\node{\circ}\arrow{e}{}\node{\circ}\\ \\
    \node{\circ}\node[2]{\circ}
  \end{diagram}
\end{equation}
We then drop central node and get the embedding $A_1\otimes A_1\subset B_2$. Then $\mathfrak{a}=A_1$ and $\mathfrak{a}_{\bot}=A_1$.
\begin{comment}
  Consider special embeddings
\end{comment}

Now we can multiply the equation (\ref{eq:4}) by the term
\begin{equation}
  \label{eq:5}
  \pi_{\mathfrak{a}}\left(\prod_{\alpha\in \Delta^{+}\setminus \Delta_{\bot}}(1-e^{-\alpha})^{mult_{\mathfrak{g}}(\alpha)} \right)
\end{equation}
This term is non-zero. 

Also we can see that 
\begin{equation}
  \label{eq:6}
  \pi_{\mathfrak{a}} (P) \pi_{\mathfrak{a}}(1-e^{-\alpha})=\pi_{\mathfrak{a}}\left(P\cdot (1-e^{-\alpha})\right)
\end{equation}
The equation (\ref{eq:4}) takes the form
\begin{multline}
  \label{eq:7}
  \pi_{\mathfrak{a}}\left(\frac{\sum_{\omega\in W} \epsilon(\omega) e^{\omega(\mu+\rho)-\rho}}{\prod_{\alpha\in\Delta_{\bot}}(1-e^{-\alpha})^{mult(\alpha)}}\right) = \\
  \pi_{\mathfrak{a}}\left(\prod_{\alpha\in \Delta^{+}\setminus \Delta_{\bot}}(1-e^{-\alpha})^{mult_{\mathfrak{g}}(\alpha)} \right)\sum_{\nu\in P^{+}_{\mathfrak{a}}}b^{(\mu)}_{\nu}
  \frac{\sum_{\omega\in W_{\mathfrak{a}}}\epsilon(\omega)e^{\omega(\nu+\rho_{\mathfrak{a}})-\rho_{\mathfrak{a}}}}{\prod_{\beta\in \Delta_{\mathfrak{a}}^{+}}(1-e^{-\beta})^{mult_{\mathfrak{a}}(\beta)}}
\end{multline}
The right-hand side of this equation can be reorganised as in the paper \cite{ilyin812pbc}. We introduce anomalous branching coefficients $k_{\lambda}$.
\begin{equation}
  \label{eq:10}
  \sum_{\nu \in P_{\frak{a}}}b_{\nu }^{\left( \mu \right) }\Psi _{\left( \frak{%
        a}\right) }^{\left( \nu \right) }=\sum_{\lambda \in P_{\frak{a}}}k_{\lambda
  }^{\left( \mu \right) }e^{\lambda } 
\end{equation}
Also we extract the common denominator of  the right-hand side of the equation (\ref{eq:7})
\begin{multline}
  \label{eq:12}
  \pi_{\mathfrak{a}}\left(\frac{\sum_{\omega\in W} \epsilon(\omega) e^{\omega(\mu+\rho)-\rho}}{\prod_{\alpha\in\Delta_{\bot}}(1-e^{-\alpha})^{mult(\alpha)}}\right) = \\
  \frac{\pi_{\mathfrak{a}}\left(\prod_{\alpha\in \Delta^{+}\setminus \Delta_{\bot}}(1-e^{-\alpha})^{mult_{\mathfrak{g}}(\alpha)} \right)}
  {  \prod_{\beta\in \Delta_{\mathfrak{a}}^{+}}(1-e^{-\beta})^{mult_{\mathfrak{a}}(\beta)}}
\sum_{\lambda \in P_{\frak{a}}}k_{\lambda
}^{\left( \mu \right) }e^{\lambda } 
\end{multline}


If the set $\Delta_{\bot}$ is non-empty then Weyl reflections corresponding to the positive roots of $\Delta_{\bot}$ generate a subgroup $W_{\bot}$ of Weyl group $W$. 

We have denoted the subalgebra with the root space spanned over the set $\Delta_{\bot}$ by $\mathfrak{a}_{\bot}$.

Then we can reorganise the summation on the left-hand side of the equation (\ref{eq:12}) in the following way
\begin{multline}
  \label{eq:13}
 \pi_{\mathfrak{a}}\left(\frac{\sum_{\omega\in W} \epsilon(\omega) e^{\omega(\mu+\rho)-\rho}}{\prod_{\alpha\in\Delta_{\bot}}(1-e^{-\alpha})^{mult(\alpha)}}\right) = \\
 \pi_{\mathfrak{a}}\left(\sum_{\omega\in W/W_{\bot}} \epsilon(\omega) \frac{\sum_{\nu\in W_{\bot}}\epsilon(\nu) e^{\nu \cdot \omega(\mu+\rho)-\rho}}{\prod_{\alpha\in\Delta_{\bot}}(1-e^{-\alpha})^{mult(\alpha)}}\right) 
\end{multline}


Then we see that
\begin{multline}
  \label{eq:14}
  \sum_{\omega\in W/W_{\bot}} \epsilon(\omega) \frac{\sum_{\nu\in W_{\bot}}\epsilon(\nu) e^{\nu \cdot \omega(\mu+\rho)-\rho}}{\prod_{\alpha\in\Delta_{\bot}}(1-e^{-\alpha})^{mult(\alpha)}} =\\
  \sum_{\omega\in W/W_{\bot}} \epsilon(\omega) e^{-\rho} \frac{e^{\rho_{\mathfrak{a}_{\bot}} }\sum_{\nu\in W_{\bot}}\epsilon(\nu) e^{\nu \cdot (\omega(\mu+\rho)-\rho_{\mathfrak{a}_{\bot}}+\rho_{\mathfrak{a}_{\bot}})-\rho_{\mathfrak{a}_{\bot}}}}{\prod_{\alpha\in\Delta_{\bot}}(1-e^{-\alpha})^{mult(\alpha)}}=\\
  \sum_{\omega\in W/W_{\bot}} \epsilon(\omega) e^{\rho_{\mathfrak{a}_{\bot}}-\rho} ch L^{\pi_{\mathfrak{a}_{\bot}}(\omega(\mu+\rho))-\rho_{\mathfrak{a}_{\bot}}}_{\mathfrak{a}_{\bot}}
\end{multline}
The projection $\pi_{\mathfrak{a}}$ of the character of the highest-weight module $L^{\pi_{\mathfrak{a}_{\bot}}(\omega(\mu+\rho))-\rho_{\mathfrak{a}_{\bot}}}_{\mathfrak{a}_{\bot}}$ is equal to the dimension of the module multiplied by $e^{\pi_{\mathfrak{a}}(\omega(\mu+\rho)-\rho_{\mathfrak{a}_{\bot}})}$
  \begin{multline}
    \label{eq:15}
    \pi_{\mathfrak{a}}\left(\sum_{\omega\in W/W_{\bot}} \epsilon(\omega) e^{\rho_{\mathfrak{a}_{\bot}}-\rho} ch L^{\omega(\mu+\rho)-\rho_{\mathfrak{a}_{\bot}}}_{\mathfrak{a}_{\bot}}\right) = \\
    \sum_{\omega\in W/W_{\bot}} \epsilon(\omega) dim\left(L^{\pi_{\mathfrak{a}_{\bot}}(\omega(\mu+\rho))-\rho_{\mathfrak{a}_{\bot}}}_{\mathfrak{a}_{\bot}}\right) e^{\pi_{\mathfrak{a}}(\omega(\mu+\rho)-\rho)}
  \end{multline}
These dimensions of the modules could be easily calculated using Weyl formula.

Thus we have the equality
\begin{multline}
  \label{eq:9}
  \sum_{\omega\in W/W_{\bot}} \epsilon(\omega) dim\left(L^{\pi_{\mathfrak{a}_{\bot}}(\omega(\mu+\rho))-\rho_{\mathfrak{a}_{\bot}}}_{\mathfrak{a}_{\bot}}\right) e^{\pi_{\mathfrak{a}}(\omega(\mu+\rho)-\rho)}=\\
   \frac{\pi_{\mathfrak{a}}\left(\prod_{\alpha\in \Delta^{+}\setminus \Delta_{\bot}}(1-e^{-\alpha})^{mult_{\mathfrak{g}}(\alpha)} \right)}
   {   \prod_{\beta\in \Delta_{\mathfrak{a}}^{+}}(1-e^{-\beta})^{mult_{\mathfrak{a}}(\beta)}}
\sum_{\lambda \in P_{\frak{a}}}k_{\lambda
}^{\left( \mu \right) }e^{\lambda } 
\end{multline}

We can rewrite the fraction on the right-hand side as in the paper \cite{ilyin812pbc}.
\begin{multline}
  \label{eq:11}
    \frac{\pi_{\mathfrak{a}}\left(\prod_{\alpha\in \Delta^{+}\setminus \Delta_{\bot}}(1-e^{-\alpha})^{mult_{\mathfrak{g}}(\alpha)} \right)}
    {   \prod_{\beta\in \Delta_{\mathfrak{a}}^{+}}(1-e^{-\beta})^{mult_{\mathfrak{a}}(\beta)}}=
    \prod_{\alpha\in \pi_{\mathfrak{a}}\circ (\Delta^{+}\setminus \Delta_{\bot})} \left(1-e^{-\alpha}\right)^{mult(\alpha)-mult_{\mathfrak{a}}(\alpha)}=\\
    = -\sum_{\gamma\in P_{\mathfrak{a}}} s(\gamma)e^{-\gamma}
\end{multline}

For the coefficient function $s\left( \gamma \right) $ define $\Phi _{\frak{a%
}\subset \frak{g}}\subset P_{\frak{a}}$ as its carrier: 
\begin{equation}
\Phi _{\frak{a}\subset \frak{g}}=\left\{ \gamma \in P_{\frak{a}}\mid s\left(
\gamma \right) \neq 0\right\} ;  \label{phi-d}
\end{equation}
\begin{equation}
\prod_{\alpha\in \pi_{\mathfrak{a}}\circ (\Delta^{+}\setminus \Delta_{\bot})}\left(
1-e^{-\alpha }\right) ^{\mathrm{{mult}\left( \alpha \right) -{mult}}_{\frak{a%
}}\mathrm{\left( \alpha \right) }}=-\sum_{\gamma \in \Phi _{\frak{a}\subset 
\frak{g}}}s\left( \gamma \right) e^{-\gamma }.  \label{fan-d}
\end{equation}

So we get the equation
\begin{multline}
  \label{eq:16}
  \sum_{\omega\in W/W_{\bot}} \epsilon(\omega) dim\left(L^{\pi_{\mathfrak{a}_{\bot}}(\omega(\mu+\rho))-\rho_{\mathfrak{a}_{\bot}}}_{\mathfrak{a}_{\bot}}\right) e^{\pi_{\mathfrak{a}}(\omega(\mu+\rho)-\rho)}=\\
  = -\sum_{\gamma \in \Phi _{\frak{a}\subset \frak{g}}} s\left( \gamma \right) e^{-\gamma }\sum_{\lambda \in P_{\frak{a}}}
  k_{\lambda }^{\left( \mu \right) }e^{\lambda } \\
  =-\sum_{\gamma \in \Phi _{\frak{a}\subset \frak{g}}}\sum_{\lambda \in P_{\frak{a}}}s\left( \gamma \right) k_{\lambda }^{\left( \mu \right)}e^{\lambda -\gamma }
\end{multline}
From the equality of the coefficients of the equal formal exponents we get
\begin{equation}
  \label{eq:17}
   \sum_{\omega\in W/W_{\bot}} \epsilon(\omega) dim\left(L^{\pi_{\mathfrak{a}_{\bot}}(\omega(\mu+\rho))-\rho_{\mathfrak{a}_{\bot}}}_{\mathfrak{a}_{\bot}}\right) \delta_{\xi,\pi_{\mathfrak{a}}(\omega(\mu+\rho)-\rho)}+
   \sum_{\gamma \in \Phi _{\frak{a}\subset \frak{g}}} s(\gamma)\; k^{(\mu)}_{\xi+\gamma}=0;\quad \xi\in P_{\mathfrak{a}}
\end{equation}

To get the recurrent relation for the anomalous branching coefficients we should use the following procedure, introduced in the paper \cite{ilyin812pbc}.

In $\left( \Phi _{\frak{a}\subset \frak{g}}\right) _{n=0}$ let $\gamma
_{0} $ be the lowest vector with respect to the natural ordering in $%
\overset{\circ }{\Delta _{\frak{a}}}$ . Decomposing the defining relation 
\begin{equation}
  \label{eq:18}
  \prod_{\alpha\in \pi_{\mathfrak{a}}\circ (\Delta^{+}\setminus \Delta_{\bot})}\left(
    1-e^{-\alpha }\right) ^{\mathrm{{mult}\left( \alpha \right) -{mult}}_{\frak{a%
      }}\mathrm{\left( \alpha \right) }}=-s\left( \gamma _{0}\right) e^{-\gamma
    _{0}}-\sum_{\gamma \in \Phi _{\frak{a}\subset \frak{g}}\setminus \gamma
    _{0}}s\left( \gamma \right) e^{-\gamma },  
\end{equation}
in (\ref{eq:17}) we obtain

\begin{equation}
k_{\xi }^{\left( \mu \right) }=-\frac{1}{s\left( \gamma _{0}\right) }\left(
  \sum_{\omega\in W/W_{\bot}} \epsilon(\omega) dim\left(L^{\pi_{\mathfrak{a}_{\bot}}(\omega(\mu+\rho))-\rho_{\mathfrak{a}_{\bot}}}_{\mathfrak{a}_{\bot}}\right) \delta_{\xi,\pi_{\mathfrak{a}}(\omega(\mu+\rho)-\rho)}+
\sum_{\gamma \in
\Gamma _{\frak{a}\subset \frak{g}}}s\left( \gamma +\gamma _{0}\right) k_{\xi
+\gamma }^{\left( \mu \right) }\right)   \label{recurrent relation}
\end{equation}
where the set 
\begin{equation}
\Gamma _{\frak{a}\subset \frak{g}}=\left\{ \xi -\gamma _{0}|\xi \in \Phi _{%
\frak{a}\subset \frak{g}}\right\} \setminus \left\{ 0\right\} .
\label{fan-defined}
\end{equation}


\section{Algorithm}
\label{sec:algorithm}


\section{Examples}
\label{sec:examples}

\subsection{Finite dimensional Lie algebras}
\label{sec:finite-dimens-lie}

B2->A1 etc.

\subsection{Affine Lie algebras}
\label{sec:affine-lie-algebras}

Branching of affine to affine and affine to classical.
\subsection{Physical applications}
\label{sec:phys-appl}

Modular invariants of WZW-models, partition functions etc.

\section{Conclusion}
\label{sec:conclusion}

\bibliography{CFTNotes}{}
\bibliographystyle{utphys}

\end{document}
