\documentclass[a4paper,12pt]{article}
\usepackage[unicode,verbose]{hyperref}
\usepackage{amsmath,amssymb,amsthm} \usepackage{pb-diagram}
\usepackage{ucs}
\usepackage[utf8x]{inputenc}
\usepackage[russian]{babel}
\usepackage{cmap}
\usepackage{graphicx}
\pagestyle{plain}
\theoremstyle{definition} \newtheorem{Def}{Definition}
\begin{document}
Мы хотим раскладывать представление $L^{\mu}$ алгебры
$\frak{g}$ на представления подалгебры $\frak{a}$
\begin{equation}
  \label{eq:1}
  L_{\frak{g}\downarrow \frak{a}}^{\mu }=\bigoplus\limits_{\nu \in P_{\frak{a}%
    }^{+}}b_{\nu }^{\left( \mu \right) }L_{\frak{a}}^{\nu }
\end{equation}
Верна следующая формула для характеров:
\begin{equation}
  \label{eq:2}
  \pi_{\mathfrak{a}}(ch L^{\mu}_{\mathfrak{g}})=\sum_{\nu\in P^{+}_{\mathfrak{a}}}b^{(\mu)}_{\nu} ch L^{\nu}_{\mathfrak{a}}
\end{equation}
Произведение характеров модулей старшего веса равно характеру тензорного произведения модулей.

Характеры инвариантны относительно группы Вейля.
\begin{equation}
  \label{eq:3}
  ch(L^{\omega\mu})=ch(L^{\mu})
\end{equation}

Явное выражение для формальных характеров даётся формулой Вейля-Каца:
\begin{equation}
  \label{eq:4}
  ch(L^{\mu})=\frac{\sum_{\omega\in W}\epsilon(\omega)e^{\omega(\mu+\rho)-\rho}}{\prod_{\alpha\in \Delta^{+}}(1-e^{-\alpha})^{mult(\alpha)}}
\end{equation}
Эту формулу также можно записать в виде
\begin{equation}
  \label{eq:5}
  ch(L^{\mu})=\left( \sum_{\omega\in W}\epsilon(\omega)e^{\omega(\mu+\rho)-\rho} \right) \left(\prod_{\alpha\in \Delta^{+}}(1-e^{-\alpha})^{mult(\alpha)}\right)^{-1}
\end{equation}
Обратный элемент для монома $(1-e^{-\alpha})$ - это бесконечный формальный ряд
\begin{equation}
  \label{eq:6}
  1+e^{-\alpha}+e^{-2\alpha}+e^{-3\alpha}+\dots
\end{equation}
В работе \cite{ilyin812pbc} рекуррентное соотношение для коэффициентов ветвления выводилось следующим образом.
Бралось соотношение \eqref{eq:2}, характеры расписывались по формуле Вейля-Каца \eqref{eq:5}.
\begin{multline}
  \label{eq:7}
  \pi_{\mathfrak{a}}\left(\left(\sum_{\omega\in W}\epsilon(\omega)e^{\omega(\mu+\rho)-\rho}\right) \left(\prod_{\alpha\in \Delta^{+}}(1-e^{-\alpha})^{mult(\alpha)}\right)^{-1}\right)= \\
  \sum_{\nu\in P^{+}_{\mathfrak{a}}}b^{(\mu)}_{\nu}
  \frac{\sum_{\omega\in W_{\mathfrak{a}}}\epsilon(\omega)e^{\omega(\nu+\rho_{\mathfrak{a}})-\rho_{\mathfrak{a}}}}{\prod_{\beta\in \Delta_{\mathfrak{a}}^{+}}(1-e^{-\beta})^{mult_{\mathfrak{a}}(\beta)}}
\end{multline}
Затем обе стороны соотношения умножались на $\pi_{\mathfrak{a}}\left(\prod_{\alpha\in \Delta^{+}}(1-e^{-\alpha})^{mult(\alpha)}\right)$. Так как умножение на моном коммутирует с проектором, то $\prod_{\alpha\in \Delta^{+}}(1-e^{-\alpha})^{mult(\alpha)}$ сокращалось в левой части.

Однако в случае, если $\pi_{\mathfrak{a}}\left(\prod_{\alpha\in \Delta^{+}}(1-e^{-\alpha})^{mult(\alpha)}\right)=0$, получить содержательное рекуррентное соотношение не удается, так как обе стороны равенства \eqref{eq:7} умножаются на ноль.

Это происходит в том случае, если среди положительных корней алгебры $\mathfrak{g}$ есть ортогональные корневой системе вложенной подалгебры $\mathfrak{a}$. Обозначим набор таких корней через $\Delta_{\bot}=\left\{\alpha\in\Delta_{\mathfrak{g}}^{+}:\forall \beta\in \Delta_{\mathfrak{a}}^{+},\; \alpha\bot\beta \right\}$. Решить эту проблему можно если умножать не на $\pi_{\mathfrak{a}}\left(\prod_{\alpha\in \Delta^{+}}(1-e^{-\alpha})^{mult(\alpha)}\right)$, а на $\pi_{\mathfrak{a}}\left(\prod_{\alpha\in \Delta^{+}\setminus \Delta_{\bot}}(1-e^{-\alpha})^{mult(\alpha)}\right)$.

Тогда
\begin{multline}
  \label{eq:9}
  \pi_{\mathfrak{a}}\left(\left(\sum_{\omega\in W}\epsilon(\omega)e^{\omega(\mu+\rho)-\rho}\right) \left(\prod_{\alpha\in \Delta_{\bot}}(1-e^{-\alpha})^{mult(\alpha)}\right)^{-1}\right)= \\
  \sum_{\nu\in P^{+}_{\mathfrak{a}}}b^{(\mu)}_{\nu}
  \frac{\sum_{\omega\in W_{\mathfrak{a}}}\epsilon(\omega)e^{\omega(\nu+\rho_{\mathfrak{a}})-\rho_{\mathfrak{a}}}}{\prod_{\beta\in \Delta_{\mathfrak{a}}^{+}}(1-e^{-\beta})^{mult_{\mathfrak{a}}(\beta)}}
\end{multline}

\bibliography{CFTNotes}{}
\bibliographystyle{utphys}

\end{document}