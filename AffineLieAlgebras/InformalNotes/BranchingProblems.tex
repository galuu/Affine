\documentclass[a4paper,12pt]{article}
\usepackage[unicode,verbose]{hyperref}
\usepackage{amsmath,amssymb,amsthm} \usepackage{pb-diagram}
\usepackage{ucs}
\usepackage[utf8x]{inputenc}
\usepackage[russian]{babel}
\usepackage{cmap}
\usepackage{graphicx}
\pagestyle{plain}
\theoremstyle{definition} \newtheorem{Def}{Definition}
\begin{document}
Мы хотим раскладывать представление $L^{\mu}$ алгебры
$\frak{g}$ на представления подалгебры $\frak{a}$
\begin{equation}
  \label{eq:1}
  L_{\frak{g}\downarrow \frak{a}}^{\mu }=\bigoplus\limits_{\nu \in P_{\frak{a}%
    }^{+}}b_{\nu }^{\left( \mu \right) }L_{\frak{a}}^{\nu }
\end{equation}
Верна следующая формула для характеров:
\begin{equation}
  \label{eq:2}
  \pi_{\mathfrak{a}}(ch L^{\mu}_{\mathfrak{g}})=\sum_{\nu\in P^{+}_{\mathfrak{a}}}b^{(\mu)}_{\nu} ch L^{\nu}_{\mathfrak{a}}
\end{equation}
Произведение характеров модулей старшего веса равно характеру тензорного произведения модулей.

Характеры инвариантны относительно группы Вейля.
\begin{equation}
  \label{eq:3}
  ch(L^{\omega\mu})=ch(L^{\mu})
\end{equation}

Явное выражение для формальных характеров даётся формулой Вейля-Каца:
\begin{equation}
  \label{eq:4}
  ch(L^{\mu})=\frac{\sum_{\omega\in W}\epsilon(\omega)e^{\omega(\mu+\rho)-\rho}}{\prod_{\alpha\in \Delta^{+}}(1-e^{-\alpha})^{mult(\alpha)}}
\end{equation}
Эту формулу также можно записать в виде
\begin{equation}
  \label{eq:5}
  ch(L^{\mu})=\sum_{\omega\in W}\epsilon(\omega)e^{\omega(\mu+\rho)-\rho} \left(\prod_{\alpha\in \Delta^{+}}(1-e^{-\alpha})^{mult(\alpha)}\right)^{-1}
\end{equation}
Обратный элемент для монома $(1-e^{-\alpha})$ - это бесконечный формальный ряд
\begin{equation}
  \label{eq:6}
  1+e^{-\alpha}+e^{-2\alpha}+e^{-3\alpha}+\dots
\end{equation}

\end{document}
