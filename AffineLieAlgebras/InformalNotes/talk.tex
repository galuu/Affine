\documentclass[a4paper,12pt]{article}
\usepackage{ucs}
\usepackage[unicode,verbose]{hyperref}
\usepackage{amsmath,amssymb,amsthm}
\usepackage{pb-diagram}
\usepackage{multicol}
\usepackage[utf8x]{inputenc}
\usepackage[russian]{babel}
\usepackage{cmap}
\usepackage{color}
\usepackage[pdftex]{graphicx}
\pagestyle{plain}

%\usepackage{verbatim} 
\newenvironment{comment}
{\par\noindent{\bf TODO}\\}
{\\\hfill$\scriptstyle\blacksquare$\par}

\newtheorem{statement}{Statement}
\theoremstyle{definition} \newtheorem{Def}{Definition}
\newcommand{\go}{\overset{\circ }{\frak{g}}}
\newcommand{\ao}{\overset{\circ }{\frak{a}}}
\newcommand{\co}[1]{\overset{\circ }{#1}}

\begin{document}
\thispagestyle{empty}

\begin{center}
{\Large \bf SAINT-PETERSBURG STATE\\[0.5cm] UNIVERSITY}
\end{center}
\vskip 1 cm
\begin{flushright}
{\large \bf SPbU-IP-09-08}
\end{flushright}
\hfill {}
\vskip 1 cm
\begin{center}
{\Huge \bf Recursive algorithms, branching coefficients \\ for affine algebras and applications\\[0.5cm]
}
\end{center}
\vskip 1.0 cm
\begin{center}
{\Large \bf
Vladimir Lyakhovsky\\[2mm]
Anton Nazarov}
\end{center}
\vfill
\begin{center}
{\large Department of Theoretical Physics\\
}
\end{center}
\vfill
\begin{center}
SAINT-PETERSBURG \\
2009
\end{center}

\newpage

\pagenumbering{arabic}
\title{\textbf{{\Large {Recursive algorithms, branching coefficients affine algebras and applications}}}}
\author{Vladimir Lyakhovsky \thanks{ Supported by
 RFFI grant N 09-01-00504 and the National Project RNP.2.1.1./1575 }\\
Theoretical Department, SPb State University,\\
198904, Sankt-Petersburg, Russia \\
e-mail:lyakh1507@nm.ru \\
[5mm] Anton Nazarov \thanks{ Supported by
the National Project RNP.2.1.1./1575 }\\
Theoretical Department, SPb State University,\\
198904, Sankt-Petersburg, Russia \\
e-mail:antonnaz@gmail.com
}
\maketitle

\begin{abstract}
\end{abstract}

\section{Постановка задачи}
\label{sec:task}

Здравствуйте!
Сегодня я хочу рассказать о работе ``Рекуррентные алгоритмы, коэффициенты ветвления аффинных алгебр
Ли и приложения''. В этой работе мы решали следующую задачу.

Даны две аффинные алгебры Ли $\mathfrak{a}\subset \mathfrak{g}$. Неприводимое представление
$L^{(\mu)}_{\mathfrak{g}}$ алгебры $\mathfrak{g}$ со старшим весом $\mu$ может быть разложено на
представления подалгебры $\mathfrak{a}$. Наша задача --- придумать практический алгоритм вычисления
коэффициентов этого разложения. 

Я начну с физической мотивации этой задачи в двумерной конформной теории поля, напомню некоторые
факты из теории представлений аффинных алгебр Ли, кратко опишу предложенное нами решение проблемы и
расскажу про физические примеры в нашей статье.

\section{Физическая мотивация}
\label{sec:physics}

\subsection{Напоминание про двумерную конформную теорию поля}
\label{sec:CFT}

В теории струн двумерная конформная теория поля описывает динамику на мировой поверхности, а в
теории критических явлений --- фазовые переходы в двумерных системах.

Двумерная конформная теория поля --- это теория поля, определенная на двумерном пространстве-времени
и обладающая инвариантностью относительно следующих преобразований:
\begin{align}
  \label{eq:1}
  & M_{\mu\nu} \equiv i(x_\mu\partial_\nu-x_\nu\partial_\mu) \,, \\
  &P_\mu \equiv-i\partial_\mu \,, \\
  &D \equiv-ix_\mu\partial^\mu \,, \\
  &K_\mu \equiv i(x^2\partial_\mu-2x_\mu x_\nu\partial^\nu) \,,
\end{align}

На двумерной мировой поверхности удобно ввести комплексные координаты $z,\bar{z}$. 
В этом случае конформная группа --- это набор всех аналитических функций $w(z)$ на плоскости. 

В двумерной конформной теории поля тензор энергии-импульса бесследовый.

Для классической теории алгебра конформных преобразований --- это алгебра Витта, которая порождается
генераторами $\{L_n, n\in \mathbb{Z}\}$ --- модами разложения оператора энергии-импульса $T$, с
коммутационными соотношениями
\begin{equation}
  \label{eq:2}
  [L_m,L_n]=(m-n)L_{m+n}
\end{equation}
При квантовании возникает конформная аномалия, что соответствует центральному расширению алгебры (то
есть появлению центрального заряда $c$). В коммутационные соотношения надо добавить член
$\frac{c}{12}(m^3-m)\delta_{m+n,0}$. В результате получаем алгебру Вирасоро.

Поля теории $\phi(z,\bar z)$ должны преобразовываться определенным образом при конформных преобразованиях.
Оказывается, что все поля группируются в конформные семейства, в которых есть одно примарное поле
\begin{equation}
  \label{eq:3}
  \begin{split}
    \phi_{\Delta,\bar \Delta}(z,\bar z)\underset{
      \genfrac{}{}{0pt}{}{z\to w(z)}
        {\bar z \to \bar w(\bar z)}
    }
    {\longrightarrow} \left(\frac{dw}{dz}\right)^{\Delta}\left(\frac{d\bar w}{d\bar
        z}\right)^{\bar\Delta}\phi_{\Delta,\bar \Delta}(w(z),\bar w(\bar z))\\
    L_n \phi=0,\quad n>0\\
    L_0 \phi=\Delta \phi\\
  \end{split}
\end{equation}
$\Delta, \bar \Delta$ называются конформными размерностями поля.
Все остальные поля называются вторичными и получаются из примарного действием операторов $L_{-n}$:
\begin{equation}
  \label{eq:67}
  L_{-n_1}L_{-n_2}\dots \phi_{\Delta}
\end{equation}
Все поля в теории оказываются суммами мультиплетов алгебры Вирасоро.

При некоторых дополнительных предположениях двумерную конформную теорию можно построить и решить
полностью, если определен набор примарных полей и аномальных размерностей. 

Существуют различные подходы к аксиоматизации двумерной конформной теории поля. В общем случае нам
не нужно знать действие, если мы полный знаем набор примарных полей и операторные разложения их
произведений. 

В том случае, если в теории конечное число примарных полей, такая теория называется минимальной.
Кроме того, теории классифицируются по значениям центрального заряда $c$. Теории с рациональным
центральным зарядом называются рациональными. Оказывается, что все такие теории могут быть получены
факторизацией так называемых моделей Весса-Зумино-Новикова-Виттена.

\subsection{WZW-модели}
\label{sec:wzw}

WZW-модели обладают дополнительной симметрией. Алгебра токов в них --- это алгебра Каца-Муди
(аффинная алгебра Ли $\mathfrak{g}$), а полная киральная алгебра --- полупрямое произведение
$Vir\ltimes \mathfrak{g}$. 

Сейчас я поясню, что все это значит и как строятся такие модели.

Модели Весса-Зумино-Виттена можно строить начиная со следующего действия:
\begin{equation}
\label{eq:4}
  S=S_0+k\Gamma
\end{equation}
где $k$ - целое.
Здесь $S_0$ --- действие нелинейной $\sigma$-модели.
\begin{equation}
  \label{eq:5}
  S_0=\frac{1}{4a^2}\int_{\partial B} d^2x\; Tr (\partial^{\mu}g^{-1}\partial_{\mu}g),
\end{equation}
где $a^2>0$ - положительный параметр, $g(x)\in G$ - поле со значениями в группе $G$, которую мы
будем считать полупростой. 

В нелинейной $\sigma$-модели конформная инвариантность теряется на квантовом уровне.
Голоморфный и антиголоморфный токи не сохраняются по отдельности.
Поэтому мы добавляем член Весса-Зумино $\Gamma$ к действию
\begin{equation}
  \label{eq:73}
\Gamma= - \frac{i }{24\pi} \int_{B}\epsilon_{ijk} Tr\left(
    \tilde g^{-1}\frac{\partial \tilde g}{\partial y^i}
      \tilde g^{-1}\frac{\partial \tilde g}{\partial y^j}
      \tilde g^{-1}\frac{\partial \tilde g}{\partial y^k}\right) d^3y
\end{equation}
Он определен на трехмерном многообразии $B$, ограниченном исходным двумерным пространством.
Через $\tilde{g}$ мы обозначили продолжение поля $g$ на $B$. Такое продолжение не единственно. В
компактифицированном трехмерном пространстве компактное двумерное многообразие разделяет два
трехмерных многообразия. Разность значений члена Весса-Зумино $\Delta\Gamma$ на этих многообразиях
дается правой частью уравнения (\ref{eq:73}) с интегралом, продолженным на все компактное трехмерное
пространство. Так как оно топологически эквивалентно три-сфере, получаем
\begin{equation}
  \label{eq:75} \Delta\Gamma= - \frac{i }{24\pi} \int_{S^3}\epsilon_{ijk} Tr'\left( \tilde
g^{-1}\frac{\partial \tilde g}{\partial y^i} \tilde g^{-1}\frac{\partial \tilde g}{\partial y^j}
\tilde g^{-1}\frac{\partial \tilde g}{\partial y^k}\right) d^3y
\end{equation}
$\Delta\Gamma$ определен по модулю $2\pi i$, поэтому Евклидов функциональный интеграл
с весом $exp(-\Gamma)$ хорошо определен. Значит константа связи, умножаемая на этот член, должна
быть целочисленной.

Уравнение движения для полного действия (\ref{eq:4}):
\begin{equation}
  \label{eq:77}
  \partial^{\mu}(g^{-1}\partial_{\mu}g)+\frac{a^2 ik}{4\pi}\epsilon_{\mu\nu}\partial^{\mu}(g^{-1}\partial^{\nu}g)=0
\end{equation}
В комплексных координатах оно записывается в виде
\begin{equation}
  \label{eq:78}
  (1+\frac{a^2 k}{4\pi})\partial_z(g^{-1}\partial_{\bar z}g)+(1-\frac{a^2 k}{4\pi})\partial_{\bar z}(g^{-1}\partial_z g)=0
\end{equation}
Видно, что при $a^2=\frac{4\pi}{k}$ у нас имеются законы сохранения
\begin{equation}
  \label{eq:79}
  \partial_z(g^{-1}\partial{\bar z}g)=0
\end{equation}
Для токов
\begin{equation}
  \label{eq:72}
  J_z=\partial_z g\;g^{-1}, \qquad J_{\bar{z}}=g^{-1}\partial{\bar z}g
\end{equation}

\begin{equation}
  \label{eq:100}
  \partial_{\bar z}J=0,\quad \partial_z \bar J=0
\end{equation}
То есть голоморфная и антиголоморфная части отщепляются, что является указанием на наличие
конформной инвариантности.

Решение классического уравнения движения имеет вид
\begin{equation}
  \label{eq:80}
  g(z,\bar z)=f(z)\bar f(\bar z)
\end{equation}
при произвольных $f(z)$ и $\bar f (\bar z)$.

Сохранение по отдельности токов $J_z,\; J_{\bar z}$ приводит к инвариантности действия при преобразованиях
\begin{equation}
  \label{eq:81}
   g(z,\bar z)\to \Omega(z)g(z,\bar z)\bar \Omega^{-1}(\bar z)
\end{equation}
где $\Omega,\;\bar \Omega \in G$. То есть мы получили локальную $G(z)\times G(\bar z)$-инвариантность.

Для перехода к квантовому случаю мы переопределяем токи
\begin{equation}
  \label{eq:82}
  J(z)\equiv -k \partial_zg g^{-1}\quad \bar J(\bar z)=k g^{-1}\partial_{\bar z}g
\end{equation}
Тогда вариация действия при инфинитезимальных преобразованиях $\Omega=1+\omega,\; \bar \Omega =1+\bar \omega$ дается выражением
\begin{equation}
  \label{eq:83}
  \delta_{\omega,\bar\omega}S=\frac{i}{4\pi}\oint dz Tr (\omega(z)J(z))-\frac{i}{4\pi}\oint d\bar z Tr(\bar\omega(\bar z)\bar J(\bar z))
\end{equation}
Раскладывая токи
\begin{equation}
  \label{eq:85}
  \begin{aligned}
    J=\sum J^a t^a,\bar J=\sum \bar J^a t^a \\
    \omega=\sum \omega^a t^a\\
  \end{aligned}
\end{equation}
получаем
\begin{equation}
  \label{eq:86}
  \delta_{\omega,\bar \omega}S=-\frac{1}{2\pi i}\oint dz \sum\omega^a J^a+\frac{1}{2\pi i} \oint d\bar z \sum \bar \omega^a \bar J^a
\end{equation}
Мы также получили тождества Уорда $\delta\left< X\right>=\left<(\delta S)X\right>$
\begin{equation}
  \label{eq:87}
  \delta_{\omega,\bar \omega}\left< X \right>=-\frac{1}{2\pi i}\oint dz \sum\omega^a \left< J^a X\right>+
  \frac{1}{2\pi i} \oint d\bar z \sum \bar \omega^a \left< \bar J^a X\right>
\end{equation}
Для токов имеем
\begin{equation}
  \label{eq:88}
  \delta_{\omega}J=[\omega,J]-k\partial_z\omega,\quad \delta_{\omega}J^a=\sum i f_{abc}\omega^b J^c-k\partial_z\omega^a
\end{equation}
Операторное разложение для токов имеет вид
\begin{equation}
  \label{eq:89}
  J^a(z) J^b(w) \sim \frac{k\delta_{ab}}{(z-w)^2}+\sum i f_{abc}\frac{J^c(w)}{(z-w)}
\end{equation}
Раскладывая токи в ряд, получаем
\begin{equation}
  \label{eq:90}
  \begin{aligned}
    J^a(z)=\sum_{n\in \mathbb Z}z^{n-1}J^a_n\\
    \left[J^a_n,J^b_m\right]=\sum_c i f^{abc}J^c_{n+m}+kn\delta^{ab}\delta_{n+m,0}
  \end{aligned}
\end{equation}
Теперь мы видим, что компоненты токов образуют аффинную алгебру Ли $\hat g$.


Тензор энергии-импульса вводится при помощи конструкции Сугавары как сумма нормально упорядоченных компонент токов
\begin{equation}
  \label{eq:102}
  T(z)=\frac{1}{2(k+h^v)}\sum_a N(J^a J^a)(z)
\end{equation}
Здесь $h^v$ - дуальное число Кокстера.

Тензор энергии-импульса можно разложить на моды $L_n$
\begin{equation}
  \label{eq:91}
  L_n=\frac{1}{2(k+h^v)}\sum_a\sum_m:J^a_m J^a_{n-m}:
\end{equation}
Тогда коммутационные соотношения для мод $L_n$ имеют вид
\begin{equation}
  \label{eq:92}
  \begin{aligned}
    \left[L_n,L_m\right]=(n-m)L_{n+m}+\frac{c}{12}(n^3-n)\delta_{n+m,0}\\
    \left[L_n,J^a_m\right]=-mJ^a_{n+m}
  \end{aligned}
\end{equation}

Таким образом, конструкция Сугавары --- это способ вложения алгебры Вирасоро в универсальную обертывающую аффинной алгебры Ли $\hat{g}$

Полная киральная алгебра модели Весса-Зумино-Виттена равна полупрямому произведению $Vir\ltimes \hat g$

Примарными оказываются поля, которые преобразуются ковариантно под действием $G(z)\times G(\bar z)$,
как $g(z,\bar z)$. В терминах операторного разложения это свойство переформулируется следующим
образом:
\begin{equation}
  \label{eq:84}
  \begin{aligned}
    J^a(z)g(w,\bar w)\sim \frac{-t^a g(w,\bar w)}{(z-w)}\\
    \bar J^a(z)g(w,\bar w)\sim \frac{ g(w,\bar w)t^a}{(z-w)}
  \end{aligned}
\end{equation}
Любое поле $\phi_{\lambda,\mu}$, преобразующееся ковариантно по отношению к некоторому
представлению, заданному весом $\lambda$ в голоморфном секторе и весом $\mu$ в антиголоморфном,
является примарным полем WZW-модели.

В модах это свойство записывается в виде
\begin{equation}
  \label{eq:93}
  \begin{aligned}
    & (J_0^a \phi_{\lambda})=-t^a_{\lambda}\phi_{\lambda}\\
    & (J^a_n\phi_{\lambda})=0\quad \mbox{для}\; n>0\\
  \end{aligned}
\end{equation}
Мы можем сопоставить состояние $\left|\phi_{\lambda}\right>$ полю $\phi_{\lambda}$
  \begin{equation}
    \label{eq:94}
    \phi_{\lambda}(0)=\left|\phi_{\lambda}\right>
  \end{equation}
Тогда условия (\ref{eq:93}) для примарных полей дают
\begin{equation}
  \label{eq:95}
  \begin{aligned}
    & J_0^a\left|\phi_{\lambda}\right>=-t^a_{\lambda}\left|\phi_{\lambda}\right>\\
    & J^a_n\left|\phi_{\lambda}\right>=0 \quad \mbox{для}\; n>0 \\
  \end{aligned}
\end{equation}
Легко видеть, что действие генераторов алгебры Вирасоро имеет вид
\begin{equation}
  \label{eq:96}
  L_0\left|\phi_{\lambda}\right>=\frac{1}{2(k+h^v)}\sum_aJ^a_0J^a_0\left|\phi_{\lambda}\right>=\frac{(\lambda,\lambda+2\rho)}{2(k+h^v)}\left|\phi_{\lambda}\right>
\end{equation}
Здесь использовано явное выражение для собственных значений квадратичного оператора Казимира.

Примарные поля живут в интегрируемых конечномерных представлениях, так как бесконечномерные и
неинтегрируемые поля отщепляются в корреляционных функциях.

Все вторичные состояния имеют вид
\begin{equation}
  \label{eq:97}
  J^{a_1}_{-n_1}J^{a_2}_{n_2}\dots\left|\phi_{\lambda}\right>
\end{equation}

\subsection{Конформные вложения и модулярно-инвариантные статсуммы}
\label{sec:modular-invariance}

Если мы изучаем конформную теорию поля на плоскости или на сфере, то мы можем рассматривать
голоморфный и антиголоморфный сектора независимо. Если говорить о WZW-моделях, то примарные поля
принадлежат тензорному произведению неприводимых представлений аффинной алгебры.

Если мы говорим о применении конформной теории для описания поведения струн, то теория должна быть
определена на римановых поверхностях большего рода ($h>0$), чтобы можно было описывать
взаимодействия струн. Считается, что для этого необходимо (и, возможно, достаточно
\cite{gaberdiel2000icf}) чтобы теория была определена на торе.

В теории критического поведения конформная инвариантность имеет место только в критической точке,
где голоморфный и антиголоморфный сектора расцеплены. Но вблизи критической точки эти сектора должны
быть связаны, и так как мы предполагаем плавный переход к критической точке в пространстве
параметров, то эта связь должна сохраняться и в критической точке. Физический спектр теории должен
плавно меняться, когда мы покидаем критическую точку, и связь голоморфного и антиголоморфного
сектора вдали от критической точки должна приводить к ограничениям на набор состояний в критической
точке. Этого можно достичь через геометрию, то есть накладывая граничные условия на состояния
\cite{difrancesco1997cft}. Здесь естественно рассматривать периодические граничные условия, которые
эквивалентны рассмотрению теории на торе.

Если мы наложили периодические граничные условия с периодами $\omega_1, \omega_2,\; \tau=\omega_2/\omega_1$, то статсумма записывается в виде
\begin{equation}
  \label{eq:1}
  Z(\tau)=Tr \exp 2\pi i (\tau (L_0-c/24)-\bar{\tau} (\bar{L}_0-c/24))
\end{equation}
Или, если ввести $q=\exp 2\pi i \tau$
\begin{equation}
  \label{eq:2}
  Z(\tau)=Tr \left (q^{L_0-c/24}\bar{q}^{\bar{L}_0-c/24}\right)
\end{equation}
Двумерный тор представляет собой фактор пространство $\mathbb{R}^2\approx \mathbb{C}$ по отношениям эквивалентности $z\sim z+w_1$ and $z\sim z+w_2$, где $w_1$ и $w_2$ не параллельны. 

Разные параметризации тора связаны модулярными преобразованиями, таким образом возникает требование модулярной инвариантности статсуммы.

Комплексная структура такого тора конформно эквивалентна тору, для которого соотношения эквивалентности записываются в виде $z\sim z+1$ и $z\sim z+\tau$, где $\tau$ в верхней полуплоскости $\mathbb{C}$.

Легко видеть, что $\tau$, $T(\tau)=\tau+1$ и $S(\tau)=-\frac{1}{\tau}$ описывают конформно-эквивалентные торы. Отображения $T$ и $S$ порождают группу  $SL(2,\mathbb{Z})/\mathbb{Z}_2$, состоящую из матриц вида
\begin{equation}
  \label{eq:99}
  A=
  \begin{pmatrix}
    a & b\\
    c & d 
  \end{pmatrix}
  \quad\mbox{где}\; a,b,c,d\in\mathbb{Z},\quad ad-bc=1,
\end{equation}
и матрицы $A$ и $-A$ действуют одинаково на $\tau$
\begin{equation}
  \label{eq:100}
  \tau\to A\tau=\frac{a\tau+b}{c\tau+d}
\end{equation}
 $\tau$ называется модулярным параметром, а группа $SL(2,\mathbb{Z})/\mathbb{Z}_2$ --- модулярной группой.

Конформная теория поля задаётся примарными полями $\Phi_a$ с конформными размерностями $\Delta_a$:
\begin{equation}
  \label{eq:3}
  \begin{split}
    \Phi_{a}(z)\underset{z\to w(z)}{\longrightarrow} \left(\frac{dw}{dz}\right)^{\Delta_a}\Phi_{a}(w(z))\\
    L_n \Phi_a=0,\quad n>0\\
    L_0 \Phi_a=\Delta_a \Phi_a\\
  \end{split}
\end{equation}

Примарные поля живут в пространствах $\mathcal{H}_{(i,j)}$, которые представляют собой тензорные произведения неприводимого представления  $\mathcal{H}_j$ киральной алгебры и неприводимого представления $\bar{\mathcal{H}}_{\bar{j}}$ антикиральной алгебры. Тогда статсуммы на торе (\ref{eq:2}) может быть переписана в виде  
\begin{equation}
  \label{eq:4}
  \sum_{(j,\bar j)}\chi_j(q)\bar \chi_{\bar j}(\bar q)
\end{equation}
где $\chi_j$ --- характер пердставления $\mathcal{H}_j$,
\begin{equation}
  \label{eq:5}
  \chi_j(\tau)=Tr_{\mathcal{H}_j}(q^{L_0-\frac{c}{24}})\quad \mbox{где}\; q=e^{2\pi i \tau}
\end{equation}
Характеры переходят друг в друга при модулярных преобразованиях:
\begin{equation}
  \label{eq:107}
  \chi_j\left(-\frac{1}{\tau}\right)=\sum_k S_{jk}\chi_k(\tau)\quad \mbox{и}\quad \chi_j(\tau+1)=\sum_kT_{jk}\chi_k(\tau),
\end{equation}
где $S$ и $T$ --- постоянные матрицы. Это верно для большого класса конформных теорий поля \cite{gaberdiel2000icf}. 

Для WZW-моделей представления определяются старшими весами $\hat \lambda, \hat \xi$. Тогда
\begin{equation}
  \label{eq:6}
  \mathcal{H}=\bigoplus_{\hat \lambda,\hat \xi\in P^{(k)}_{+}}M_{\hat \lambda,\hat \xi} L_{\hat \lambda}\otimes L_{\hat \xi}
\end{equation}
Статсумма даётся выражением
\begin{equation}
  \label{eq:7}
  Z(\tau)=\sum_{\hat \lambda,\hat \xi\in P^{(k)}_{+}} \chi_{\hat \lambda}(\tau)M_{\hat \lambda\hat\xi}\bar \chi_{\hat \xi}(\bar \tau)
\end{equation}
Элементы так называемой матрицы масс $M_{\hat \lambda\hat\xi}$ можно рассматривать как кратности примарных полей с весами $\hat\lambda,\hat \xi$. У них есть следующие свойства: $M_{\hat \lambda\hat\xi}\in \mathbb{Z}_+$, модулярная инвариантность
\begin{equation}
  \label{eq:8}
  \begin{aligned}
    T^{\dagger}MT=S^{\dagger}MS=M,\\
    [M,S]=[M,T]=0,
  \end{aligned}
\end{equation}
и $M_{00}=1$ для единственности вакуума.

Простейший случай диагональной матрицы $M$ соответствует равным голоморфным и антиголоморфным конформным размерностям. Есть несколько способов построения недиагональных модулярных инвариантов из диагональных \cite{difrancesco1997cft}:
\begin{itemize}
\item Метод внешних автоморфизмов
\item Конформное вложение в большую теорию
\item Перестановки Галуа
\end{itemize}
Мы будем обсуждать только конформные вложения.

\section{Конформные вложения}
\label{sec:conformal-embeddings}

Состояния в теории соответствующей алгебре $g$ (с киральной алгеброй $Vir\ltimes \hat g$) имеют вид
\begin{equation}
  \label{eq:9}
  J^{a_1}_{-n_1}J^{a_2}_{-n_2}\dots\left|\lambda\right>\quad n_1\geq n_2\geq\dots>0
\end{equation}
А для подалгебры $\mathfrak{p}\subset\mathfrak{g}$
\begin{equation}
  \label{eq:10}
  \tilde{J}^{a'_1}_{-n_1}\tilde{J}^{a'_2}_{-n_2}\dots\left|\mathcal{P}\lambda\right>
\end{equation}
Здесь $\tilde{J}^{a'_j}_{-n_j}$ --- генераторы $\mathfrak{p}$, а $\mathcal{P}$ --- проекция $\mathfrak{g}$ на $\mathfrak{p}$. Очевидно, что $\mathfrak{g}$-инвариантность вакуума ведёт к его $\mathfrak{p}$-инвариантности, но нет оснований считать, что проекция сохраняет конформную инвариантность. Это легко увидеть из рассмотрения тензора энергии - импульса. Действительно, в тензоре энергии - импульса в виде Сугавары можно выделить часть, составленную из тех комбинаций генераторов $g$, которые входят в $p$. Но есть ещё и остаток. Из-за него действие генераторов Вирасоро на состояния (\ref{eq:10}) будет выводить из этого набора.
Таким образом в ситуации общего положения конформная инвариантность нарушается.

Однако есть исключения. Так для представлений уровня 1 simply-laced алгебр тензор энергии - импульса в форме Сугавары состоит только из генераторов Картана. В этом случае он может быть пере-выражен через генераторы подалгебры и конформная инвариантность сохраняется. Тогда $T_{\hat{\mathfrak{g}}_k}=T_{\hat{\mathfrak{p}}_{\tilde k}}\Rightarrow c(\hat{\mathfrak{g}}_k)=c(\hat{\mathfrak{p}}_{\tilde k})$. Это можно переписать в виде равенства
\begin{equation}
  \label{eq:11}
  \frac{k\;dim\mathfrak{g}}{k+g}=\frac{x_e k\; dim\mathfrak{p}}{x_ek+p}
\end{equation}
Где $x_e$ - индекс вложения, а $g$, $p$ - дуальные числа Кокстера соответствующих алгебр и было использовано равенство $\tilde k=x_e k$.
\begin{equation}
  \label{eq:12}
  x_e=\frac{\left|\mathcal{P}\Theta_g\right|^2}{\left| \Theta_p\right|^2},\quad
  x_e=\sum_{\mu\in P_{+}}b_{\lambda\mu}\frac{x_{\mu}}{x_{\lambda}}
\end{equation}
(Здесь $x_{\lambda}$ - индекс представления со старшим весом $\lambda$:
\begin{equation}
  \label{eq:13}
  x_{\lambda}=\frac{dim \left|\lambda\right|(\lambda,\lambda+\rho)}{2 dim g}
\end{equation})

Вложения, которые удовлетворяют условию (\ref{eq:11}), называются конформными.

Нетрудно показать, что решения уравнения (\ref{eq:11}) существуют только для уровня $k=1$.

(Существует конечное число конформных вложений, они классифицированы).

{\bf Примеры}
\begin{itemize}
\item $su(2)\subset su(3),\; x_e=4$
\item $\hat{su}(2)_{10}\subset\hat{sp}(4)_1$
\item $\hat{su}(2)_{28}\subset(\hat{G_2})_1$
\item $\hat{su}(2)_{16}\oplus\hat{su}(3)\subset (\hat{E_8})_1$
\end{itemize}

\section{Конформные коэффициенты ветвления}
\label{sec:conformal-branching-rules}

Для ветвления аффинных алгебр $\hat{\lambda}\to \bigoplus_{\hat{\mu}}b_{\hat{\lambda}\hat{\mu}}\hat{\mu}$ существуют разные алгоритмы, однако мы можем существенно сократить работу при рассмотрении конформных вложений.

Во-первых, заметим, что если коэффициент $b_{\hat\lambda\hat\mu}$ отличен от нуля, то конечная часть старшего веса $\mu$ модуля $L_{\hat\mu}$ находится в некотором грейде $n$  бесконечномерного модуля $L_{\hat\lambda}$ уровня 1. Сохранение конформной инвариантности ведет к соотношению для конформных размерностей соответствующих полей
\begin{equation}
  \label{eq:14}
  \Delta_{\hat\lambda}+n=\Delta_{\hat\mu},
\end{equation}
которое переписывается в виде

\begin{equation}
  \label{eq:15}
  \frac{(\lambda,\lambda+2\rho)}{2(1+g)}+n=\frac{(\mu,\mu+2\rho)}{2(x_e+p)}
\end{equation}

Используя этот факт можно  легко вычислять правила ветвления. Для этого надо вычислить размерности интегрируемых представлений обеих алгебр  $\mathfrak{g},\mathfrak{p}$ и найти все тройки  $(\lambda,\mu,n)$, удовлетворяющие соотношению (\ref{eq:15}). Затем рассматриваем разложения представления $L_{\hat\lambda}$ в грейде $n$ в сумму неприводимых представлений конечномерной алгебры $\mathfrak{g}$ и выписываем правила ветвления этих представлений на представления подалгебры $\mathfrak{p}$. Коэффициент ветвления $b_{\hat{\lambda},\hat{\mu}}$ --- это сколько раз представление со старшим весом $\mu$ появляется в этом списке.

{\bf Пример}

$\hat{su}(2)_4\subset \hat{su}(3)_1$.
Список конформных размерностей:
\begin{equation}
  \label{eq:16}
  \begin{aligned}
    \hat{su}(2)_4&: h_{[4,0]}=0, h_{[3,1]}=\frac{1}{8}, h_{[2,2]}=\frac{1}{3}, h_{[1,3]}=\frac{5}{8}, h_{[0,4]}=1\\
    \hat{su}(3)_1&: h_{[1,0,0]}=0, h_{[0,1,0]}=h_{[0,0,1]}=\frac{1}{3}    
  \end{aligned}
\end{equation}
Отсюда видно, что
\begin{equation}
  \label{eq:17}
  \begin{aligned}
    \left[1,0,0\right] &\to c_1 [4,0]_0\oplus c_2 [0,4]_1\\
    [0,1,0] &\to c_3 [2,2]_0\\
    [0,0,1] &\to c_4 [2,2]_0
  \end{aligned}
\end{equation}
Где $c_1, c_2, c_3, c_4$ - коэффициенты, которые надо найти, а нижний индекс указывает грейд $n$.

Коэффициенты $c_1, c_3, c_4$ вычисляются из правил ветвления в нулевом грейде. В этом грейде $L_{\hat\lambda}$ содержит только $L_{\lambda}$. Из правил ветвления
\begin{equation}
  \label{eq:18}
  (0,0)\to (0),\quad (1,0)\to (2),\quad (0,1)\to (2)
\end{equation}
мы получаем, что $c_1=c_3=c_4=1$. $c_2$ вычисляется из представления грейда 1, содержащего конечномерное представление $(1,1)$ с правилом ветвления
\begin{equation}
  \label{eq:19}
  (1,1)\to (4)\oplus (2)
\end{equation}
То есть $c_2$ тоже равен 1.

После нахождения коэффициентов ветвления, недиагональные модулярные инварианты строятся путем подстановки соотношений для характеров. Например, зная коэффициенты ветвления для вложения $\hat{su}(2)_{28}\subset (\hat{G_2})_1$
  \begin{equation}
    \label{eq:20}
    \begin{aligned}
      & [1,0,0]\to [28,0]\oplus [18,10]\oplus [10,18]\oplus [0,28]\\
      & [0,0,1]\to [22,6]\oplus [16,12]\oplus [12,16]\oplus [6,22]\\
    \end{aligned}
  \end{equation}
мы получаем следующую модулярно-инвариантную статсумму:
\begin{equation}
  \label{eq:21}
  Z=\left|\chi_{[28,0]}+\chi_{[18,10]}+\chi_{[10,18]}+\chi_{[0,28]}\right|^2+\left|\chi_{[22,6]}+\chi_{[16,12]}+\chi_{[12,16]}+\chi_{[6,22]}\right|^2
\end{equation}

\section{Рекуррентные соотношения для коэффициентов ветвления}
\label{sec:branching}

\subsection{Примеры}
\label{sec:examples}

\section{Заключение. Обсуждение перспектив.}
\label{sec:conlusion}


\end{document}


