\documentclass[a4paper,12pt]{article}
\usepackage{ucs}
\usepackage[unicode,verbose]{hyperref}
\usepackage{amsmath,amssymb,amsthm}
\usepackage{pb-diagram}
\usepackage{multicol}
%\usepackage[utf8x]{inputenc}
%\usepackage[russian]{babel}
\usepackage{cmap}
\usepackage{color}
\usepackage{graphicx}
\pagestyle{plain}

%\usepackage{verbatim} 
\newenvironment{comment}
{\par\noindent{\bf Comment}\\}
{\\\hfill$\scriptstyle\blacksquare$\par}

\newtheorem{statement}{Statement}
\theoremstyle{definition} \newtheorem{Def}{Definition}
\newcommand{\go}{\overset{\circ }{\frak{g}}}
\newcommand{\ao}{\overset{\circ }{\frak{a}}}
\newcommand{\co}[1]{\overset{\circ }{#1}}

\begin{document}

\title{Modular properties of string, branching and dual functions}

\author{Vladimir Lyakhovsky \thanks{ Supported by
 RFFI grant N 09-01-00504 and the National Project RNP.2.1.1./1575 }\\
Theoretical Department, SPb State University,\\
198904, Sankt-Petersburg, Russia \\
e-mail:lyakh1507@nm.ru \\
[5mm] Anton Nazarov \thanks{ Supported by
the National Project RNP.2.1.1./1575 }\\
Theoretical Department, SPb State University,\\
198904, Sankt-Petersburg, Russia \\
e-mail:antonnaz@gmail.com
}
\maketitle

\begin{abstract}
  We review the appearance of modular transformation in the representation theory of affine Lie algebras, especially the modular properties of branching and string function.
\end{abstract}
\begin{Def}
  {\bf Modular transformation}
  \begin{equation}
    \label{eq:1}
    A=
    \begin{pmatrix} a & b\\ c & d
    \end{pmatrix} \quad\mbox{where}\; a,b,c,d\in\mathbb{Z},\quad ad-bc=1,
  \end{equation}
  \begin{equation}
    \label{eq:2}
    z\to A z=\frac{az+b}{cz+d}
  \end{equation}
\end{Def}
\begin{Def}
  {\bf Modular form} of weight $k$ is the analytic function $F$ on the upper half plane with infinity ( $F(i\infty)$ is holomorphic) which transforms in the following way under the modular transformations
  \begin{equation}
    \label{eq:3}
    F(Az)=(cz+d)^k F(z)
  \end{equation}
\end{Def}

This definition can be reformulated as the requirement for the modular form to be the function of lattices. 
Consider the general lattice on the complex plane. It consists of the points $(pv,qw)$ where $p,q\in \mathbb{Z},\; w,v\in \mathbb{C}$. This lattice can be generated by $(z,1), \; z=v/w$ or by $(az+b,cz+d), \; a,b,c,d\in\mathbb{Z},\; ad-bc=1$. It is denoted by $\Lambda_z$.
\begin{equation}
  \label{eq:4}
  \frac{1}{cz+d}\Lambda_z=\Lambda_{Az}
\end{equation}
Then we can say that the modular form of weight $k$ is the function $f$ of lattices such that $f(\alpha\Lambda)=\alpha^{-k}f(\Lambda)$.

For $k\geq 4$ we have the following example (Eisenstein series):
\begin{equation}
  \label{eq:5}
  G_k(\Lambda)=\sum_{\lambda\in \Lambda, \lambda\neq 0} \lambda^{-k}
\end{equation}
Vector space of weight $k$ modular forms is finite dimensional.

The quotient of the complex numbers by a lattice is an elliptic curve and every elliptic curve arises in this way. ($y^2=f(x)$, $f$ - cubic polynomial).

For $N\in \mathbb{N},$ define
\begin{equation}
  \label{eq:12}
  \Gamma(N)=\left\{
      \begin{pmatrix} a & b \\ c & d
      \end{pmatrix}
      \right|\left.
        \begin{matrix}
          a,b,c,d\in \mathbb{Z},\quad ad-bc=1\\
          a\equiv d\equiv 1\; \text{mod}\; N,\quad b\equiv c\equiv 0\; \text{mod}\; N
        \end{matrix}
      \right\}
  \end{equation}
The subgroup $\Gamma(N)$ has {\it finite index} (define!) in $\Gamma(1)=SL(2,\mathbb{Z})$. For any $\tau\in \mathbb{C}$ there exists element of $SL(2,\mathbb{Z})$, which transforms it to the fundamental domain.

For $A\in SL(2,\mathbb{Z})$ define
\begin{equation}
  \label{eq:13}
  j(A,\tau)=c\tau+d.
\end{equation}
Note that
\begin{equation}
  \label{eq:14}
  j(BA,\tau)=j(B,A\tau)j(A,\tau).
\end{equation}
For any subgroup $\Gamma$ of finite index define a {\bf multiplier system} $\chi$ as a mapping
\begin{equation}
  \label{eq:15}
  \chi:\Gamma\to U(1).
\end{equation}
Let $k\in \mathbb{R}$. A complex valued holomorphic function $f$ on the upper half-plane of the complex plane is a modular form of weight $k$ with respect to $\Gamma$ if there exists a multiplier system $\chi$ for $\Gamma$ with
\begin{equation}
  \label{eq:16}
  f(\tau)=\chi(A) j(A,\tau)^{-k} f(A\tau),
\end{equation}
for all $\tau$ in the upper half-plane and $A\in \Gamma$.


For the $d$-dimensional lattice $L$ with the non-degenerate form $(\cdot,\cdot)$ we can define theta-functions
\begin{equation}
  \label{eq:8}
  \Theta_L (\tau)=\sum_{x\in L}e^{\pi i \tau (x,x)}
\end{equation}
$\Theta_L$ is modular of weight $d/2$.

\begin{comment}
  How can we link the $d$-dimensional lattice $L$ with the lattice $\Lambda_{\tau}$?
\end{comment}

Now we can consider the weight lattice of the affine Lie algebra $\mathfrak{g}$.
Cartan subalgebra:
\begin{equation}
  \label{eq:6}
  \mathfrak{h}=\mathbb{C}\Lambda_0\oplus \co{\mathfrak{h}}\oplus \mathbb{C}\delta
\end{equation}
$h\in \mathfrak{h}$
\begin{equation}
  \label{eq:7}
  h=(z;\tau;t)=2\pi i (-\tau \Lambda_0 +z + t\delta)\quad (\tau,t\in \mathbb{Z}, z\in \co{\mathfrak{h}})
\end{equation}

It is known that string functions
\begin{eqnarray}
  \label{eq:9}
  \sigma^{(\mu)}_{\lambda}=\sum_n mult^{(\mu)}(\lambda-n\delta)e^{\lambda-n\delta}\\
  \sigma^{(\mu)}_{\lambda}(\tau)=\sigma^{(\mu)}_{\lambda}|(0;\tau;0)=\sum_n mult^{(\mu)}(\lambda-n\delta)e^{2\pi i n\tau}
\end{eqnarray}
can be rescaled to become modular forms:
\begin{equation}
  \label{eq:10}
  c^{(\mu)}_{\lambda}=e^{2\pi i s(\mu,\lambda)}\sum_n mult^{(\mu)}(\lambda-n\delta)e^{2\pi i n\tau}
\end{equation}
where
\begin{equation}
  \label{eq:11}
  s(\mu,\lambda)=\frac{(\mu+\rho,\mu+\rho)}{2(k+g)}-\frac{(\rho,\rho)}{2g}-\frac{(\lambda,\lambda)}{2k}.
\end{equation}
(Here $k$ - level of the representation, $g$ - dual Coxeter number).

See Kac, Peterson 1984 for the original proof.
\section{Questions}
\label{sec:questions}
\begin{itemize}
\item The modular properties of WZW models are naturally connected with the problem of theory definition on the torus. What can we say about the pure representation theory? What is the analogue of the torus?
\item What kind of constrains on the stars and fans follow from the modular invariance?
\item The modular invariance is important in the number theory. What problems of number theory are connected through the modular transformations with the representation theory?
\item I need to write about the Hecke algebras
\item How are Hecke algebras connected to the modular forms?
\item Are there any kind of modular invariance analogues for the finite-dimensional Lie algebras? There exists the lattice.
\item What become of modular invariance during the quantization? Are there any properties of quantum integrable systems that follow from the modular invariance?
\end{itemize}

\bibliography{article}{}
\bibliographystyle{utphys}

\end{document}
