\documentclass[a4paper,12pt]{article}
\usepackage{ucs}
\usepackage[unicode,verbose]{hyperref}
\usepackage{amsmath,amssymb,amsthm}
\usepackage{pb-diagram}
\usepackage{multicol}
%\usepackage[utf8x]{inputenc}
%\usepackage[russian]{babel}
\usepackage{cmap}
\usepackage{color}
\usepackage{graphicx}
\usepackage{epstopdf}
\pagestyle{plain}

%\usepackage{verbatim} 
\newenvironment{comment}
{\par\noindent{\bf TODO}\\}
{\\\hfill$\scriptstyle\blacksquare$\par}

\newtheorem{statement}{Statement}
\newtheorem{lemma}{Lemma}
\newtheorem{theorem}{Theorem}
\theoremstyle{definition}
\newtheorem{corollary}{Corollary}[theorem]
\theoremstyle{definition}
\newtheorem{mynote}{Note}[section]
\theoremstyle{definition}
\newtheorem{definition}{Definition}
\newcommand{\go}{\stackrel{\circ }{\mathfrak{g}}}
\newcommand{\ao}{\stackrel{\circ }{\mathfrak{a}}}
\newcommand{\co}[1]{\stackrel{\circ }{#1}}
\newcommand{\pia}{\pi_{\mathfrak{a}}}
\newcommand{\piab}{\pi_{\mathfrak{a}_{\bot}}}
\newcommand{\af}{\mathfrak{a}}
\newcommand{\afb}{\mathfrak{a}_{\bot}}
\begin{document}
\title{\textbf{{\Large {System of dual functions in the reduction of affine Lie algebra representations}}}}
\author{Vladimir Lyakhovsky \thanks{ Supported by
 RFFI grant N 09-01-00504 and the National Project RNP.2.1.1./1575 }\\
Theoretical Department, SPb State University,\\
198904, Sankt-Petersburg, Russia \\
e-mail:lyakh1507@nm.ru \\
[5mm] Anton Nazarov \thanks{ Supported by
the National Project RNP.2.1.1./1575 }\\
Theoretical Department, SPb State University,\\
198904, Sankt-Petersburg, Russia \\
e-mail:antonnaz@gmail.com
}
\maketitle

\begin{abstract}
  Using the recurrent relation for the series expansion of the affine Lie algebra branching functions we construct the dual set of the functions, which can be completely described using Weyl symmetry of affine Lie algebra. Explicit construction of the dual functions can be carried out using the geometric finite automata approach. The task of the explicit construction of the branching functions is then rewritten as the procedure of the inversion of matrix of dual functions. Finally we study the properties of the dual functions under modular and conformal transformations and discuss possible applications in two-dimensional conformal field theory.
\end{abstract}

\section{Introduction}
\label{sec:introduction}

We consider the following problems.
\begin{itemize}
\item Calculation of the weight multiplicities in the (irreducible highest-weight) module of (simple) finite-dimensional Lie algebra.
\item Calculation of the branching coefficients for the decomposition of the module of finite-dimensional algebra to the modules of the sub-algebra.
\item Calculation of the weight multiplicities in the (irreducible highest-weight) module of (affine) Kac-Moody algebra.
\item Calculation of the branching coefficients for the decomposition of the module of affine Lie algebra to the modules of the affine sub-algebra.
\end{itemize}

All these problems are solved using the recurrent relations, such as Freudenthal or Racah formula. 
The recurrent relations can be written in matrix form if one introduces the ordering on the weight lattice (or fundamental domain of it).

If we consider these problems not for one selected module but for all possible modules, we get the infinite multi-diagonal matrix. It is similar to the Jacobi matrices, which give rise to the systems of orthogonal functions.

Studying the recurrent relations in matrix form in more details we can find out that there exists two tightly connected systems of the orthogonal functions. These systems are dual in the sense that they correspond to the mutually inverse (infinite) matrices. 

In this paper we describe the construction of these systems of functions and study their properties. 

The paper is organised as follows. In section \ref{sec:finite-dimens-lie} we describe the recurrent process that gives rise to the systems of the orthogonal polynomials. This construction in some sense is not new (e.g. see \cite{2010arXiv1001}), but we explicitly describe the dual system of functions.  The section \ref{sec:affine-lie-algebras} is devoted to the systems of special functions connected with the affine Lie algebras. We establish the modular properties of the dual functions. Possible applications in physics are described in the next section \ref{sec:poss-appl}. We discuss the importance of the tensor product decompositions for the integrable systems which follows from the Hecke duality. Also we consider conformal field theory.
In the conclusion \ref{sec:conclusion} we overview the results and state some open questions.

\subsection{Notation}
\label{sec:notation}
We use the following notation in this paper.

Consider affine Lie algebras $\frak{g}$ and $\frak{a}$ with the
underlying finite-dimensional subalgebras $\go$ and $%
\ao$ and an injection $\frak{a}\longrightarrow \frak{g%
}$ such that $\frak{a}$ is a reductive subalgebra $\frak{a\subset g}$ with
correlated root spaces: $\frak{h}_{\frak{a}}^{\ast }\subset \frak{h}_{\frak{g%
}}^{\ast }$ and $\frak{h}_{\ao}^{\ast }\subset \frak{h%
}_{\go}^{\ast }$\ .

We use the following notations:

$L^{\mu }$\ $\left( L_{\frak{a}}^{\nu }\right) $\ --- the integrable module
of $\frak{g}$ with the highest weight $\mu $\ ; (resp. integrable $\frak{a}$
-module with the highest weight $\nu $ );

$r$ , $\left( r_{\frak{a}}\right) $ --- the rank of the algebra $\frak{g}$ $%
\left( \mbox{resp. }\frak{a}\right) $ ;

$\Delta $ $\left( \Delta _{\frak{a}}\right) $--- the root system; $\Delta
^{+} $ $\left( \mbox{resp. }\Delta _{\frak{a}}^{+}\right) $--- the positive
root system (of $\frak{g}$ and $\frak{a}$ respectively);

$\mathrm{mult}\left( \alpha \right) $ $\left( \mathrm{mult}_{\frak{a}}\left(
\alpha \right) \right) $ --- the multiplicity of the root $\alpha$ in $\Delta
$ (resp. in $\left( \Delta _{\frak{a}}\right) $);

$\co{\Delta}$ , $\left( \co{\Delta _{\frak{a}}}%
\right)$ --- the finite root system of the subalgebra $\co{%
\frak{g}}$ (resp. $\co{\frak{a}}$);
$\Theta$, $(\Theta_{\af})$ --- the highest root of the algebra $\mathfrak{g}$ (resp. subalgebra $\af$);

$\mathcal{N}^{\mu }$ , $\left( \mathcal{N}_{\frak{a}}^{\nu }\right) $ --- the
weight diagram of $L^{\mu }$ $\left( \mbox{resp. }L_{\frak{a}}^{\nu }\right)
$ ;

$W$ , $\left( W_{\frak{a}}\right) $--- the corresponding Weyl group;

$C$ , $\left( C_{\frak{a}}\right) $--- the fundamental Weyl chamber;

$\bar{C}, \left(\bar{C_{\af}}\right)$ --- the closure of the fundamental Weyl chamber;

$\rho $\ , $\left( \rho _{\frak{a}}\right) $\ --- the Weyl vector;

$\epsilon \left( w\right) :=\det \left( w\right) $ ;

$\alpha _{i}$ , $\left( \alpha _{\left( \frak{a}\right) j}\right) $ --- the $i
$-th (resp. $j$-th) basic root for $\frak{g}$ $\left( \mbox{resp. }\frak{a}%
\right) $; $i=0,\ldots ,r$,\ \ $\left( j=0,\ldots ,r_{\frak{a}}\right) $;

$\delta $ --- the imaginary root of $\frak{g}$ (and of $\frak{a}$ if any);

$\alpha _{i}^{\vee }$ , $\left( \alpha _{\left( \frak{a}\right) j}^{\vee
}\right) $--- the basic coroot for $\frak{g}$ $\left( \mbox{resp. }\frak{a}%
\right) $ , $i=0,\ldots ,r$ ;\ \ $\left( j=0,\ldots ,r_{\frak{a}}\right) $;

$\co{\xi }$ , $\co{\xi _{\left( \frak{a}\right) }}$
--- the finite (classical) part of the weight $\xi \in P$ , $\left( \mbox{%
resp. }\xi _{\left( \frak{a}\right) }\in P_{\frak{a}}\right) $;

$\lambda =\left( \co{\lambda };k;n\right) $ --- the
decomposition of an affine weight indicating the finite part $\co{\lambda }$, level $k$ and grade $n$;

$P$ $\left( \mbox{resp. } P_{\frak{a}}\right) $ \ --- the weight lattice;

$M \left( \mbox{resp. }M_{\frak{a}}\right) :=$

\noindent $=\left\{
\begin{array}{c}
\sum_{i=1}^{r}\mathbf{Z}\alpha _{i}^{\vee }\mbox{ }\left( \mbox{resp. }%
\sum_{i=1}^{r}\mathbf{Z}\alpha _{\left( \frak{a}\right) i}^{\vee }\right)
\mbox{for untwisted algebras or }A_{2r}^{\left( 2\right) }, \\
\sum_{i=1}^{r}\mathbf{Z}\alpha _{i}\mbox{ }\left( \mbox{resp. }\sum_{i=1}^{r}%
\mathbf{Z}\alpha _{\left( \frak{a}\right) i}\right) \mbox{for }A_{r}^{\left(
u\geq 2\right) }\mbox{ and }A\neq A_{2r}^{\left( 2\right) },
\end{array}
\right\} ;$\\
$\Psi ^{\left( \mu \right) }:=\sum\limits_{w\in W}\epsilon (w)e^{w\circ (\mu +\rho )-\rho }$ --- the singular weight element for the $\frak{g}$-module $L^{\mu }$;
$\Psi _{\left( \frak{a}\right) }^{\left( \nu \right) }:=\sum\limits_{w\in W_{\frak{a}}}\epsilon (w)e^{w\circ (\nu +\rho
_{_{\frak{a}}})-\rho _{_{\frak{a}}}}$ --- the corresponding singular weight
element for the $\frak{a}$-module $L_{\frak{a}}^{\nu }$;

$\widehat{\Psi ^{\left( \mu \right) }}$ $\left( \widehat{\Psi _{\left( \frak{%
a}\right) }^{\left( \nu \right) }}\right) $ --- the set of singular weights $%
\xi \in P$ $\left( \mbox{resp. }\in P_{\frak{a}}\right) $ for the module $%
L^{\mu }$ $\left( \mbox{resp. }L_{\frak{a}}^{\nu }\right) $ with the
coordinates $\left( \co{\xi },k,n,\epsilon \left( w\left( \xi
\right) \right) \right) \mid _{\xi =w\left( \xi \right) \circ (\mu +\rho
)-\rho },$ (resp. $\left( \co{\xi },k,n,\epsilon \left(
w_{a}\left( \xi \right) \right) \right) \mid _{\xi =w_{a}\left( \xi \right)
\circ (\nu +\rho _{a})-\rho _{a}}$ );

$m_{\xi }^{\left( \mu \right) }$ , $\left( m_{\xi }^{\left( \nu \right)
}\right) $ --- the multiplicity of the weight $\xi \in P$ \ $\left( \mbox{%
resp. }\in P_{\frak{a}}\right) $ in the module $L^{\mu }$ , (resp. $\xi \in
L_{\frak{a}}^{\nu } $);

$ch\left( L^{\mu }\right) $ $\left( \mbox{resp. }ch\left( L_{\frak{a}}^{\nu
}\right) \right) $--- the formal character of $L^{\mu }$ $\left( \mbox{resp. }%
L_{\frak{a}}^{\nu }\right) $;

$ch\left( L^{\mu }\right) =\frac{\sum_{w\in W}\epsilon (w)e^{w\circ (\mu
+\rho )-\rho }}{\prod_{\alpha \in \Delta ^{+}}\left( 1-e^{-\alpha }\right) ^{%
\mathrm{{mult}\left( \alpha \right) }}}=\frac{\Psi ^{\left( \mu \right) }}{%
\Psi ^{\left( 0\right) }}$ --- the Weyl-Kac formula;

$R:=\prod_{\alpha \in \Delta ^{+}}\left( 1-e^{-\alpha }\right) ^{\mathrm{{%
mult}\left( \alpha \right) }}=\Psi ^{\left( 0\right) }\quad $;

\noindent $\left( \mbox{resp. }R_{\frak{a}}:=\prod_{\alpha \in \Delta _{%
\frak{a}}^{+}}\left( 1-e^{-\alpha }\right) ^{\mathrm{mult}_{\frak{a}}\mathrm{%
\left( \alpha \right) }}=\Psi _{\frak{a}}^{\left( 0\right) }\right) $--- the
denominator;

$  L_{\frak{g}\downarrow \frak{a}}^{\mu }=\bigoplus
\limits_{\nu \in P_{\frak{a}}^{+}}b_{\nu }^{\left( \mu \right) }L_{\frak{a}}^{\nu }$
 --- the module decomposition with respect to $\frak{a}\longrightarrow \frak{g}$;


$b^{(\mu)}_{\nu}$ --- the branching coefficients,

\begin{equation}
  \label{eq:21}
  \sum_{\nu \in \bar{C_{\af}}}b_{\nu }^{\left( \mu \right) }\Psi _{\left( \frak{%
        a}\right) }^{\left( \nu \right) }=\sum_{\lambda \in P_{\frak{a}}}k_{\lambda
  }^{\left( \mu \right) }e^{\lambda };
\end{equation}

 $k_{\lambda}$ --- the anomalous branching coefficients, notice that
\begin{equation*}
%  \label{eq:20}
  b^{(\mu)}_{\nu}=k^{(\mu)}_{\nu} \; \mbox{for} \; \nu\in \bar{C}_{\af};
\end{equation*}

$x_e=\frac{\left|\pia \Theta\right|^2}{\left|\Theta_{\af}\right|^2}$ --- the embedding index.


\section{Finite-dimensional Lie algebras}
\label{sec:finite-dimens-lie}
Consider the irreducible highest-weight module $L^{(\mu)}_{\mathfrak{g}}$ of the simple Lie algebra $\mathfrak{g}$. 
The recurrent relation for the weight multiplicities can be written from the Weyl character formula \cite{lyakhovsky1996rra}:
\begin{equation}
  \label{eq:1}
  \mathrm{ch} L^{(\mu)}_{\mathfrak{g}}=\frac{\sum_{w\in W}\epsilon(w) e^{w(\mu+\rho)-\rho}}{\sum_{w\in W}\epsilon(w) e^{w \rho -\rho}}
\end{equation}
Let us limit our consideration to the fundamental domain (main Weyl chamber). The left side of this equation can be written as
\begin{equation}
  \label{eq:2}
   \mathrm{ch} L^{(\mu)}_{\mathfrak{g}}=\sum_{\lambda\in \bar{C}_{\mathfrak{g}}} m^{(\mu)}_{\lambda} e^{\lambda}
\end{equation}
Multiplying the equation \eqref{eq:1} by the denominator  and collecting the common terms we get the recurrent relation for the weight multiplicities $m^{(\mu)}_{\lambda}$:
\begin{equation}
  \label{eq:3}
  \sum_{w\in W} \epsilon(w) m^{(\mu)}_{\lambda-w\rho+\rho}=\sum_{w\in W}\epsilon(w) \delta_{\lambda,w(\mu+\rho)-\rho}
\end{equation}
If we consider the weights inside the main Weyl chamber the sum on the right-hand side is reduced to the single term $\delta_{\lambda,\mu}$. For the weight $\lambda-w\rho+\rho\not\in \bar{C}_{\mathfrak{g}}$ there exists weight $\xi\in \bar{C}_{\mathfrak{g}}$ and Weyl group element $v\in W$ such that $v\xi=\lambda-w\rho+\rho$, so we can limit the computation of weight multiplicities by the main Weyl chamber. This procedure was called ``folding of the fan'' in \cite{il2010folded}.
Now we want to rewrite the equation \eqref{eq:3} in matrix form similarly to \cite{2010arXiv1001}.
We need to choose the ordering on the set of weights in the closure of fundamental domain $\bar{C}_{\mathfrak{g}}$. We use the following natural procedure. Consider some ordering of the simple roots (e.g. the numbering of nodes of Dynkin diagram from left to right). Let's denote the weights by the Dynkin labels. Then the task is to enumerate the set $\{0\}\cup Z_{+}^r$. $[0,\dots,0]$ get the index 1, $[0,\dots,0,1]$ the index 2, $[0,\dots,0,1,0]$ - 3 etc.
\section{Affine Lie algebras}
\label{sec:affine-lie-algebras}
Consider the (affine) Lie algebra $\mathfrak{g}$  and its reductive subalgebra $\af\subset \mathfrak{g}$. Introduce the following objects.
\begin{definition}
  Denote by $\Delta^{+}_{\bot}=\left\{\alpha\in \Delta^{+}_{\mathfrak{g}}:\forall \beta\in\Delta_{\af} \; \alpha\bot \beta\right\}$ the subset of the positive roots of $\mathfrak{g}$ orthogonal to the root system of $\af$.
\end{definition}

\begin{definition}
  Denote by $W_{\bot}$ the subgroup of the Weyl group $W$ generated by the reflections $\omega_{\alpha}$ corresponding to the roots $\alpha\in \Delta^{+}_{\bot}$.
\end{definition}
\begin{mynote}
  The subgroup $W_{\bot}$ preserves $\Delta_{\af}$ since it is generated by the reflections in hyperplanes containing $\Delta_{\af}$.
\end{mynote}
\begin{definition}
  The subsystem $\Delta_{\bot}=\Delta^{+}_{\bot}\cup \left(-\delta^{+}_{\bot}\right)$ determines the subalgebra $\af_{\bot}\subset \mathfrak{g}$.
\end{definition}

Let us state the following lemma.
\begin{lemma}
Denote by $\mathcal{D}=\rho_{\af_{\bot}}-\pi_{\mathfrak{a}_{\bot}}\rho$ the difference between the Weyl vector of orthogonal subalgebra $\af_{\bot}$ and the projection of the Weyl vector of the algebra $\mathfrak{g}$. Then
the quotient of the singular weights element $\Psi^{(\mu)}$ by the denominator $R_{\af_{\bot}}$ is given by the formal element corresponding to the modified set of the singular weights of $L_{\af\oplus\mathfrak{h}_d}^{(\pi_{\af\oplus\mathfrak{h}_d}\mu)}$ with the multiplicities given by the dimensions of $\af_{\bot}$-modules.
\begin{equation}
  \label{eq:4}
  \frac{\Psi^{(\mu)}_{\mathfrak{g}}}{\prod_{\alpha\in \Delta^{+}_{\bot}}\left(1-e^{-\alpha}\right)^{\mathrm{mult}(\alpha)}}=
    \sum_{u\in W/W_{\bot}}\epsilon(u)e^{\pi_{\af\oplus\mathfrak{h}_d}\left[(u(\mu+\rho)-\rho)\right]+\mathcal{D}}\cdot \mathrm{ch}_{\af_{\bot}}
      L^{\piab(u(\mu+\rho)-\rho)-\mathcal{D}}
\end{equation}
\begin{proof}
  Consider $\Psi^{(\mu)}_{\mathfrak{g}}$. It is given by the sum over the Weyl group.
  \begin{equation}
    \label{eq:6}
    \Psi^{(\mu)}_{\mathfrak{g}}=\sum_{w\in W}\epsilon(w)e^{w(\mu+\rho)-\rho}
  \end{equation}
  We can divide the summation to the sum over $W_{\bot}$ and sum over all the classes of $W/W_{\bot}$: $w=vu,\; v\in W_{\bot},\; u \in W/W_{\bot}$. We have the following obvious property of the projections:
  \begin{equation}
    \label{eq:5}
    u(\mu+\rho)=\pia(u(\mu+\rho))+\piab(u(\mu+\rho))+\pi_{(\mathfrak{h}\setminus
      (\mathfrak{h}_{\af}+\mathfrak{h}_{\af_{\bot}}))^*}(u(\mu+\rho))
  \end{equation}
  Let's denote $\mathfrak{h}\setminus (\mathfrak{h}_{\af}+\mathfrak{h}_{\af_{\bot}})$ by $\mathfrak{h}_d$,
  $\mathfrak{h}=\mathfrak{h}_{\af}+\mathfrak{h}_{\af_{\bot}}+\mathfrak{h}_d$. Since $\mathfrak{h}_{\af}^*,\mathfrak{h}_d^*$ are invariant under the action of $v\in W_{\bot}$, we have
  \begin{equation}
    \label{eq:8}
    vu(\mu+\rho)-\rho=v\cdot(\piab(u(\mu+\rho))-\rho_{\afb}+\rho_{\afb})-\rho_{\afb}+\pia(u(\mu+\rho))-\rho+\rho_{\afb}+\pi_{\mathfrak{h}_d^*}(u(\mu+\rho))
  \end{equation}
  Only the first term depends on $v$. So
  \begin{equation}
    \label{eq:9}
    \frac{\Psi^{(\mu)}_{\mathfrak{g}}}{\mathrm{R}_{\afb}}=
    \frac{\sum_{u\in W/W_{\bot}}\epsilon(u)e^{\pia(u(\mu+\rho))-\rho+\rho_{\afb}+\pi_{\mathfrak{h}_d^*}(u(\mu+\rho))}\cdot
      \sum_{v\in W_{\bot}}\epsilon(v)e^{v(\piab u(\mu+\rho)-\rho_{\afb}+\rho_{\afb})-\rho_{\afb}}}{\mathrm{R}_{\afb}}
  \end{equation}
  We can rewrite the r.h.s. as
  \begin{equation}
    \label{eq:10}
    \sum_{u\in W/W_{\bot}}\epsilon(u)e^{\pi_{(\af\oplus\mathfrak{h}_d)}\left[u(\mu+\rho)-\rho\right]+\pi_{(\af\oplus \mathfrak{h}_d)}\cdot \rho-\rho+\rho_{\afb}}\cdot
    \mathrm{ch}_{\afb}L^{\pi_{\afb}u(\mu+\rho)-\rho_{\afb}}.
  \end{equation}
  Since $\pi_{\afb}(u(\mu+\rho)-\rho)-(\rho_{\afb}-\pi_{\afb}\rho)=\pi_{\afb}(u(\mu+\rho)-\rho)-\mathcal{D}$ and $\pi_{(\af\oplus\mathfrak{h}_d)}\rho-\rho=-\pi_{\afb}\rho$ we get the statement of the Lemma.

\end{proof}
\end{lemma}
\subsection{Modular properties of the dual functions}
\label{sec:modul-prop-dual}

\section{Possible applications}
\label{sec:poss-appl}

\section{Conclusion}
\label{sec:conclusion}



\bibliography{article}{}
\bibliographystyle{utphys}

\end{document}
