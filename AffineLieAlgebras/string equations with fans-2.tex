
\documentclass{article}
\usepackage{amsfonts}

%%%%%%%%%%%%%%%%%%%%%%%%%%%%%%%%%%%%%%%%%%%%%%%%%%%%%%%%%%%%%%%%%%%%%%%%%%%%%%%%%%%%%%%%%%%%%%%%%%%
\usepackage{graphicx}
\usepackage{amsmath}

%TCIDATA{OutputFilter=LATEX.DLL}
%TCIDATA{Created=Fri Jun 19 20:51:42 2009}
%TCIDATA{LastRevised=Sun Jun 21 12:26:33 2009}
%TCIDATA{<META NAME="GraphicsSave" CONTENT="32">}
%TCIDATA{<META NAME="DocumentShell" CONTENT="General\Blank Document">}
%TCIDATA{CSTFile=LaTeX article (bright).cst}

\newtheorem{theorem}{Theorem}
\newtheorem{acknowledgement}[theorem]{Acknowledgement}
\newtheorem{algorithm}[theorem]{Algorithm}
\newtheorem{axiom}[theorem]{Axiom}
\newtheorem{case}[theorem]{Case}
\newtheorem{claim}[theorem]{Claim}
\newtheorem{conclusion}[theorem]{Conclusion}
\newtheorem{condition}[theorem]{Condition}
\newtheorem{conjecture}[theorem]{Conjecture}
\newtheorem{corollary}[theorem]{Corollary}
\newtheorem{criterion}[theorem]{Criterion}
\newtheorem{definition}[theorem]{Definition}
\newtheorem{example}[theorem]{Example}
\newtheorem{exercise}[theorem]{Exercise}
\newtheorem{lemma}[theorem]{Lemma}
\newtheorem{notation}[theorem]{Notation}
\newtheorem{problem}[theorem]{Problem}
\newtheorem{proposition}[theorem]{Proposition}
\newtheorem{remark}[theorem]{Remark}
\newtheorem{solution}[theorem]{Solution}
\newtheorem{summary}[theorem]{Summary}
\newenvironment{proof}[1][Proof]{\textbf{#1.} }{\ \rule{0.5em}{0.5em}}
%\input{tcilatex}

\begin{document}


\section{\protect\bigskip Basic definitions and relations.}

\bigskip

Consider the affine Lie algebra $\frak{g}$ with the underlying
finite-dimensional subalgebra $\overset{\circ }{\frak{g}}$ .

The following notation will be used:

$L^{\mu }$\ \ -- the integrable module of $\frak{g}$ with the highest weight
$\mu $\ ;

$r$ -- the rank of the algebra $\frak{g}$ ;

$\Delta $ -- the root system; $\Delta ^{+}$ -- the positive root system for $%
\frak{g}$ ;

$\mathrm{mult}\left( \alpha \right) $ -- the multiplicity of the root $%
\alpha $ in $\Delta $ ;

$\overset{\circ }{\Delta }$ -- the finite root system of the subalgebra $%
\overset{\circ }{\frak{g}}$ ;

$\mathcal{N}^{\mu }$ -- the weight diagram of $L^{\mu }$ ;

$W$ -- the corresponding Weyl group;

$C^{\left( 0\right) }$ -- the fundamental Weyl chamber;

$\overline{C_{k}^{\left( 0\right) }}$ -- the intersection of the closure of
the fundamental Weyl chamber $C^{\left( 0\right) }$\ with the plane with
fixed level $k=\mathrm{const}$ .

$\rho $\ -- the Weyl vector;

$\epsilon \left( w\right) :=\det \left( w\right) $ , $w \in W$;

$\alpha _{i}$ -- the $i$-th simple root for $\frak{g}$ ; $i=0,\ldots ,r$ ;

$\delta $ -- the imaginary root of $\frak{g}$ ;

$\alpha _{i}^{\vee }$ -- the simple coroot for $\frak{g}$ , $i=0,\ldots ,r$ ;

$\overset{\circ }{\xi }$ -- the finite (classical) part of the weight $\xi
\in P$ ;

$\lambda =\left( \overset{\circ }{\lambda };k;n\right) $ -- the
decomposition of an affine weight indicating the finite part $\overset{\circ
}{\lambda }$, level $k$ and grade $n$\ .

$P$ -- the weight lattice;

$Q$ -- the root lattice;

$M:=$

\noindent $=\left\{
\begin{array}{c}
\sum_{i=1}^{r}\mathbf{Z}\alpha _{i}^{\vee }\text{ for untwisted algebras or }%
A_{2r}^{\left( 2\right) }, \\
\sum_{i=1}^{r}\mathbf{Z}\alpha _{i}\text{ for }A_{r}^{\left( u\geq 2\right) }%
\text{ and }A\neq A_{2r}^{\left( 2\right) },
\end{array}
\right\} ;$

$\mathcal{E}$\ -- the group algebra of the group $P$ ;

$\Theta _{\lambda }:=e^{-\frac{\left| \lambda \right| ^{2}}{2k}\delta
}\sum\limits_{\alpha \in M}e^{t_{\alpha }\circ \lambda }$ -- the classical
theta-function;

$A_{\lambda }:=\sum\limits_{s\in \overset{\circ }{W}}\epsilon (s)\Theta
_{s\circ \lambda }$ ;

$\Psi ^{\left( \mu \right) }:=e^{\frac{\left| \mu +\rho \right| ^{2}}{2k}%
\delta \ -\ \rho }A_{\mu +\rho }=$

\noindent $=\sum\limits_{w\in W}\epsilon (w)e^{w\circ(\mu +\rho )-\rho }$ --
the singular weight element for the $\frak{g}$-module $L^{\mu }$;

$m_{\xi }^{\left( \mu \right) }$ -- the multiplicity of the weight $\xi \in
P $ in the module $L^{\mu }$ ;

$\mathrm{ch}\left( L^{\mu }\right) $ -- the formal character of $L^{\mu }$ ;

$\mathrm{ch}\left( L^{\mu }\right) =\frac{\sum_{w\in W}\epsilon (w)e^{w\circ
(\mu +\rho )-\rho }}{\prod_{\alpha \in \Delta ^{+}}\left( 1-e^{-\alpha
}\right) ^{\mathrm{{mult}\left( \alpha \right) }}}=\frac{\Psi ^{\left( \mu
\right) }}{\Psi ^{\left( 0\right) }}$ -- the Weyl-Kac formula;

$R:=\prod_{\alpha \in \Delta ^{+}}\left( 1-e^{-\alpha }\right) ^{\mathrm{{%
mult}\left( \alpha \right) }}=\Psi ^{\left( 0\right) }$ -- the denominator;

$\max (\mu )$ -- the set of maximal weights of $L^{\mu }$ ;

$\sigma _{\xi }^{\mu }\left( q\right) =\sum_{n=0}^{\infty }m_{\left( \xi
-n\delta \right) }^{\left( \mu \right) }q^{n}$ -- the string function
through the maximal weight $\xi $ .

We shall use the recurrent property for the string coefficients that was
found in the paper P.Kulish, V.Lyakhovsky, ''String functions for affinr Lie
algebras integrable modules'', SIGMA, 4, (2008), 085, (arxiv: 0812.2381)).
Let $\overline{C_{k;0}^{\left( 0\right) }}$ be the intersection of $%
\overline{C_{k}^{\left( 0\right) }}$ with the plane $\delta =0$ , that is
the ''classical'' part of the closure of the affine Weyl chamber at level $k$
. For the extended string functions $\sigma _{j}^{\mu ,k}$ belonging to the $%
\overline{C_{k}^{\left( 0\right) }}$ the sets$\Sigma _{k}^{\mu }$ and $\Xi
_{k}^{\mu }$ were introduced:
\begin{equation}
\Sigma _{k}^{\mu }:=\left\{ \sigma _{j}^{\mu ,k}\mid j=1,\ldots ,p_{\max
}^{\left( \mu \right) }\right\} .  \label{strings-for-mu}
\end{equation}
\begin{equation}
\Xi _{k}^{\mu }:=\left\{ \xi =\pi \circ \zeta \mid \zeta \in \mathcal{Z}%
_{k}^{\mu }\right\}  \label{maxes-for-mu}
\end{equation}
Let $L^{\mu }\left( \frak{g}\right) $ be the integrable highest weight
module of $\frak{g}$, $\mu =\left( \overset{\circ }{\mu };k;0\right) $, $%
p_{\max }^{\left( \mu \right) }:=\#\left( \Xi _{k}^{\mu }\right) $\ , $\xi
_{j}=\left( \overset{\circ }{\xi _{j}};k;n_{j}\right) \in \Xi _{k}^{\mu
}+n_{j}\delta $ , let $F\Psi \left( \overset{\circ }{\xi _{j}}\right) $ be
the full folded fan for $\overset{\circ }{\xi _{j}}$ and $\eta _{j,s}\left(
n\right) =-\sum_{\tilde{w}_{j,s},}\epsilon (\tilde{w}_{j,s})$ where the
summation is over the elements $\tilde{w}_{j,s}$ of $W$ satisfying the
equation $w_{\phi \left( \xi ,w\right) }\circ \left( \xi _{j}-\left( \tilde{w%
}_{j,s}\circ \rho -\rho \right) \right) =\left( \overset{\circ }{\xi _{s}}%
;k;n_{j}+n\right) ,$ then for the string function coefficients $%
m_{s,n_{j}+n}^{\left( \mu \right) }$ the following recurrent relation holds:
\begin{equation}
\sum_{s=1}^{p_{\max }^{\left( \mu \right) }}\sum_{n=-n_{j}}^{0}\eta
_{j,s}\left( n\right) m_{s,n_{j}+n}^{\left( \mu \right) }=-\delta _{\xi
_{j},\mu }.  \label{recursion-prop-string}
\end{equation}
Suppose here $\overset{\circ }{\xi _{j}}=\mu $ and $n_{j}\neq 0$\ , then $%
\xi $ cannot coincide with $\mu $\ and we have
\begin{equation*}
\sum_{s=1}^{p_{\max }^{\left( \mu \right) }}\sum_{n=-n_{j}}^{0}\eta
_{j,s}\left( n\right) m_{s,n_{j}+n}^{\left( \mu \right) }=0.
\end{equation*}
We can multiply it by $q^{-n_{j}}$. The corresponding equations for other
strings $\sigma _{i}^{\mu ,k},i\neq j$\ evidently have zeros in the r.h.s.
and all of them can be multiplied by $q^{-n_{i}}$\ . Let us perform the
summation
\begin{eqnarray*}
\sum_{n_{j}=1}^{-\infty }\sum_{s=1}^{p_{\max }^{\left( \mu \right)
}}\sum_{n=-n_{j}}^{0}\eta _{j,s}\left( n\right)
q^{-n_{j}}m_{s,n_{j}+n}^{\left( \mu \right) } &=&0, \\
\sum_{n_{i}=0}^{-\infty }\sum_{s=1}^{p_{\max }^{\left( \mu \right)
}}\sum_{n=-n_{i}}^{0}\eta _{i,s}\left( n\right)
q^{-n_{i}}m_{s,n_{i}+n}^{\left( \mu \right) } &=&0.
\end{eqnarray*}
When $n_{j}=0$ that is $\xi =\mu $\ the recurrent relation degenerates into
an obvious one
\begin{equation*}
\sum_{s=1}^{p_{\max }^{\left( \mu \right) }}\eta _{j,s}\left( 0\right)
m_{s,0}^{\left( \mu \right) }=-1,
\end{equation*}
that can be joint with the first one. Thus the full set of relations is
\begin{eqnarray*}
\sum_{n_{j}=0}^{-\infty }\sum_{s=1}^{p_{\max }^{\left( \mu \right)
}}\sum_{n=-n_{j}}^{0}\eta _{j,s}\left( n\right)
q^{-n_{j}}m_{s,n_{j}+n}^{\left( \mu \right) } &=&-1, \\
\sum_{n_{i}=0}^{-\infty }\sum_{s=1}^{p_{\max }^{\left( \mu \right)
}}\sum_{n=-n_{i}}^{0}\eta _{i,s}\left( n\right)
q^{-n_{i}}m_{s,n_{i}+n}^{\left( \mu \right) } &=&0.
\end{eqnarray*}
The new variable
\begin{equation*}
r_{i}=n_{i}+n.
\end{equation*}
can have the same limits as $n_{i}$ (due to the properties of $%
m_{s,n_{j}+n}^{\left( \mu \right) }$):
\begin{eqnarray*}
\sum_{r_{j}=0}^{-\infty }\sum_{s=1}^{p_{\max }^{\left( \mu \right)
}}\sum_{n=-n_{j}}^{0}q^{n}\eta _{j,s}\left( n\right)
q^{-r_{j}}m_{s,r_{j}}^{\left( \mu \right) } &=&-1, \\
\sum_{r_{i}=0}^{-\infty }\sum_{s=1}^{p_{\max }^{\left( \mu \right)
}}\sum_{n=-n_{i}}^{0}q^{n}\eta _{i,s}\left( n\right)
q^{-r_{i}}m_{s,r_{i}}^{\left( \mu \right) } &=&0.
\end{eqnarray*}
thus we get
\begin{eqnarray*}
\sum_{s=1}^{p_{\max }^{\left( \mu \right) }}\sum_{n=-n_{j}}^{0}q^{n}\eta
_{j,s}\left( n\right) \sigma _{s}^{\mu } &=&-1, \\
\sum_{s=1}^{p_{\max }^{\left( \mu \right) }}\sum_{n=-n_{i}}^{0}q^{n}\eta
_{i,s}\left( n\right) \sigma _{s}^{\mu } &=&0.
\end{eqnarray*}
Now the summation limits for $n$\ can be extended (due to the properties of $%
\eta _{j,s}\left( n\right) $) and the final system of relations for the
string functions is obtained
\begin{eqnarray*}
\sum_{s=1}^{p_{\max }^{\left( \mu \right) }}\sigma _{s}^{\mu
}\sum_{n=0}^{\infty }q^{n}\eta _{j,s}\left( n\right) &=&-1, \\
\sum_{s=1}^{p_{\max }^{\left( \mu \right) }}\sigma _{s}^{\mu
}\sum_{n=0}^{\infty }q^{n}\eta _{i,s}\left( n\right) &=&0. \\
i &=&1,2,\ldots ,\widehat{j},\ldots ,p_{\max }^{\left( \mu \right) }
\end{eqnarray*}

\section{Example}

Consider $\frak{g}=A_{2}^{\left( 2\right) }$.\ The fan of the injection $%
\frak{h}\rightarrow \frak{g}$ consists of the sets
\begin{equation*}
\left\{
\begin{array}{c}
\left( -3p,0,-\frac{3}{2}p^{2}+\frac{1}{2}p;+1\right) \\
\left( -3p-1,0,-\frac{3}{2}p^{2}-\frac{1}{2}p;-1\right)
\end{array}
\right\} _{p\in \mathbf{Z}},
\end{equation*}
(the last coordinate is the anomalous multiplicity of the fan weights). In
the level $K=3$\ we have one congruence class comprising two highest weight
integrable modules, with $\mu =\frac{1}{2}e_{1}$ and $\mu =\frac{3}{2}e_{1}$%
. Consider the fan glissing (with the parameter $n$) along the strings $%
\left\{ \sigma _{i}^{\mu }\mid i=1,2\right\} $\ contained in the chamber $%
\overline{C_{k}^{\left( 0\right) }}$\ . For $i=1$ $\left( \mu =\frac{1}{2}%
e_{1}\right) $ we get
\begin{equation*}
\left\{
\begin{array}{c}
\left( 3p+\frac{3}{2},3,\frac{3}{2}p^{2}+\frac{1}{2}p-n;+1\right) \\
\left( 3p+\frac{1}{2},3,\frac{3}{2}p^{2}-\frac{1}{2}p-n;-1\right)
\end{array}
\right\} ,
\end{equation*}
and for $i=2$ $\left( \mu =\frac{3}{2}e_{1}\right) $ --
\begin{equation*}
\left\{
\begin{array}{c}
\left( 3p+\frac{5}{2},3,\frac{3}{2}p^{2}+\frac{1}{2}p-n;+1\right) \\
\left( 3p+\frac{3}{2},3,\frac{3}{2}p^{2}-\frac{1}{2}p-n;-1\right)
\end{array}
\right\} ,
\end{equation*}
here $p\in \mathbf{Z,\quad }n\in \mathbf{Z}_{+}\mathbf{.}$\

To get the folded fans in these two positions we are to act by the $W\left(
A_{2}^{\left( 2\right) }\right) $ group transformations (parameterised by $%
q\in \mathbf{Z}$ ) on the corresponding shifted unfolded fans. The folded
sets are presented below (In the square brackets the first row is
transformed by the pure translations, the second -- by translations with the
classical reflections.) For the first fan --
\begin{equation*}
\widetilde{\widetilde{F\Psi }}_{1}=\left\{
\begin{array}{c}
\left[
\begin{array}{c}
\left( 3p+\frac{3}{2}-3q,3,\frac{3}{2}p^{2}+\frac{1}{2}p-n+\frac{3}{2}q-%
\frac{3}{2}q^{2}+3pq;+1\right)  \\
\left( -3p-\frac{3}{2}-3q,3,\frac{3}{2}p^{2}+\frac{1}{2}p-n-\frac{3}{2}q-%
\frac{3}{2}q^{2}-3pq;+1\right)
\end{array}
\right]  \\
\left[
\begin{array}{c}
\left( 3p+\frac{1}{2}-3q,3,\frac{3}{2}p^{2}-\frac{1}{2}p-n+\frac{1}{2}q-%
\frac{3}{2}q^{2}+3pq;-1\right)  \\
\left( -3p-\frac{1}{2}-3q,3,\frac{3}{2}p^{2}-\frac{1}{2}p-n-\frac{1}{2}q-%
\frac{3}{2}q^{2}-3pq;-1\right)
\end{array}
\right]
\end{array}
\right\} .
\end{equation*}
and for the second fan
\begin{equation*}
\widetilde{\widetilde{F\Psi }}_{2}=\left\{
\begin{array}{c}
\left[
\begin{array}{c}
\left( 3p+\frac{5}{2}-3q,3,\frac{3}{2}p^{2}+\frac{1}{2}p-n+\frac{5}{2}q-%
\frac{3}{2}q^{2}+3pq;+1\right)  \\
\left( -3p-\frac{5}{2}-3q,3,\frac{3}{2}p^{2}+\frac{1}{2}p-n-\frac{5}{2}q-%
\frac{3}{2}q^{2}+3pq;+1\right)
\end{array}
\right]  \\
\left[
\begin{array}{c}
\left( 3p+\frac{3}{2}-3q,3,\frac{3}{2}p^{2}-\frac{1}{2}p-n+\frac{3}{2}q-%
\frac{3}{2}q^{2}+3pq;-1\right)  \\
\left( -3p-\frac{3}{2}-3q,3,\frac{3}{2}p^{2}-\frac{1}{2}p-n-\frac{3}{2}q-%
\frac{3}{2}q^{2}-3pq;-1\right)
\end{array}
\right]
\end{array}
\right\}
\end{equation*}
We are interested only on the vectors in the $\overline{C_{k}^{\left(
0\right) }}\cap P.$ Impose the conditions
\begin{equation*}
\left( \overset{\circ }{\widetilde{\widetilde{f\psi }}}\right) =\frac{1}{2}%
\quad or\quad \frac{3}{2}.
\end{equation*}
There are the following solutions:
\begin{equation*}
\mathrm{for}\quad \widetilde{\widetilde{F\Psi }}_{1}=\left\{
\begin{array}{c}
\left[
\begin{array}{c}
\left( \overset{\circ }{\widetilde{\widetilde{f\psi }}}\right) =\frac{3}{2}%
\quad =>\quad q=p, \\
\left( \overset{\circ }{\widetilde{\widetilde{f\psi }}}\right) =\frac{3}{2}%
\quad =>\quad q=-p-1,=>\quad \left(
\begin{array}{c}
\mathrm{the\quad same\quad vectors,} \\
\mathrm{to\quad be\quad ignored}
\end{array}
\right)
\end{array}
\right]  \\
\left[
\begin{array}{c}
\left( \overset{\circ }{\widetilde{\widetilde{f\psi }}}\right) =\frac{1}{2}%
\quad =>\quad q=p, \\
\left( \mathrm{no\quad solutions}\right)
\end{array}
\right]
\end{array}
\right\}
\end{equation*}
\begin{equation*}
\mathrm{for}\quad \widetilde{\widetilde{F\Psi }}_{2}=>\left\{
\begin{array}{c}
\left[
\begin{array}{c}
\left( \mathrm{no\quad solutions}\right)  \\
\left( \overset{\circ }{\widetilde{\widetilde{f\psi }}}\right) =\frac{1}{2}%
\quad =>\quad q=-p-1,
\end{array}
\right]  \\
\left[
\begin{array}{c}
\left( \overset{\circ }{\widetilde{\widetilde{f\psi }}}\right) =\frac{3}{2}%
\quad =>\quad q=p,=>\left(
\begin{array}{c}
\mathrm{the\quad same\quad vectors,} \\
\mathrm{to\quad be\quad ignored}
\end{array}
\right)  \\
\left( \overset{\circ }{\widetilde{\widetilde{f\psi }}}\right) =\frac{3}{2}%
\quad =>\quad q=-p-1.
\end{array}
\right]
\end{array}
\right\} .
\end{equation*}

Notice that both sets can be obtained by applying only the transformations
from one class of elements in the group $W\left( A_{2}^{\left( 2\right)
}\right) $ : the pure translations in the first fan and the translations
with reflections in the second. The subscription ''the same vectors, to be
ignored'' signify that the solution leads finally to the set of folded
vectors bequivalent to the second sets in the same square brackets. This
effect is due to the nontrivial stability subgroup $W_{\xi }\approx Z_{2}$
corresponding to the fixed solution of the Diafantine equations and the set
of vectors that are are included in the fan can be considered as obtained by
applying the representatives of the factor-space $W/W_{\xi }$.

Substituting the solutions of the Diafantine equations into the sets of
corresponding folded vectors we obtain the two folded fans:
\begin{equation*}
F\Psi _{1}=\left\{
\begin{array}{c}
\left[ \frac{1}{2},3,3p^{2}-n;-1\right]  \\
\left[ \frac{3}{2},3,3p^{2}+2p-n;+1\right]
\end{array}
\right\}
\end{equation*}
\begin{equation*}
F\Psi _{2}=\left\{
\begin{array}{c}
\left[ \frac{1}{2},3,3p^{2}+3p-n+1;+1\right]  \\
\left[ \frac{3}{2},3,3p^{2}+p-n;-1\right]
\end{array}
\right\}
\end{equation*}
These equations lead to the recurrent relation for the string coefficients:
\begin{equation*}
+\sum_{p}\left( \sigma _{2}^{\frac{3}{2}}\right)
_{n-3p^{2}-2p}-\sum_{p}\left( \sigma _{1}^{\frac{3}{2}}\right)
_{n-3p^{2}}=\delta _{n,0};
\end{equation*}
\begin{equation*}
-\sum_{p}\left( \sigma _{2}^{\frac{3}{2}}\right)
_{n-3p^{2}-p}+\sum_{p}\left( \sigma _{1}^{\frac{3}{2}}\right)
_{n-3p^{2}-3p-1}=0;
\end{equation*}
According to the general algorithm one must extract the $n=0$ equation from
the first set of relations, multiply both sets by $q^{n}$ , sum over $n$ and
return the $n=0$ equation. The result is
\begin{equation*}
+\sum_{n=0}^{\infty }q^{n}\sum_{p}\left( \sigma _{2}^{\frac{3}{2}}\right)
_{n-3p^{2}-2p}-\sum_{n=0}^{\infty }q^{n}\sum_{p}\left( \sigma _{1}^{\frac{3}{%
2}}\right) _{n-3p^{2}}=1;
\end{equation*}
\begin{equation*}
-\sum_{n=0}^{\infty }q^{n}\sum_{p}\left( \sigma _{2}^{\frac{3}{2}}\right)
_{n-3p^{2}-p}+\sum_{n=0}^{\infty }q^{n}\sum_{p}\left( \sigma _{1}^{\frac{3}{2%
}}\right) _{n-3p^{2}-3p-1}=0;
\end{equation*}
Now shift the variable $n$ , in each sum this is performed separately with
\begin{eqnarray*}
n &=&k+3p^{2}+2p;\quad n=k+3p^{2}; \\
n &=&k+3p^{2}+p;\quad n=k+3p^{2}+3p+1;
\end{eqnarray*}
Originally in the recurrent relations the summation over $p$ is finite and
we can chang the order of summations,
\begin{equation*}
+\sum_{p}q^{3p^{2}+2p}\sum_{n=0}^{\infty }q^{k}\left( \sigma _{2}^{\frac{3}{2%
}}\right) _{k}-\sum_{p}q^{3p^{2}}\sum_{n=0}^{\infty }q^{k}\left( \sigma
_{1}^{\frac{3}{2}}\right) _{k}=1;
\end{equation*}
\begin{equation*}
-\sum_{p}q^{3p^{2}+p}\sum_{n=0}^{\infty }q^{k}\left( \sigma _{2}^{\frac{3}{2}%
}\right) _{k}+\sum_{p}q^{3p^{2}+3p+1}\sum_{n=0}^{\infty }q^{k}\left( \sigma
_{1}^{\frac{3}{2}}\right) _{k}=0;
\end{equation*}
Now remember that our second string starts with zero. To obtain the relation
for the canonical string functions we must shift the numbers of the
coefficients for $\sigma _{2}^{\frac{3}{2}}$ by the unity. The final result
is the following relations for string functions for the representation $%
L^{\omega _{0}+\omega _{1}}(A_{2}^{\left( 2\right) })$ :
\begin{equation*}
+\sum_{p}q^{3p^{2}+2p+1}\sigma _{2}-\sum_{p}q^{3p^{2}}\sigma _{1}=1;
\end{equation*}
\begin{equation*}
-\sum_{p}q^{3p^{2}+p+1}\sigma _{2}+\sum_{p}q^{3p^{2}+3p+1}\sigma _{1}=0;
\end{equation*}

\end{document}
