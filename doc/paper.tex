%% This template can be used to write a paper for
%% Computer Physics Communications using LaTeX.
%% For authors who want to write a computer program description,
%% an example Program Summary is included that only has to be
%% completed and which will give the correct layout in the
%% preprint and the journal.
%% The `elsarticle' style is used and more information on this style
%% can be found at 
%% http://www.elsevier.com/wps/find/authorsview.authors/elsarticle.
%%
%%
\documentclass[preprint,12pt]{elsarticle}

%% Use the option review to obtain double line spacing
%% \documentclass[preprint,review,12pt]{elsarticle}

%% Use the options 1p,twocolumn; 3p; 3p,twocolumn; 5p; or 5p,twocolumn
%% for a journal layout:
%% \documentclass[final,1p,times]{elsarticle}
%% \documentclass[final,1p,times,twocolumn]{elsarticle}
%% \documentclass[final,3p,times]{elsarticle}
%% \documentclass[final,3p,times,twocolumn]{elsarticle}
%% \documentclass[final,5p,times]{elsarticle}
%% \documentclass[final,5p,times,twocolumn]{elsarticle}

%% if you use PostScript figures in your article
%% use the graphics package for simple commands
\usepackage{graphics}
\usepackage{hyperref}
%% or use the graphicx package for more complicated commands
%% \usepackage{graphicx}
%% or use the epsfig package if you prefer to use the old commands
%% \usepackage{epsfig}

%% The amssymb package provides various useful mathematical symbols
\usepackage{amssymb}
%% The amsthm package provides extended theorem environments
%% \usepackage{amsthm}

%% The lineno packages adds line numbers. Start line numbering with
%% \begin{linenumbers}, end it with \end{linenumbers}. Or switch it on
%% for the whole article with \linenumbers after \end{frontmatter}.
%% \usepackage{lineno}

%% natbib.sty is loaded by default. However, natbib options can be
%% provided with \biboptions{...} command. Following options are
%% valid:

%%   round  -  round parentheses are used (default)
%%   square -  square brackets are used   [option]
%%   curly  -  curly braces are used      {option}
%%   angle  -  angle brackets are used    <option>
%%   semicolon  -  multiple citations separated by semi-colon
%%   colon  - same as semicolon, an earlier confusion
%%   comma  -  separated by comma
%%   numbers-  selects numerical citations
%%   super  -  numerical citations as superscripts
%%   sort   -  sorts multiple citations according to order in ref. list
%%   sort&compress   -  like sort, but also compresses numerical citations
%%   compress - compresses without sorting
%%
%% \biboptions{comma,round}

% \biboptions{}

%% This list environment is used for the references in the
%% Program Summary
%%
\newcounter{bla}
\newenvironment{refnummer}{%
\list{[\arabic{bla}]}%
{\usecounter{bla}%
 \setlength{\itemindent}{0pt}%
 \setlength{\topsep}{0pt}%
 \setlength{\itemsep}{0pt}%
 \setlength{\labelsep}{2pt}%
 \setlength{\listparindent}{0pt}%
 \settowidth{\labelwidth}{[9]}%
 \setlength{\leftmargin}{\labelwidth}%
 \addtolength{\leftmargin}{\labelsep}%
 \setlength{\rightmargin}{0pt}}}
 {\endlist}

\journal{Computer Physics Communications}

\begin{document}

\begin{frontmatter}

%% Title, authors and addresses

%% use the tnoteref command within \title for footnotes;
%% use the tnotetext command for the associated footnote;
%% use the fnref command within \author or \address for footnotes;
%% use the fntext command for the associated footnote;
%% use the corref command within \author for corresponding author footnotes;
%% use the cortext command for the associated footnote;
%% use the ead command for the email address,
%% and the form \ead[url] for the home page:
%%
%% \title{Title\tnoteref{label1}}
%% \tnotetext[label1]{}
%% \author{Name\corref{cor1}\fnref{label2}}
%% \ead{email address}
%% \ead[url]{home page}
%% \fntext[label2]{}
%% \cortext[cor1]{}
%% \address{Address\fnref{label3}}
%% \fntext[label3]{}

\title{{\bf Affine.m} -- {\it Mathematica} package for computations in representation theory of finite-dimensional and affine Lie algebras}

%% use optional labels to link authors explicitly to addresses:
%% \author[label1,label2]{<author name>}
%% \address[label1]{<address>}
%% \address[label2]{<address>}

\author[a,b]{Anton Nazarov\corref{author}}
%\author[a,b]{Second Author}
%\author[b]{Third Author}

\cortext[author] {Corresponding author.\\\textit{E-mail address:} antonnaz@gmail.com}
\address[a]{Department of High Energy Physics, Faculty of physics, SPb State University\\ 198904, Sankt-Petersburg, Russia}
\address[b]{Chebyshev Laboratory, Faculty of Mathematics and Mechanics, SPb State University\\ 199178, Saint-Petersburg, Russia}

\begin{abstract}
%% Text of abstract
We present {\bf Affine.m} -- program for computations in representation theory of finite-dimensional and affine Lie algebras. Algorithms are based upon the properties of weights and Weyl symmetry. Central problems are weight multiplicity computation, branching of representation to representations of subalgebra and tensor product decomposition. These problems has numerous applications in physics.

Unique features of our programs are the treatment of affine Lie algebras and implementation in popular computer algebra system {\it Mathematica}.

% A submitted program is expected to be of benefit to other physicists or physical chemists, or be an exemplar of good programming practice, or illustrate new or novel programming techniques which are of importance to some branch of computational physics or physical chemistry.
%
% Acceptable program descriptions can take different forms. The following Long Write-Up structure is a suggested structure but it is not obligatory. Actual structure will depend on the length of the program, the extent to which the algorithms or software have already been described in literature, and the detail provided in the user manual.

%Your manuscript and figure sources should be submitted through the Elsevier Editorial System (EES) by using the online submission tool at \\
% http://www.ees.elsevier.com/cpc.

%In addition to the manuscript you must supply: the program source code; job control scripts, where applicable; a README file giving the names and a brief description of all the files that make up the package and clear instructions on the installation and execution of the program; sample input and output data for at least one comprehensive test run; and, where appropriate, a user manual. These should be sent, via email as a compressed archive file, to the CPC Program Librarian at cpc@qub.ac.uk.

\end{abstract}

\begin{keyword}
%% keywords here, in the form: keyword \sep keyword
%keyword1; keyword2; keyword3; etc.

Mathematica; Lie algebra; affine Lie algebra; Kac-Moody algebra; root system; weights; irreducible modules, CFT, Integrable systems
\end{keyword}

\end{frontmatter}

%%
%% Start line numbering here if you want
%%
% \linenumbers

% Computer program descriptions should contain the following
% PROGRAM SUMMARY.

{\bf PROGRAM SUMMARY}%/NEW VERSION PROGRAM SUMMARY}
  %Delete as appropriate.

\begin{small}
\noindent
{\em Manuscript Title:}{\bf Affine.m} -- {\it Mathematica} package for computations in representation theory of finite-dimensional and affine Lie algebras                                       \\
{\em Authors:}Anton Nazarov                                                \\
{\em Program Title:}Affine.m                                          \\
{\em Journal Reference:}                                      \\
  %Leave blank, supplied by Elsevier.
{\em Catalogue identifier:}                                   \\
  %Leave blank, supplied by Elsevier.
{\em Licensing provisions:}none                                   \\
  %enter "none" if CPC non-profit use license is sufficient.
{\em Programming language:}Mathematica                                   \\
{\em Computer:}i386-i686, x86\textunderscore 64                                               \\
  %Computer(s) for which program has been designed.
{\em Operating system:} Linux, Windows, MacOS, Solaris                                       \\
  %Operating system(s) for which program has been designed.
{\em RAM:} 5-500 Mb                                              \\
  %RAM in bytes required to execute program with typical data.
%{\em Number of processors used:}                              \\
  %If more than one processor.
%{\em Supplementary material:}                                 \\
  % Fill in if necessary, otherwise leave out.
{\em Keywords:} Mathematica; Lie algebra; affine Lie algebra; Kac-Moody algebra; root system; weights; irreducible modules, CFT, Integrable systems\\
  % Please give some freely chosen keywords that we can use in a
  % cumulative keyword index.
{\em Classification:} 5 Computer Algebra, 4.2 Other algebras and groups                                         \\
  %Classify using CPC Program Library Subject Index, see (
  % http://cpc.cs.qub.ac.uk/subjectIndex/SUBJECT_index.html)
  %e.g. 4.4 Feynman diagrams, 5 Computer Algebra.
%{\em External routines/libraries:}                            \\
  % Fill in if necessary, otherwise leave out.
%{\em Subprograms used:}                                       \\
  %Fill in if necessary, otherwise leave out.
%{\em Catalogue identifier of previous version:}*              \\
  %Only required for a New Version summary, otherwise leave out.
%{\em Journal reference of previous version:}*                  \\
  %Only required for a New Version summary, otherwise leave out.
%{\em Does the new version supersede the previous version?:}*   \\
  %Only required for a New Version summary, otherwise leave out.
{\em Nature of problem:}\\
  %Describe the nature of the problem here.
Representation theory of finite-dimensional Lie algebras has many applications in different branches of physics, including elementary particle physics, molecular physics, nuclear physics. Representations of affine Lie algebras appear in string theories and two-dimensional conformal field theory which is used for the description of critical phenomena in two-dimensional systems. Also Lie symmetries play major role in study of quantum integrable systems. 
   \\
{\em Solution method:}\\
  %Describe the method solution here.
We work with weights and roots of finite-dimensional and affine Lie algebras and use Weyl symmetry extensively. Central problems which are the computations of weight multiplicities, branching and fusion coefficients are solved using one general recurrent algorithm based on generalization of Weyl character formula. We also offer alternative implementation based on Freudenthal multiplicity formula which can be faster in some cases. 
   \\
%{\em Reasons for the new version:}*\\
  %Only required for a New Version summary, otherwise leave out.
%   \\
%{\em Summary of revisions:}*\\
  %Only required for a New Version summary, otherwise leave out.
%   \\
{\em Restrictions:}\\
  %Describe any restrictions on the complexity of the problem here.
Computational complexity grows fast  with the rank of algebra, so computations for algebra of rank greater than 8 are non practical. 
   \\
{\em Unusual features:}\\
  %Describe any unusual features of the program/problem here.
We offer the possibility to use traditional mathematical notation for objects in representation theory of Lie algebras in computations if {\bf Affine.m} is used in {\it Mathematica} notebook interface. 
   \\
%{\em Additional comments:}\\
  %Provide any additional comments here.
%   \\
{\em Running time:}\\
  %Give an indication of the typical running time here.
From seconds to days depending on rank of algebra and complexity of representations.
   \\

% \begin{thebibliography}{0}
% \bibitem{1}Reference 1         % This list should only contain those items referenced in the                 
% \bibitem{2}Reference 2         % Program Summary section.   
% \bibitem{3}Reference 3         % Type references in text as [1], [2], etc.
%                                % This list is different from the bibliography at the end of 
%                                % the Long Write-Up.
% \end{thebibliography}
% * Items marked with an asterisk are only required for new versions
% of programs previously published in the CPC Program Library.\\
\end{small}


%% main text
\section{Introduction}
\label{intro}

Representation theory of Lie algebras is of central importance for different areas of physics and mathematics. Lie algebras are used to for the description of symmetries of quantum and classical systems. Computational method in representation theory have long history \cite{belinfante1989survey}, there exist numerous software packages for computations related to Lie algebras \cite{simplie}, \cite{vanleeuwen1994lsp}, \cite{stembridge1995mps,coxweyl}, \cite{fischbacher2002ilp}, \cite{Fuchs:1996dd}. 

Most popular programs \cite{simplie}, \cite{vanleeuwen1994lsp}, \cite{fischbacher2002ilp}, \cite{coxweyl} are created to study representation theory of simple finite-dimensional Lie algebras. The main computational problems are the following:
\begin{enumerate}
\item Construction of root system which is used for compact description of algebra commutation relations.
\item Weyl group traversal which is important due to Weyl symmetry of root system and characters of representations.
\item Calculation of weight multiplicities, branching and fusion coefficients, which are essential for the construction and study of representations. 
\end{enumerate}
There are well-known algorithms for these tasks \cite{moody1982fast}, \cite{stembridge2001computational}, \cite{belinfante1989survey}, \cite{casselman1994machine}. 
The third problem is most computation-intensive. There are two different recurrent algorithms which are based on Weyl character formula and Freudenthal multiplicity formula. In this paper we analyze them. 

Infinite-dimensional Lie algebras also have growing number of applications in physics for example in conformal field theory and study of quantum integrable systems. But infinite-dimensional algebras are much harder to study and quantity of available computer programs is much smaller.

Affine Lie algebras \cite{kac1990idl} constitute important and tractable class of infinite-dimensional Lie algebras. They are constructed as the central extensions of loop algebras of (semi-simple) finite-dimensional Lie algebras and appear naturally in the study of Wess-Zumino-Witten and coset models of conformal field theory \cite{Walton:1999xc}, \cite{difrancesco1997cft}, \cite{Goddard198588}, \cite{Dunbar:1992gh}. 

Construction of affine Lie algebras allows to extend computational algorithms created for finite-dimensional Lie algebras  \cite{Fuchs:1996dd}, \cite{gannon2001algorithms}, \cite{kass1990ala}. The book \cite{kass1990ala} with the tables of multiplicities and other computed characteristics of algebras and representations was published in 1990. But we are not aware of software packages for popular computer algebra systems which can be used to extend these results. 
We address this issue and present {\bf Affine.m} -- {\it Mathematica} package for computations in representation theory of affine and finite-dimensional Lie algebras. 

Our package can be used for study of root and weight systems, Weyl groups, subalgebras, computations of weight multiplicities, branching and fusion coefficients. We describe the features and limitations of the package in present paper and in the manual \cite{affinemanual}. In present paper we also provide representation-theoretical background of implemented algorithms and present some example of computations relevant to physics. 

The paper is started with overview of Lie algebras and their representation theory (Sec. \ref{sec:theor-backgr}). Then we describe datastructures of {\bf Affine.m} used to present different objects related to Lie algebras and their representations (Sec. \ref{sec:core-datastructures} and used in implemented algorithms (Sec. \ref{sec:comp-algor}). Next section consists of physically interesting examples (Sec. \ref{sec:examples}). The paper is concluded with the discussion of possible extensions and refinements (Sec. \ref{sec:conclusion}).

\section{Theoretical background}
\label{sec:theor-backgr}

\section{Core datastructures}
\label{sec:core-datastructures}

\section{Computational algorithms}
\label{sec:comp-algor}

\section{Examples}
\label{sec:examples}

\section{Conclusion}
\label{sec:conclusion}


%% The Appendices part is started with the command \appendix;
%% appendix sections are then done as normal sections
%% \appendix

%% \section{}
%% \label{}

%% References
%%
%% Following citation commands can be used in the body text:
%% Usage of \cite is as follows:
%%   \cite{key}         ==>>  [#]
%%   \cite[chap. 2]{key} ==>> [#, chap. 2]
%%

%% References with bibTeX database:

\bibliographystyle{elsarticle-num}
\bibliography{bibliography}

%% Authors are advised to submit their bibtex database files. They are
%% requested to list a bibtex style file in the manuscript if they do
%% not want to use elsarticle-num.bst.

%% References without bibTeX database:

% \begin{thebibliography}{00}

%% \bibitem must have the following form:
%%   \bibitem{key}...
%%

% \bibitem{}

% \end{thebibliography}


\end{document}

%%
%% End of file 